\documentclass[]{exam}
\usepackage{amsmath,amssymb}
\usepackage{inputenc}
\pagestyle{headandfoot}
\usepackage[all,cmtip]{xy}
\DeclareMathOperator{\coker}{coker}
\inputencoding{utf8}


%opening
\title{Questionario asignatura Topoogía de Variedades}
\author{W. Pitsch}
\date{}
\begin{document}

\maketitle

%\begin{abstract}
%
%\end{abstract}

El objetivo de éste questionario es tener un retorno sobre el curso y preparar el curso del año que viene. Por favor contesten lo mas sinceramente las preguntas. Pueden contestar anónimamente o con sus nombres. Evidentemente vuestras respuestas no influirán en la nota final.

\begin{questions}

\question ¿Qué esperabas de la asignatura de Topología de Variedades?

Principalmente entender conceptos que han ido apareciendo de manera informal sobretodo en
asignaturas de Física, como el espacio tangente, el producto tensorial o las formas
diferenciales, además de una asignatura interesante puesto que la geometría me
atrae en general. 

\question ¿ Ha respondido la asignatura a lo que esperabas?

Esencialmente sí. 

\question¿ Has echado de menos algo en clase?

Por norma general las clases de teoría han sido muy provechosas aunque en ocasiones el
hablar de una variedad de temas sin demasiada profundidad (pienso en el cobordismo o las
construcciones con fibrados vectoriales, por ejemplo) hacía que se perdiese un poco el
hilo conductor.

En la mayoría de clases de problemas se trataban las listas antes que en teoría, lo que
hacía que fuesen de menos utilidad.

\question ¿Cuánto tiempo a la semana has dedicado al estudio de la asignatura fuera de las horas de clase? 

La clase de teoría de los lunes requería de unas dos horas semanales para interiorizarla
bien.


\question ¿Qué documento utilizas más para estudiar: apuntes de clase, fuentes bibliográficas, otros?

Los apuntes de clase y ocasionalmente algún otro libro (el texto de Geometría Diferencial
de Spivak o otros apuntes del tema de otras universidades) o clases grabadas en YouTube. 

\question ¿Crees que tu nivel de inglés sería suficiente para estudiar a partir de fuentes en és idioma : libros, vídeos, etc.?

Sí.

\question ¿ Como prefieres trabajar: en grupo, solo?

Las discusiones en grupo suelen ser provechosas pero la redacción de los problemas
prefiero hacerla solo.

\question ¿Cómo aprendes mejor las matemáticas ? (Estudio personal, escuchando en clase,
trabajando sobre apuntes, etc...) 

Por norma general respondo mejor a una buena explicación con visualizaciones y ejemplos.

\question ¿Cómo crees que puedes mejor mostrar lo que has aprendido? 

A parte de un examen tradicional, también podría ser útil un examen oral que evaluase la
comprensión de los conceptos más que los detalles de cálculo.

\end{questions}

\end{document}
