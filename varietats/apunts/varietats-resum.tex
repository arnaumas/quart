\documentclass[12pt,twocolumn]{article}

% PREAMBLE FOR ALGEBRAIC TOPOLOGY SUMMARY

% ----------------------------------------------------------
% Packages
\usepackage[utf8]{inputenc}
\usepackage[T1]{fontenc}
\usepackage[catalan]{babel}
\usepackage{lmodern}
\usepackage{geometry}
\usepackage{hyperref}
\usepackage[dvipsnames]{xcolor}
\usepackage[bf,sf,small,pagestyles]{titlesec}
\usepackage{titling}
\usepackage[font={footnotesize, sf}, labelfont=bf]{caption} 
\usepackage{siunitx}
\usepackage{graphicx}
\usepackage{tikz-cd}
\usetikzlibrary{babel}
\usepackage{booktabs}
\usepackage{amsmath,amssymb}
\usepackage[sort]{cleveref}
\usepackage{amsthm,thmtools}
\usepackage[shortlabels]{enumitem}

% ----------------------------------------------------------
% Geometry setup
\geometry{
	a4paper,
	right = 3cm,
	left = 3cm,
	bottom = 3cm,
	top = 3cm
}
% Wider space between lines
\renewcommand{\baselinestretch}{1.3}

% ----------------------------------------------------------
% Hyperref setup
\hypersetup{
	colorlinks,
	linkcolor = {red!50!blue},
	linktoc = page
}
\numberwithin{table}{section}
\numberwithin{equation}{section}
\numberwithin{figure}{section}

% ----------------------------------------------------------
% Definition of theorem, example, etc environments
\newcommand{\qedtriangle}{\ensuremath{\triangle}}
\newcommand{\qedtriangledown}{\ensuremath{\bigtriangledown}}
\declaretheoremstyle[spaceabove=6pt, spacebelow=6pt, headfont=\bfseries\sffamily, notefont=\normalfont, notebraces={(}{)}, qed=\qedtriangle]{definition}
\declaretheoremstyle[spaceabove=6pt, spacebelow=6pt, headfont=\bfseries\sffamily, notefont=\normalfont, notebraces={(}{)}, qed=\qedtriangledown]{example}
\declaretheoremstyle[spaceabove=6pt, spacebelow=6pt, headfont=\bfseries\sffamily,
notefont=\normalfont, notebraces={(}{)}]{theorem}


\declaretheorem[name=Teorema, style=theorem, refname={teorema,teoremes},
Refname={Teorema,Teoremes}, numberwithin=section]{theorem}
\declaretheorem[name=Proposició, style=theorem, refname={proposició,proposicions},
Refname={Proposició,Proposicions}, numberlike=theorem]{proposition}
\declaretheorem[name=Lema, style=theorem, refname={lema,lemes},
Refname={Lema,Lemas}, numberlike=theorem]{lema}
\declaretheorem[name=Definició, style=definition, refname={definició,definicions},
Refname={Definició,Definicions}, numberwithin=section]{definition}
\declaretheorem[name=Exemple, style=example, refname={exemple,exemples},
Refname={Exemple,Exemples}, numberwithin=section]{example}
\declaretheorem[name=Observació, style=example, refname={observació,observacions},
Refname={Observació,Observacions}, numberwithin=section]{observation}

% ----------------------------------------------------------
% Definition of custom list style
\newlist{points}{enumerate}{1}
\setlist[points,1]{label=\textup{(}{\itshape \roman*}\textup{)}, wide}

% ----------------------------------------------------------
% Definition of page styles
\newpagestyle{page}[\sffamily \footnotesize]{
	\headrule
	\sethead{}{\bfseries{Topologia de Varietats}}{}
	\footrule
	\setfoot{}{}{\thepage}
}
\pagestyle{page}

% CUSTOM COMMANDS FOR ALGEBRAIC TOPOLOGY
% ----------------------------------------------------------

% Restriction of a function
\newcommand{\rest}[1]{\raisebox{-.5ex}{$|$}_{#1}}

% Real numbers
\newcommand{\R}{\mathbb{R}}
\newcommand{\PR}{\mathbb{PR}}

% Rational numbers
\newcommand{\Q}{\mathbb{Q}}

% Complex numbers
\newcommand{\C}{\mathbb{C}}

% Natural numbers
\newcommand{\N}{\mathbb{N}}

% Integers
\newcommand{\Z}{\mathbb{Z}}

% Vector bold
\renewcommand{\vec}[1]{\mathbf{#1}}

% Span
\newcommand{\gen}[1]{\langle #1 \rangle}

% Set
\newcommand{\set}[1]{\{ #1 \}}

% Script A, B, M, P
\newcommand{\A}{\mathcal{A}}
\newcommand{\B}{\mathcal{B}}
\newcommand{\M}{\mathcal{M}}
\renewcommand{\P}{\mathcal{P}}
\renewcommand{\S}{\mathfrak{S}}

% Identity
\newcommand{\id}{\mathrm{id}}

% Kernel and image
\DeclareMathOperator{\im}{im}
\DeclareMathOperator{\coker}{coker}

% Absolute value
\newcommand{\abs}[1]{\lvert #1 \rvert}

% Norm
\newcommand{\norm}[1]{\lVert #1 \rVert}

% Category of Vector Spaces
\newcommand{\Vect}{\mathsf{Vect}}
\newcommand{\VectK}{\Vect_{K}}
\newcommand{\VectR}{\Vect_{\R}}
\DeclareMathOperator{\Hom}{Hom}
\DeclareMathOperator{\Bil}{Bil}
\newcommand{\dual}{^{\vee}}
\DeclareMathOperator{\tr}{tr}

% Category of Manifolds
\DeclareMathOperator{\Diff}{Diff}

% Epi and monomorphisms
\newcommand{\onto}{\twoheadrightarrow}
\newcommand{\into}{\tailrightarrow}

\newcommand{\parbreak}{
	\begin{center}
		--- $\ast$ ---
	\end{center} 
}

% Defined as
\makeatletter
\newcommand*{\defeq}{\mathrel{\rlap{%
    \raisebox{0.3ex}{$\m@th\cdot$}}%
  \raisebox{-0.3ex}{$\m@th\cdot$}}%
	=
}
\makeatother

% Support
\DeclareMathOperator{\supp}{supp}

% Categories
\newcommand{\Top}{\mathsf{Top}}


\title{\bfseries \sffamily Topologia de Varietats: Resultats més rellevants}
\author{\sffamily Arnau Mas}
\date{\sffamily 2020}

\begin{document}
\maketitle

\section{Varietats}
\subsection{Definicions}
\begin{definition}[Carta]
	Una \emph{carta} al voltant d'un punt \( x \in M \) és un entorn obert \( U \) d'\( x \)
	i un homeomorfisme \( \phi \colon U \to \phi(U) \subseteq \R^n \). 
\end{definition}

\begin{definition}[Varietat]
	Un espai topològic \( M \) és una \emph{varietat} si
	\begin{points}
	\item \( M \) és Hausdorff i segon numerable,
	\item hi ha un conjunt de cartes \( (U_\alpha, \phi_\alpha) \) tals que \(
		\set{U_\alpha}_\alpha \) és un recobriment de \( M \) i si \( U_\alpha \cup U_\beta
		\neq \emptyset \) aleshores l'aplicació canvi de carta 
		\begin{equation*}
			\phi_\alpha^\beta \colon \phi_\alpha(U_\alpha \cap U_\beta)
			\xrightarrow{\phi_\alpha^{-1} \circ \phi_\beta} \phi_\beta(U_\alpha \cap U_\beta)
		\end{equation*}
		és un homeomorfisme. El conjunt d'aquestes cartes s'anomena un \emph{atles}.
	\end{points}

	Quan totes les aplicacions de canvi de carta són de classe \( C^k \) parlem d'una
	varietat de classe \( C^k \). En el cas \( C^\infty \) diem que la varietat és
	\emph{llisa}. En el cas \( C^0 \) se sol parlar de varietat topològica.
\end{definition}

\begin{example}
	\( \R^k \) és una varietat de dimensió \( k \) amb l'atles \( \set{(\R^k, \id)}	\).
\end{example}

\begin{example}[Projecció estereogràfica]
	\( S^n \defeq \set{x \in \R^{n+1} \mid \norm{x} = 1} \) és una varietat de
	dimensió \( n \). Un atles és la projecció estereogràfica:
	\begin{align*}
		\phi_\delta \colon S^n - p_\delta & \to \R^n \\
		(x_1, \dots, x_{n+1}) & \mapsto \frac{\delta}{\delta - x_{n+1}}(x_1, \dots, x_n)
	\end{align*}
	i
	\begin{align*}
		\phi_\delta^{-1} \colon  \R^n & \to S^n - p_\delta \\
		(y_1, \dots, y_n) & \mapsto \frac{1}{\norm{y}^2 + 1}(2y, \delta(\norm{y}^2 - 1))
	\end{align*}
	on \( p_\delta = (0,\dots,0,\delta) \) per \( \delta = 1 \) i \( \delta = -1 \).
\end{example}

\begin{definition}[Varietat amb frontera]
	És la mateixa definició que varietat però ara les cartes tenen imatge al semiespai \(
	[0,\infty) \times \R^{n-1} \). El conjunt de punts tals que la seva imatge per alguna
	carta és a \( 0 \times \R^{n-1} \) és la frontera de la varietat, \( \partial M \). 
\end{definition}
\begin{observation}
	Es pot demostrar que si la imatge per alguna carta d'algun punt és a \( 0 \times
	\R^{n+1} \) també ho és per qualsevol altra carta, per tant que la frontera d'una
	varietat està ben definida. 
\end{observation}

\subsection{Aplicacions llises}
\begin{definition}[Funció llisa]
	Una aplicació \( f \colon M \to \R^k \) es diu \emph{llisa} si per tota carta \(
	(U,\phi) \) de \( M \) l'aplicació \( \phi^{-1} \circ f \colon \phi(U) \subseteq \R^n
	\to \R^k \) és de classe \( C^\infty \).
\end{definition}

\begin{definition}[Aplicació llisa]
	Una aplicació \( f \colon M \to N \) entre varietats és una \emph{funció llisa} si per
	tota carta \( (U,\phi) \) de \( M \) i carta \( (V,\psi) \) de \( N \) tal que \( f(U)
	\subseteq V \) l'aplicació \( \phi^{-1} \circ f \circ \psi \colon \phi(U) \to \psi(V) \)
	és de classe \( C^\infty \). El conjunt de funcions llises d'\( M \) s'escriu \(
	C^\infty(M) \) o \( \Omega^0(M) \) (és el conjunt de les 0-formes de \( M \)).
\end{definition}
\begin{example}
	\begin{points}
	\item Les funcions constants són llises.
	\item Les components d'una carta, és a dir \( \phi_i = \pi_i \circ \phi \) són llises. 
	\end{points}
\end{example}

\begin{definition}[Difeomorfisme]
	Una aplicació llisa \( f \colon M \to N \) és un \emph{difeomorfisme} si és invertible
	i la inversa	\( f^{-1} \colon N \to M \) és llisa. El conjunt d'aplicacions llises
	entre \( M \) i \( N \) s'escriu \( C^\infty(M,N) \).
\end{definition}

\begin{observation}
	\begin{points}
	\item La composició d'aplicacions llises és llisa.
	\item La composició de difeomorfismes és un difeomorfisme, per tant els difeomorfismes
		d'una varietat en ella mateixa formen un grup, \( \Diff(M) \).
	\end{points}
\end{observation}

\subsection{Subvarietats}
\begin{definition}[Subvarietat]
	Donat \( N \subseteq M \) (\( \dim M = m \)), diem que una carta \( (U,\phi) \) d'\( M
	\) és una linearització local d'\( N \) si \( \phi(U \cap N)  \) és difeomorfa a \( U'
	\cap (\R^n \times 0) \) on \( U' \) és un obert de \( R^n \).  Si \( N \) es pot
	recobrir amb linearitzacions locals diem que és una subvarietat de \( M \). Aquest
	recobriment indueix un atles per a \( N \): si \( (U_\alpha, \phi_\alpha) \) és una de
	les cartes del recobriment aleshores \( (U_\alpha \cap N, (\pi_1 \times \dots \pi_n)
	\circ \phi_\alpha \rest{N}) \) és una carta de \( N \). 
\end{definition}

\begin{example}
	\begin{points}
	\item	Qualsevol obert \( U \) d'una varietat \( M \) n'és una subvarietat de la mateixa
		subvarietat. 
	\item \( S^n \) és una subvarietat de \( \R^{n+1} \). 
	\end{points}
\end{example}

\subsection{Construccions bàsiques}
\begin{definition}[Producte]
	El producte de dues varietats \( M \times N \) té una estructura diferenciable natural:
	si \( \set{(U_\alpha, \phi_\alpha)}_{\alpha \in I} \) és l'atles d'\( M \) i \(
	\set{(V_\beta, \psi_\beta)}_{\beta \in J} \) l'atles d'\( N \) aleshores l'atles d'\( M
	\times N \) és \( \set{(U_\alpha \times V_\beta, \phi_\alpha \times \psi_\beta)}_{\alpha
	\in I, \beta \in J} \). Les projeccions \( \pi_M \colon M \times N \onto M \) i \( \pi_N
	\colon M \times N \onto N \) són llises. 
\end{definition}

\begin{definition}[Quocient per l'acció d'un grup]
	Si un grup discret \( G \) actua sobre \( M \) ---és a dir, hi ha un morfisme \( \rho
	\colon G \to \Diff(M) \)--- aleshores el quocient \( M/G \) és una varietat amb la
	mateixa dimensió que \( M \) si l'acció és lliure (si \( \phi(g) \) té punts fixos
	aleshores \( \phi(g) = \id_M \)) i pròpia (per tot compacte \( K \subseteq M \) el
	conjunt \( \set{g \in G \mid \rho(g)(K) \cap K \neq \emptyset} \) és finit). En aquest
	cas la projecció \( \pi \colon M \onto M/G \) és llisa.
\end{definition}

\subsection{Classificació de varietats compactes en dimensió petita}
\begin{itemize}
	\item \emph{Dimensió 0.} Són unions finites de punts.
	\item \emph{Dimensió 1.} Unions disjuntes finites de \( S^1 \) (sense frontera) i de \(
		[0,1] \)
	\item \emph{Dimensió 2.} En el cas sense frontera són sumes connexes de \( S^2 \), el
		torus \( S^1 \times S^1 \) (cas orientable) i de \( \PR^2 \) (cas no orientable).
\end{itemize}

\section{Espai Tangent}
\subsection{Vectors tangents}
\begin{definition}[Vector tangent]
	Un vector tangent en el punt \( x \in M \) és una aplicació \( X \colon \Omega^0(U) \to
	\R \) on \( U \) és un entorn de \( X \) que
	\begin{points}
	\item és lineal,
	\item satisfà la regla de Leibniz, és a dir, per \( f, g \in \Omega^0(U) \),
		\begin{equation*}
			X(fg) = X(f)g(x) + f(x)X(g),
		\end{equation*}
	\item si \( U_1 \subseteq U \) i \( U_2 \subseteq U \) són entorns oberts de \( X \)
		tals que \( f\rest{U_1 \cap U_2} = g \rest{U_1 \cap U_2} \) aleshores \( X(f) = X(g)
		\), o, dit d'un altra manera, \( X \) es pot traslladar a una funció \( \mathcal{O}_x
		\to \R \) on \( \mathcal{O}_x \) és el resultat d'identificar a \( \Omega^0(M) \) les
		funcions que coincideixen en un entorn de \( x \), el \emph{germen} de funcions
		diferenciables en \( x \).  
	\end{points}
\end{definition}
La intució és que \( X(f) \) és la derivada de \( f \) en la direcció \( X \) en el punt \( x \). 
\begin{definition}[Espai tangent]
	El conjunt de vectors tangents a un punt \( x \) és un espai vectorial de la mateixa
	dimensió que \( M \) i s'escriu \( T_x M \).
\end{definition}

\begin{definition}[Base induïda per una carta]
	Una carta \( (U,\phi) \) al voltant del punt \( x \) indueix una base de \( T_xM \):
	\begin{align*}
		\partial_i^\phi \colon \Omega^0(U) & \to \R \\
		f & \mapsto \partial_i(f \circ \phi^{-1})(0)
	\end{align*}

	Un vector tangent s'expressa en aquesta base com
	\begin{equation*}
		X = \sum_{i = 1}^{n} X(\phi_i)\partial_i^{\phi} 
	\end{equation*}
	on \( \phi_i = \pi_i \circ \phi \). 
\end{definition}

\begin{definition}[Aplicació tangent]
	Donada una aplicació llisa \( f \in C^{\infty}(M,N) \), l'aplicació lineal tangent a \(
	f \) en \( x \) (diferencial d'\( f \) en \( x \), pushforward d'\( f \) en \( x \)), \(
	T_x f \) és
	\begin{align*}
		T_x f \colon T_xM & \to T_{f(x)} N \\
		X & \mapsto (g \mapsto X(g \circ f)).
	\end{align*}
	Escollint la carta \( (U,\phi) \) al voltant d'\( x \) i \( (V,\psi) \) al voltant de \(
	f(x) \) es té
	\begin{equation*}
		T_xf(\partial_i^{\phi}) = \sum_{j = 1}^{m} \partial_i(\psi_j \circ f \circ
		\phi^{-1})(\phi(x)) \partial_j^{\psi} 
	\end{equation*}
	és a dir, que en les bases induïdes per \( \phi \) i \( \psi \), la matriu de \( T_f \)
	és la matriu de de la diferencial de \( \psi \circ f \circ \phi^{-1} \). 
\end{definition}

\subsection{El fibrat tangent}
\begin{definition}[Fibrat tangent]
	El fibrat tangent \( TM \) és el conjunt \( \bigsqcup_{x \in M}T_xM \). \( TM \) és una
	varietat de dimensió \( 2 \dim M \). Donada una carta \( (U,\phi) \) d'\( M \),
	s'indueix una carta \( (TU, \Phi) \) on
	\begin{points}
	\item	\( TU \defeq \pi^{-1}(U) \) on \( \pi \colon TM \onto M \) és la projecció
	\item \( \Phi \) es defineix com
		\begin{align*}
			\Phi \colon TU & \to \phi(U) \times \R^n \\
			X & \mapsto (\phi(\pi(X)), X(\phi_1), \dots, X(\phi_n)).
		\end{align*}
	\end{points}
\end{definition}

\begin{definition}[Aplicació tangent]
	Donada \( f \in C^{\infty}(M,N) \) es defineix \( Tf \colon TM \to TN \) com \( Tf(X)
	\defeq T_{\pi(X)}(X) \), que fa commutar el diagrama
	\begin{equation*}
		\begin{tikzcd}
			TM \arrow[d, two heads, "\pi"] \arrow[r, "Tf"] & TN \arrow[d, two heads, "\pi", two
			heads, "\pi"] \\
			M \arrow[r, "f"] & N
		\end{tikzcd}
	\end{equation*}
\end{definition}

\begin{proposition}
	El fibrat tangent és functorial, és a dir, \( T(f \circ g) = Tf \circ Tg \)  i \( T
	\id_M = \id_{TM} \). En particular, si \( f \) és un difeomorfisme aleshores \( Tf \)
	també i \( T(f^{-1}) = (Tf)^{-1} \).
\end{proposition}

\begin{proposition}
	Si \( N \) és una subvarietat d'\( M \) aleshores \( TN \) és una subvarietat de \( TM \).
\end{proposition}

\subsection{Immersions i submersions. Transversalitat}
\begin{definition}[Submersió i immersió]
	Una aplicació llisa \( f \in C^{\infty}(M,N) \) és una \emph{immersió }si per tot \( x \in M \)
	\( T_xf \) és injectiva. Si en canvi \( T_xf \) és exhaustiva per tot \( x \in M \), \(
	f\) és una \emph{submersió}.  
\end{definition}
\begin{observation}
	Un difeomorfisme local, és a dir, \( f \in C^{\infty}(M,N) \) tal que tot punt té un
	entorn \( U \) de manera que \( f\rest{U} \) és un difeomorfisme entre \( U \) i \( f(U)
	\) són submersions i immersions. Al revés també és veritat: si \( T_xf \) és invertible,
	aleshores hi ha un entorn \( U \) d'\( x \) tal que \( f \) és un difeomorfisme entre \(
	U \) i \( f(U) \) (Teorema de la Funció Inversa).
\end{observation}

\begin{definition}[Punt crític i punt regular]
	Donada \( f \in C^{\infty}(M,N) \), un punt \( x \in M \) és \emph{regular} si \( T_xf
	\) és exhaustiva, i \emph{crític} si no és regular, és a dir, si \( T_xf \) és
	exhaustiva. 

	Un punt \( y \in N \) és un \emph{valor regular} si i només si tot element de \(
	f^{-1}(y) \) és regular. En canvi \( y \) és un \emph{valor crític} si no és regular, és
	a dir, si hi ha algun punt de \( f^{-1}(y) \) crític.  
\end{definition}
\begin{observation}
	Un valor regular no ha de ser un valor, de fet, tot punt de \( N - f(M) \) és un valor
	regular. En canvi els valors crítics sempre són valors, i de fet el conjunt de valors
	crítics és exactament la imatge del conjunt de punts crítics.
\end{observation}

\begin{theorem}[Sard]
	El conjunt de valors crítcs té mesura nu\l.la al codomini, o, equivalentment, el conjunt
	de valors regulars és dens al codomini. 
\end{theorem}

\begin{theorem}[Forma local de les immersions i les ubmersions] 
	Sigui \( M \) una varietat de dimensió \( m \) i \( N \) una varietat de dimensió \( n
	\). Si \( T_xf \) és injectiva (per tant \( m \leq n \)), hi ha una carta \( (U,\phi) \)
	al voltant d'\( x \) i una carta \( (V,\psi) \) al voltant d'\( f(x) \) tal que \( \psi
	\circ f \circ \phi^{-1} = \iota_\mathrm{can} \) on \( \iota_\mathrm{can} \) és la
	immersió canònica de \( \R^m \) en \( \R^m \). Si \( T_xf \) és exhaustiva (per tant \(
	m \geq n \)), es té \( \psi \circ f \circ \phi^{-1} = \sigma_\mathrm{can} \) on \(
	\sigma_\mathrm{can} \) és la submersió canònica de \( \R^m \) a \( \R^n \).
\end{theorem}

\begin{theorem}
	Si \( f \colon M \to N \) és llisa i \( y \in N \) és un valor regular aleshores \(
	Z = f^{-1}(y) \) és una subvarietat de \( M \) amb
	\begin{equation*}
		\dim Z = \dim M - \dim N
	\end{equation*}
	i a més \( T_x Z = \ker T_xf \).
\end{theorem}

\begin{example}
	Amb això apareixen moltes subvarietats de \( M_n(\R) \):
	\begin{points}
	\item \( GL_n(\R) \). \( GL_n(\R) \) és la preimatge de \( \R - 0 \) per \( \det \), per
		tant obert i per tant subvarietat de dimensió \( \dim M_n(\R) = n^2 \). 
	\item \( SL_n(\R) \). \( SL_n(\R) \) és \( \det^{-1}(1) \). En general, si \( B \in
		GL_n(\R) \) es té
		\begin{equation*}
			T_B \det (A) = \det(B) \tr(B^{-1}A)
		\end{equation*}
		per tant és exhaustiva per tot \( B \in GL_n(\R) \). En particular ho és per \(
		SL_n(\R) \), de manera que \( SL_n(\R) \) és una subvarietat de dimensió \( n^2 - 1
		\) i \( T_B SL_n(\R) = \set{A \in M_n(\R) \mid \tr(B^{-1}A) = 0} \).
		
	\item \( O(n) \). Per definició \( O(n) = \set{A \in M_n(\R) \mid A^\top A = 1} \).
		Considerem l'aplicació de simetrització:
		\begin{align*}
			s \colon M_n(\R) & \to S_n(\R) \\
			A & \mapsto A^\top A
		\end{align*}
	on \(  S_n(\R) = \set{B \in M_n(R) \mid B = B^\top} \cong \R^{\frac{n(n+1)}{2}} \).
	Aleshores \( T_As(B) = A^\top B + B^\top A \), que és exhaustiu a \( O(n) \):
	\begin{equation*}
		T_A s\left(\frac{1}{2}AC\right) = C
	\end{equation*}
	amb \( A \in O(n) \) i \( C \in S_n(\R) \). Per tant \( O(n) = s^{-1}(1) \) és una
	subvarietat i 
	\begin{align*}
		\dim O(n) & = \dim M_n(\R) - \dim S_n(\R) \\
							& = n^2 - \frac{n(n+1)}{2} \\ 
							& = \frac{n(n-1)}{2}.
	\end{align*}
	\end{points}
\end{example}

\begin{definition}[Transversalitat]
	Diem que \( f \colon M \to N \) és transversa a una subvarietat \( Y \subseteq N \) si
	per tot \( y \in Y \) es té
	\begin{equation*}
		\im T_x f + T_{f(x)} Y = T_x N.
	\end{equation*}
	(Intuïtivament, que \( f(M) \) i \( N \) no són tangents)
\end{definition}

\begin{theorem}
	Si \( f \colon M \to N \) és transversa a una subvarietat \( Y \subseteq N \) aleshores
	\( f^{-1}(Y) \) és una subvarietat de \( M \) i
	\begin{equation*}
		\dim f^{-1}(Y) = \dim M - \dim N + \dim Y.
	\end{equation*}
\end{theorem}

\subsection{Camps vectorials}
\begin{definition}[Camp vectorial]
	Un camp vectorial és una secció llisa de la projecció \( TM \xrightarrow{\pi} M \), és a
	dir, una aplicació \( X \colon M \to TM \) tal que \( \pi \circ X = \id M \).
	
	Equivalentment, un camp vectorial és una derivació de \( \Omega^0(M) \), és a dir, una
	aplicació \( X \colon \Omega^0(M) \to \Omega^0(M) \) lineal i que satisfà la regla de
	Leibniz. 
\end{definition}
\begin{observation}
	\begin{points}
	\item En un obert de carta \( U \), un camp vectorial sobre una varietat de dimensió \(
		n \) ve donat per \( n \) funcions \( X^i \colon U \to \R \), i definint \( X(x) =
		\sum_{i = 1}^{n} X^i(x) \partial^\phi_i \). Globalment això només es pot fer per
		varietats para\l.lelitzables, tals que \( TM \cong M \times \R^n \), en les que hi ha
		\( n \) camps linealment independents en tot punt. 
	\item El conjunt de camps vectorials és un \( \R \)-espai vectorial, en general de
		dimensió finita, i un	\( \Omega^0(M) \)-mòdul.

	\item En general la composició de camps vectorials (entesos com a derivacions) no és un
		camp vectorial, però si que ho és el commutador \( [X,Y] = XY - YX \), que dóna
		estructura d'àlgebra de Lie: és una operació bilineal, antisimètrica i que satisfà la
		identitat de Jacobi
		\begin{equation*}
			[X,[Y,Z]] + [Y,[Z,X]] + [Z,[X,Y]] = 0.
		\end{equation*}
		Localment es té
		\begin{equation*}
			[X,Y]^i = X^j \partial_j Y^j - Y^j \partial_j X^i.
		\end{equation*}
	\end{points}
\end{observation}

\parbreak

\begin{theorem}[Flux d'un camp]
	Donat un camp vectorial \( X \) sobre \( M \) existeix un obert \( W \subseteq M \times
	\R \) i una aplicació llisa \( \gamma \colon W \to M \) tal que \( W \) és un entorn de
	\( (x,0) \) per tot \( x \in M \) i a \( W \) es té
	\begin{equation*}
		\partial_t \gamma(x,t) = X(\gamma(x,t))
	\end{equation*}
	i \( \gamma(x,0) \), és a dir, \( \gamma \) és solució del problema de Cauchy.

	Si \( M \) és compacta i sense frontera aleshores \( W = M \times \R \) i en particular
	per tot \( t \in \R \), \( \gamma(\cdot, t) = \gamma_t \) és un difeomorfisme d'\( M \). 
\end{theorem}

\begin{definition}[Derivada de Lie]
	Donat el camp vectorial \( X \) i \( \gamma \) el seu flux, definim la derivada de Lie
	de \( f \Omega^0(M) \) com
	\begin{equation*}
		\mathcal{L}_X(f)(x) = \lim_{t \to 0} \frac{f(\gamma_t(x)) - f(x)}{t}
	\end{equation*}
	(la intuició és que \( \gamma_t(x) \) és com \( x + \epsilon \): \( x \) desplaçat una
	mica en la direcció que marca el camp	\( X \)).
\end{definition}
Això dóna un isomorfisme entre els camps vectorials entesos com a seccions i els camps
entesos com a derivacions de \( \Omega^0(M) \). 

\section{Àlgebra multilineal}
\subsection{Producte tensorial}
\begin{definition}[Producte tensorial]
	El producte tensorial de dos espais vectorials \( V \) i \( W \) és l'espai vectorial \(
	V \otimes W \) amb la propietat de que per tota aplicació bilineal \( \beta \colon V
	\times W \to Z \) hi ha una única aplicació lineal \( \hat{\beta} \colon V \otimes W \to
	Z \) tal que \( \beta = \hat{\beta} \circ \iota \), on \( \iota \colon V \times W \to Z
	\) és la inclusió canònica (que és part de la definició). Així \( V \otimes W \)
	realitza l'isomorfisme
	\begin{equation*}
		\Hom(V, \Hom(W, Z)) \cong \Hom(V \otimes W, Z).
	\end{equation*}
	

	En termes concrets, \( V \otimes W \) te dimensió \( \dim V \dim W \) i donades bases
	\( \set{e_i}_{i=1}^{\dim V} \) i \( \set{f_j}_{j=1}^{\dim W} \) té com a base
	\begin{equation*}
		\set{e_i \otimes f_j}_{i = 1, j = 1}^{i = \dim V, j = \dim W}
	\end{equation*}
	amb les propietats
	\begin{points}
	\item \( \lambda(v \otimes w) = (\lambda v) \otimes w = v \otimes (\lambda w) \)
	\item \( v \otimes (w_1 + w_2) = v \otimes w_1 + v \otimes w_2 \)
	\item \( (v_1 + v_2) \otimes v = v_1 \otimes w +  v_2 \otimes w \)
	\end{points}
\end{definition}
\begin{observation}
	Els tensors (elements del producte tensorial) de la forma \( v \otimes w \) amb \( v \in
	V	\) i \( w \in W \) s'anomenen tensors elementals, però no tots els tensors són
	elementals.
\end{observation}
\begin{observation}
	Es tenen els isomorfismes
	\begin{points}
	\item \( V \otimes W \cong W \otimes V \)
	\item \( V \otimes (W \otimes U) \cong (V \otimes W) \otimes U \)
	\item \( \R \otimes V \cong V \)
	\item \( \gen{0} \otimes V \cong \gen{0} \)
	\item \( V \otimes V^\ast \cong \Hom(V,V) \)
	\end{points}
\end{observation}

\subsection{Àlgebra tensorial}
Denotem el producte tensorial de \( V \) amb ell mateix \( n \) vegades per \( T^n(V) \).
Aleshores tenim una aplicació bilineal
\begin{equation*}
	\otimes \colon T^n(V) \times T^m(V) \to T^{n+m}(V).
\end{equation*}
Agafem la ``clausura'' de tots aquests espais,
\begin{equation*}
	T(V) \defeq \bigoplus_{n = 0}^\infty T^n(V)
\end{equation*}
que és un espai vectorial amb un producte \( \otimes \colon T(V) \times T(V) \to T(V) \), per
tant una àlgebra, que s'anomena l'\emph{àlgebra tensorial}. 

\subsection{Àlgebra alternada}
Tenim una acció del grup simètric \( \S_n \) sobre \( T^n(V) \), que sobre els tensors
elementals és
\begin{equation*}
	\sigma(v_1 \otimes \cdots \otimes v_n) = v_{\sigma(1)} \otimes \dots \otimes
	v_{\sigma(n)}. 
\end{equation*}
Això dóna lloc a dos subespais rellevants, el dels tensors simètrics, \( S^n(V) =
T^n(V)^{\S_n} \) i el dels tensors antisimètrics:
\begin{align*}
	\bigwedge^n V & \defeq \set{x \in T^n(V) \mid \\
						 & \forall \sigma \in \S_n \colon \sigma x =
	(-1)^{\sigma} x}. 
\end{align*}
Podem ``antisimetritzar'' un tensor definint
\begin{equation*}
	v_1 \wedge \cdots \wedge v_n \defeq \frac{1}{n!} \sum_{\sigma \in \S_n} v_{\sigma(1)}
	\otimes \cdots \otimes v_{\sigma(n)}
\end{equation*}
i aleshores si \( e_i \) és una base de \( V \) els tensors \( e_{i_1} \wedge \cdots \wedge
e_{i_n} \) formen una base de \( \bigwedge^nV \), que es demostra que té dimensió \( \binom{\dim
V}{n} \) quan \( n \leq \dim V \) i dimensió 0 sino. Aleshores es tenen els isomorfismes
\begin{points}
\item \( \bigwedge^{\dim V - n}V \cong \bigwedge^n V\)
\item \( \bigwedge^0 V \cong \R \), per tant \( \bigwedge^{\dim V} V \cong V \)
\item \( \bigwedge^{1} V \cong V \)
\end{points}

L'operació \( \wedge \) extesa a tot \( \bigwedge^n V \) dóna una aplicació bilineal alternada 
\begin{equation*}
	\wedge \colon \bigwedge^n V \times \bigwedge^mV \to \bigwedge^{n+m}V.
\end{equation*}
Aleshores, igual com amb el producte tensorial, definim l'àlgebra alternada com
\begin{equation*}
	\bigwedge V \defeq \bigotimes_{n = 0}^{\dim V} \bigwedge^n V
\end{equation*}
que és una subàlgebra de \( T^n(V) \). 

L'utilitat del producte alternat és que l'espai \( \bigwedge^n V^\ast \) és l'espai de les formes
multilineals alternades, i és el que ens permetrà definir les formes diferencials sobre una
varietat. 

\section{Fibrats vectorials}
El concepte generalitza el del fibrat tangent. 
\begin{definition}[Fibrat vectorial]
	Un fibrat vectorial (de rang \( n \)) és un triple \( E \xrightarrow{\pi} B \) on \( E
	\) i \( B \) són varietats (diferenciables) i \( \pi \) una aplicació llisa exhaustiva
	de manera que
	\begin{points}
	\item per tot \( x \in B \) la fibra \( \pi^{-1}(x) \) és un espai vectorial,
	\item \( E \) és localment trivial, és a dir, hi ha un recobriment de \( B \) per oberts
		\( U_\alpha \) amb difeomorfismes \( h_\alpha \colon \pi^{-1}(U) \to U \times \R^n \)
		que fan commutar
		\begin{equation*}
			\begin{tikzcd}
				E \arrow[d, two heads, "\pi"] & \arrow[l, tail] \arrow[d, two heads, "\pi"]
				\pi^{-1}(U_\alpha) \arrow[r, "h_\alpha"] & U_\alpha \times \R^n \arrow[dl, two heads, "\pi_1"] \\
				B & \arrow[l, tail] U_\alpha
			\end{tikzcd}
		\end{equation*}
		i tal que \( h_\alpha \rest{\pi^{-1}(x)} \) és un isomorfisme entre la fibra \(
		\pi^{-1}(x) \) i \( x \times \R^{n} \). 
	\end{points}
\end{definition}
És a dir, la part de fibrat fa referència a que l'espai localment és com un producte \( B
\times \R^n \) i \( \pi \) actua com una projecció del producte. La part vectorial és el
requeriment de que les fibres siguin espais vectorials. 
\begin{example}
	\begin{points}
	\item	El fibrat trivial de dimensió \( n \), \( M \times \R^n \xrightarrow{\pi_1} M \).
	\item El fibrat vectorial \( TM \xrightarrow{\pi} M \). 
	\item Sobre \( S^1 \) tenim el fibrat tangent, \( TS^1 \) que és trivial per ser \( S^1
		\) para\l.lelitzablem i la cinta de Möbius, que es pot interpretar com un fibrat. No
		són isomorfs. 
	\end{points}
\end{example}

\begin{definition}[Morfisme de fibrats vectorials]
	Un morfisme de fibrats vectorials \( E_1 \xrightarrow{\pi_1} B_1 \) i \( E_2
	\xrightarrow{\pi_2} B_2 \) és una parella d'aplicacions llises \( f \colon B_1 \to B_2
	\) i \( \hat{f} \colon E_1 \to E_2 \) tals que el diagrama
	\begin{equation*}
		\begin{tikzcd}
			E_1 \arrow[d, two heads, "\pi_1"]\arrow[r, "\hat{f}"] & E_2 \arrow[d, two heads, "\pi_2"] \\
			B_1 \arrow[r, "f"] & B_2
		\end{tikzcd}
	\end{equation*}
	i \( \hat{f}_x \defeq \hat{f}\rest{\pi_1^{-1}(x)} \) és una aplicació lineal entre les
	fibres \( \pi_1^{-1}(x) \) i \( \pi_2^{-1}(f(x)) \). 

	Si \( f \) i \( \hat{f} \) són difeomorfismes aleshores tenim un isomorfisme de
	fibrats.
\end{definition}


\begin{theorem}
	Si \( (f,\hat{f}) \) és un morfisme entre fibrats del mateix rang, \( f \) és un
	difeomorfisme i \( \hat{f}_x \) és un isomorfisme entre les fibres per tot \( x \)
	aleshores \( (f, \hat{f}) \) és un isomorfisme de fibrats.
\end{theorem}

\section{Formes diferencials}
\subsection{Definició}
Les construccions sobre espais vectorials es poden pujar a fibrats fibra a fibra, així
podem parlar del fibrat dual o de les potències alternes d'un fibrat. En el cas del fibrat
tangent tenim el \emph{fibrat cotangent,} \( T^\ast M \), i les seves potències
alternades, \( \bigwedge^p T^\ast M \cong (\bigwedge^p TM)^\ast \). 

\begin{definition}[Forma diferencial]
	Una \( p \)-forma diferencial és una secció llisa del fibrat \( \bigwedge^p T^\ast M \).
	El conjunt de les \( p \)-formes diferencials és \( \Omega^p(M) \). Localment una forma
	\( \omega \) ens dóna en cada punt una forma multilineal alternada sobre \( T_x M \),
	\begin{equation*}
		\omega_x(v_1, \dots, v_p).
	\end{equation*}
	També podem fer-la actuar globalment sobre camps vectorials:
	\begin{equation*}
		\omega(X_1, \dots, X_p)(x) = \omega_x(X_1(x), \dots, X_p(x)).
	\end{equation*}
\end{definition}
\begin{observation}
	\( \Omega^p(M) \) és un \( \R \)-espai vectorial i un	\( C^{\infty}(M) \)-mòdul, en els
	dos casos de dimensió, en general, infinita.	
\end{observation}
\begin{example}[Gradients]
	Donada una funció llisa \( f \colon M \to \R \), la seva aplicació tangent \( Tf \) és
	una 1-forma, que en aquest context se sol denotar \( df \). En concret, si tenim una
	carta \( (x^1, \dots,x^n) \) (en el context de les formes es fa servir aquesta notació),
	tenim, localment, les formes \( dx^1, \dots, dx^n \), que són una base de \( T_p^\ast M
	\) i per tant els seus productes alternats són una base de \( \bigwedge^p T^\ast M \), és
	a dir que localment una \( p \)-forma s'escriu
	\begin{equation*}
		\omega = \omega_{i_1, \dots, i_p} dx^{i_1} \wedge \dots \wedge dx^{i_p}.
	\end{equation*}
\end{example}

\subsection{Pullbacks i el complex de de Rham}
\begin{definition}[Pullback]
	El pullback d'una aplicació llisa \( f \colon M \to N \) és l'aplicació
	\begin{align*}
		f^\ast \colon \Omega^p(N) & \to \Omega^p(M) \\
		\omega & \mapsto f^\ast \omega
	\end{align*}
	on
	\begin{gather*}
		f^\ast \omega_x(v_1, \dots, v_n) \defeq \\
		 \omega_{f(x)}(T_xf(v_1) \dots, T_xf(v_n)) 
	\end{gather*}
\end{definition}
\begin{proposition}
	Es té que \( (f \circ g)^\ast = g^\ast \circ f^\ast \) i \( \id_M^\ast =
	\id_{\Omega^p(M)} \) per tant el pullback és un functor contravariant. 
\end{proposition}

A partir del producte exterior tenim un producte exterior de formes diferencials,
\begin{equation*}
	\wedge \colon \Omega^p(M) \times \Omega^q(M) \to \Omega^{p+q}(M) \\
\end{equation*}
que és bilineal, associatiu i semicommutatiu, és a dir, si \( \alpha \in \Omega^p(M) \) i
\( \beta \in \Omega^q(M)\) aleshores \( \alpha \wedge \beta = (-1)^{pq} \beta \wedge
\alpha \). També \( f^\ast(\alpha \wedge \beta) = f^\ast \alpha \wedge f^\ast \beta \). 

Amb això definim una generalització del gradient: la derivada exterior
\begin{equation*}
	d \colon \Omega^p(M) \to \Omega^{p+1}(M)
\end{equation*}
que localment és
\begin{equation*}
	\omega = d\omega_{i_1, \dots, i_p} \wedge dx^{i_1} \wedge \dots \wedge dx^{i_p}.
\end{equation*}

\begin{proposition}
	La derivada exterior compleix
	\begin{points}
	\item \( d(f^\ast \omega) = f^\ast d\omega \)
	\item \( d^2 = 0 \)
	\item \( d(\alpha + \beta) = d\alpha + d\beta \)
	\item \( d(\alpha \wedge \beta) = d\alpha \wedge \beta + (-1)^p \alpha \wedge d\beta \)
		on \( p \) és el rang \( d'\alpha \). 
	\end{points}
\end{proposition}

\begin{definition}[Complex de de Rham]
	El complex de de Rham és la successió
	\begin{equation*}
		\begin{tikzcd}[column sep = 10] 
			0 \arrow[r] & \Omega^0(M) \arrow[r, "d"] & \cdots \arrow[r, "d"] & \Omega^{\dim
			M}(M) \arrow[r] & 0
		\end{tikzcd}
	\end{equation*}
\end{definition}

\begin{definition}[Formes exactes i tancades]
	Una forma \( \omega \in \Omega^p(M) \) és \emph{tancada} si \( d\omega = 0 \) i és
	\emph{exacta} si hi ha \( \theta \in \Omega^{p-1}(M) \) tal que \( d\theta = \omega \).
	Tota forma exacta és tancada, però al revés no. El conjunt de formes tancades és \(
	Z^p(M) \subseteq \Omega^p(M) \) i el de les formes exactes és \( B^p(M) = d\Omega^{p-1}(M)
	\subseteq \Omega^p(M) \). Tots dos són subespais vectorials. 
\end{definition}

\begin{definition}[Grups de cohomologia de de Rham]
	El \( p \)-èssim grup de cohomologia de de Rham és el quocient
	\begin{equation*}
		H^p(M) = Z^p(M)/B^p(M).
	\end{equation*}
\end{definition}

\subsection{Formes amb suport compacte}
\begin{definition}[Suport]
	El \emph{suport} d'una forma és 
	\begin{equation*}
		\supp \omega \defeq \overline{\set{x \in M \mid \omega_x \neq 0}}
	\end{equation*}
	els punts on \( \omega \) no és la forma nu\l.la.
\end{definition}
\begin{observation}
	Es té
	\begin{points}
	\item \( \supp(\alpha + \beta) \subset \supp(\beta) \cup \supp(\alpha) \)
	\item \( \supp(\alpha \wedge \beta) = \supp(\alpha) \cap \supp(\beta) \)
	\end{points}
\end{observation}
\begin{definition}
	El conjunt de formes de suport compacte és \( \Omega_c(M) \). Els grups de cohomologia
	d'aquest complex s'escriuen \( H_c^p(M) \).
\end{definition}
\begin{observation}
	Śi \( M \) és compacta no hi ha diferència entre les formes amb suport compacte i les
	que no, perquè totes tenen suport compacte. 
\end{observation}

\section{Àlgebra homològica}
\subsection{Successions exactes}
\begin{definition}[Successió exacta]
	La successió \( A \xrightarrow{f} B \xrightarrow{g} C \) és exacta en \( B \) si \( \im
	f = \ker g \).
\end{definition}
\begin{observation}[Successió exacta curta]
	Dir que la successió
	\begin{equation*}
		0 \xrightarrow{f_0} A \xrightarrow{f_1} B \xrightarrow{f_2} C \xrightarrow{f_3} 0
	\end{equation*}
és exacta equival a dir que \( f_1 \) és injectiva, que \( f_2 \) és exhaustiva i que \(
\im f_1 = \ker f_2 \). 
\end{observation}

\subsection{Complexos de cocadenes}
\begin{definition}[Complex de cocadenes]
	Una successió \( \cdots \xrightarrow{d_{n-1}} X^n \xrightarrow{d_{n}} X^{n+1}
	\xrightarrow{d_{n+1}}	\cdots \) és un \emph{complex de cocadenes} si \( d_{n+1} \circ
	d_n = 0 \), per tant si \( \im d_n \subseteq d_{n+1} \). S'escriu \( (X^\bullet, d) \)
\end{definition}

\begin{definition}[Morfisme de complexos]
	Un morfisme entre els complexos \( (X^\bullet, d_X) \) i \( Y^\bullet, d_Y) \) és una
	família de morfismes \( f_n \colon X^n \to Y^n \) tals que \( f^{n+1} \circ d_X = d_Y
	\circ f^n	\). 
\end{definition}

\begin{definition}[Homotopia de complexos]
	Dos morfismes de complexos \( f, g \) entre \( (X^\bullet,d) \) i \( (Y^\bullet, d) \)
	són homòtops si hi ha un morfisme \( K \colon X^\bullet \to Y^{\bullet -1} \) tal que \(
	f^n - g^n = K^n \circ d_Y + d_X \circ K^{n+1} \). 
\end{definition}

\begin{proposition}
	Si \( f \) i \( g \) són dos operadors entre complexos homòtops aleshores són iguals en
	cohomologia. 
\end{proposition}

\subsection{Axiomes de la cohomologia}
Una teoria de cohomologia és una successió de functors contravariants \( H^n \colon \Top^2
\to \VectR \) i un morfisme de connexió \( \delta_{(X,A)} \colon H^n(A) \to
H^{n+1}(X,A) \) (\( H^n(A) \defeq H^n(A,\emptyset) \)) tals que   
\begin{points}
\item \emph{invariància homotòpica:} si \( f \) i \( g \) són homòtopes aleshores \( H^n(f) = H^n(g) \),,
\item hi ha una successió exacta
	\begin{equation*}
		\begin{tikzcd}[column sep = 15]
			0 \arrow[r] & H^0(X,A) \arrow[r] & H^0(X) \arrow[r] & H^0(A) \arrow[dll, "\delta",
			out = -10, in = 170, looseness = 1.2, overlay] \\
				& H^1(X,A) \arrow[r] & \cdots
		\end{tikzcd}
	\end{equation*}
\item \emph{successió de Mayer-Vietoris:} si \( X = U \cup V \) amb \( U \) i \( V \)
	oberts aleshores hi ha una successió exacta	
	\begin{equation*}
		\begin{tikzcd}[column sep = 5]
			0 \arrow[r] & H^0(X) \arrow[r] & H^0(U) \oplus H^0(V) \arrow[r] & H^0(U \cap V)
			\arrow[dll,	out = -20, in = 160, overlay] \\
									& H^1(X) \arrow[r] & \cdots
		\end{tikzcd}
	\end{equation*}
\item \emph{axioma de Milnor:} 
	\begin{equation*}
		H^n\left(\bigsqcup_{k}(X_k, A_k)\right) = \bigoplus_{k}H^n(X_k,A_k) 
	\end{equation*}
\item \emph{axioma de la dimensió:} \( H^0(\ast) \cong \R \) i \( H^n(\ast) \cong 0 \) per
	\( n \geq 1 \). 
	
\end{points}

\section{Orientabilitat i integració}
\begin{definition}[Orientabilitat]
	Una varietat és \emph{orientable} si té un atles compatible amb la seva estructura
	diferenciable pel qual la diferencial de cada aplicació de canvi de carta té determinant
	positiut. Una varietat amb un atles així és una varietat \emph{orientada}. 
\end{definition}

\begin{proposition}
	Una varietat és orientable si i només si té una forma volum, és a dir, una forma de \(
	\Omega^{\dim M}(M) \) que no s'anu\l.la en cap punt. 
\end{proposition}

\subsection{Integració de formes}
\begin{definition}[Partició de la unitat]
	Una partició de la unitat subjecte a un recobriment \( \set{U_\alpha}_\alpha \) de \( M
	\) és una família \( \phi_\alpha \colon U_\alpha \to [0,\infty) \) tals que \( \supp
	\rho_\alpha \subseteq U_\alpha \) i \( \sum_\alpha \rho_\alpha = 1 \) i tal que cada
	punt té un entorn que només interseca un nombre finit de \( \supp \rho_\alpha \). 
\end{definition}

\begin{definition}[Integral d'una forma]
	La integral d'una forma \( \omega \in \Omega^n_c(M) \) on \( M \) és una varietat
	orientada amb un atles \( \set{(U_\alpha, \phi_\alpha)}_\alpha \) és
	\begin{equation*}
		\int_M \omega \defeq \sum_{\alpha}  \int_M \phi_\alpha^\ast(\rho_\alpha \omega)
	\end{equation*}
	on \( \rho \) és una partició de la unitat subjecte a l'atles. 
	o localment en un obert de carta:
	\begin{align*}
		\int_{U_\alpha} \omega & =  \\
													 & \int_{\phi(U_\alpha)}
													 \omega_{\phi_\alpha^{-1}(x)}(\partial_1^{\phi_\alpha} \wedge
													 \cdots \wedge \partial_n^{\phi_\alpha}) \, dx
	\end{align*}
\end{definition}

\begin{theorem}[Fórmula d'Stokes]
	En una varietat orientada de dimensió \( n \) es té, per tot \( \omega \in
	\Omega^{n-1}_c(M) \)
	\begin{equation*}
		\int_M d\omega = \int_{\partial M} \omega
	\end{equation*}
\end{theorem}

\begin{theorem}
	Una forma \( \omega \in \Omega^{k-1}(M) \) és tancada si i només si per tota subvarietat de
	\( M \) difeomorfa a \( B(0,1) \subseteq \R^k \), \( D \),
	\begin{equation*}
		\int_{\partial D} \omega = 0. 
	\end{equation*}
\end{theorem}

\subsection{Dualitat de Poincaré}
\begin{theorem}[Dualitat de Poincaré]
	Si \( M \) és compacta i orientada, l'aplicació
	\begin{align*}
		\Phi \colon H^{n-k}(M) \times H^n(M) &  \to \R \\
		\alpha, \beta & \mapsto \int_M \alpha \wedge \beta
	\end{align*}
	és una dualitat perfecta (és una forma bilineal no degenerada).
\end{theorem}
\begin{proposition}
	Si \( M \) és compacta i orientada, \( H^k(M)^\ast \cong H^{n-k}(M)^\ast \cong
	H^{n-k}(M) \). 
\end{proposition}
\begin{observation}
	Si \( M \) és connexa, orientada i compacta \( H^0(M) \cong H^n(M) \cong \R \).
\end{observation}

\begin{observation}
	Si \( M \) és compacta i orientada i de dimensió parella \( 2n \) aleshores \( H^n(M)
	\cong H^n(M)^\ast \) per tant hi ha una forma bilineal no degenerada \( H^n(M) \times
	H^n(M) \to \R \), \( (\alpha,\beta) \mapsto \int_M \alpha \wedge \beta \). Com que \(
	\alpha \wedge \beta = (-1)^{n^2} \alpha \wedge \beta \), per \( n \) parell és simètrica
	i per \( n \) senar antisimètrica. 
	\begin{points}
	\item \emph{\( n \) parell}: pel teorema de Sylvester, hi ha \( n \) possibles formes,
		una per cada signatura \( (l, n-l) \). 
	\item \emph{\( n \) senar}: la forma és antisimètrica i és un resultat general que només
		n'hi ha una, la forma simplèctica, i que només pot existir en espais de dimensió
		parella, és a dir, \( \dim H^n(M) \) és parell. 
	\end{points}
\end{observation}

\begin{observation}[Cohomologia de les superfícies compactes orientables]
	Si la superfície és conexa aleshores \( H^0(M) \cong H^2(M) \cong \R \). A més \( H^1(M)
	\) té dimensió parella, és a dir \( \dim H^1(M) = 2g \) on \( g \) és el gènere de la
	superfície. 
\end{observation}

\begin{theorem}[Dualitat de Poincaré no compacta]
	Si \( M \) és una varietat sense frontera i orientada la forma bilineal
	\begin{align*}
		\Phi \colon H_c^{k}(M) \times H^{n-k}(M) &  \to \R \\
		\alpha, \beta & \mapsto \int_M \alpha \wedge \beta
	\end{align*}
	és no degenerada. Per tant \( H^{n-k}(M)^\ast \cong H^{n-k}(M) \cong H^k_c(M)  \).
\end{theorem}

\begin{proposition}
	Si la característica d'Euler-Poincaré d'una varietat,
	\begin{equation*}
		\chi(M) = \sum_{k = 0}^{\dim M} (-1)^k \dim H^k(M) 
	\end{equation*}
	és diferent de zero aleshores tot camp vectorial de la varietat té almenys un zero. 
\end{proposition}

\section{Cohomologies}
\subsection{\( \R^n \) (i qualsevol espai contràctil)}
\begin{equation*}
	H^p(\R^n) \cong \begin{cases}
		\R \text{ si } p = 0 \\
		0 \text{ si } p \geq 1
	\end{cases}
\end{equation*}
En el cas compacte, tots els grups són zero. Però la cohomologia amb suport compacte d'un
punt és la mateixa que en el cas general, per tant la cohomologia amb suport compacte
\emph{no} és invariant per homotopia.

\subsection{Esferes}
\begin{equation*}
	H^p(S^n) \cong \begin{cases}
		\R \text{ si } p = 0 \text{ o } p = n \\
		0 \text{ si }  1 \leq p < n \text{ o } p > n
	\end{cases}
\end{equation*}

Per \( n = 0 \), \( H^0(S^0) \cong \R \oplus \R \). 

\subsection{Espais projectius}
\( H^p(\PR^n) \cong H^p(S^n) \) per \( p < n \) i \( H^n(\PR^n) \cong \R \) quan \( n \)
és senar i \( H^n(\PR^n) \cong 0 \) quan \( n \) és parell.
\end{document}
