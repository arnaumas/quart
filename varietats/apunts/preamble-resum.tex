% PREAMBLE FOR ALGEBRAIC TOPOLOGY SUMMARY

% ----------------------------------------------------------
% Packages
\usepackage[utf8]{inputenc}
\usepackage[T1]{fontenc}
\usepackage[catalan]{babel}
\usepackage{lmodern}
\usepackage{geometry}
\usepackage{hyperref}
\usepackage[dvipsnames]{xcolor}
\usepackage[bf,sf,small,pagestyles]{titlesec}
\usepackage{titling}
\usepackage[font={footnotesize, sf}, labelfont=bf]{caption} 
\usepackage{siunitx}
\usepackage{graphicx}
\usepackage{tikz-cd}
\usetikzlibrary{babel}
\usepackage{booktabs}
\usepackage{amsmath,amssymb}
\usepackage[sort]{cleveref}
\usepackage{amsthm,thmtools}
\usepackage[shortlabels]{enumitem}

% ----------------------------------------------------------
% Geometry setup
\geometry{
	a4paper,
	right = 3cm,
	left = 3cm,
	bottom = 3cm,
	top = 3cm
}
% Wider space between lines
\renewcommand{\baselinestretch}{1.3}

% ----------------------------------------------------------
% Hyperref setup
\hypersetup{
	colorlinks,
	linkcolor = {red!50!blue},
	linktoc = page
}
\numberwithin{table}{section}
\numberwithin{equation}{section}
\numberwithin{figure}{section}

% ----------------------------------------------------------
% Definition of theorem, example, etc environments
\newcommand{\qedtriangle}{\ensuremath{\triangle}}
\newcommand{\qedtriangledown}{\ensuremath{\bigtriangledown}}
\declaretheoremstyle[spaceabove=6pt, spacebelow=6pt, headfont=\bfseries\sffamily, notefont=\normalfont, notebraces={(}{)}, qed=\qedtriangle]{definition}
\declaretheoremstyle[spaceabove=6pt, spacebelow=6pt, headfont=\bfseries\sffamily, notefont=\normalfont, notebraces={(}{)}, qed=\qedtriangledown]{example}
\declaretheoremstyle[spaceabove=6pt, spacebelow=6pt, headfont=\bfseries\sffamily,
notefont=\normalfont, notebraces={(}{)}]{theorem}


\declaretheorem[name=Teorema, style=theorem, refname={teorema,teoremes},
Refname={Teorema,Teoremes}, numberwithin=section]{theorem}
\declaretheorem[name=Proposició, style=theorem, refname={proposició,proposicions},
Refname={Proposició,Proposicions}, numberlike=theorem]{proposition}
\declaretheorem[name=Lema, style=theorem, refname={lema,lemes},
Refname={Lema,Lemas}, numberlike=theorem]{lema}
\declaretheorem[name=Definició, style=definition, refname={definició,definicions},
Refname={Definició,Definicions}, numberwithin=section]{definition}
\declaretheorem[name=Exemple, style=example, refname={exemple,exemples},
Refname={Exemple,Exemples}, numberwithin=section]{example}
\declaretheorem[name=Observació, style=example, refname={observació,observacions},
Refname={Observació,Observacions}, numberwithin=section]{observation}

% ----------------------------------------------------------
% Definition of custom list style
\newlist{points}{enumerate}{1}
\setlist[points,1]{label=\textup{(}{\itshape \roman*}\textup{)}, wide}

% ----------------------------------------------------------
% Definition of page styles
\newpagestyle{page}[\sffamily \footnotesize]{
	\headrule
	\sethead{}{\bfseries{Topologia de Varietats}}{}
	\footrule
	\setfoot{}{}{\thepage}
}
\pagestyle{page}
