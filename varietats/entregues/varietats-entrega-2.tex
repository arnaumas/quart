\documentclass[12pt]{article}
% Packages
\usepackage[utf8]{inputenc}
\usepackage[T1]{fontenc}
\usepackage[catalan]{babel}
\usepackage{lmodern}
\usepackage{geometry}
\usepackage{hyperref}
\usepackage[dvipsnames]{xcolor}
\usepackage[bf,sf,small,pagestyles]{titlesec}
\usepackage{titling}
\usepackage[font={footnotesize, sf}, labelfont=bf]{caption} 
\usepackage{siunitx}
\usepackage{graphicx}
\usepackage{booktabs}
\usepackage{amsmath,amssymb}
\usepackage[catalan,sort]{cleveref}
\usepackage{enumitem}

% Geometry setup
\geometry{
	a4paper,
	right = 2.5cm,
	left = 2.5cm,
	bottom = 3cm,
	top = 3cm
}
% Wider space between lines
\renewcommand{\baselinestretch}{1.3}

\hypersetup{
	colorlinks,
	linkcolor = {red!50!blue},
	linktoc = page
}

% Cref names in catalan
\crefname{figure}{figura}{figures}
\crefname{table}{taula}{taules}
\numberwithin{table}{section}
\numberwithin{figure}{section}

\graphicspath{{./figs/}}


\input{commands-varietats.tex}

\newpagestyle{pagina}{
	\headrule
	\sethead*{\sffamily {\bfseries Topología de variedades:} Entrega 2}{}{\theauthor}
	\footrule
	\setfoot*{}{}{\sffamily \thepage}
}
\renewpagestyle{plain}{
	\footrule
	\setfoot*{}{}{\sffamily \thepage}
}
\pagestyle{pagina}

\title{\sffamily {\bfseries Topología de variedades:} Entrega 2}
\author{\sffamily Arnau Mas}
\date{\sffamily 22 de noviembre de 2019}


\begin{document}
\maketitle

Sean \( N \) y \( M \) variedades y \( f \colon N \to M \) una aplicación lisa. A partir
de \( f \) definimos la aplicación
\begin{align*}
	F \colon N & \longrightarrow N \times M \\
	x & \longmapsto (x, f(x)).
\end{align*}

Comprovemos que \( F \) es lisa. En general, una aplicación \( g \colon Z \to N \times M
\) es lisa si y solo si lo son sus componentes, \( \pi_N \circ g \) y \( \pi_M \circ g \),
donde \( \pi_N \colon N \times M \to N \) y \( \pi_M \colon N \times M \to M \) son las
proyecciones canónicas. Es un resultado de topología general que \( g \) es contínua si y
solo si lo son \( \pi_N \circ g \) y \( \pi_M \circ g \). Por lo tanto sólo tenemos que
verificar que la lectura de \( g \) en cartas es lisa. 

Sea \( (U, \phi) \) una carta de \( Z \) alrededor de un punto \( x \in Z \). Puesto que
\( g(x) \in N \times M \) podemos tomar \( (V_1 \times V_2, \psi_1 \times \psi_2) \) como
carta alrededor de \( g(x) \), donde \( (V_1, \psi_1) \) es una carta de \( N \) alrededor
de \( \pi_N(g(x)) \) y \( (V_2, \psi_2) \) es una carta de \( M \) alrededor de \(
\pi_M(g(x)) \). Todas estas aplicaciones hacen que el diagrama a continuación commute,
\begin{equation*}
	\begin{tikzcd}[row sep = small]
			& & & & V_2 \arrow[ddd, "\psi_2"] \\
		U \arrow[rr, "g"] \arrow[ddd, "\phi"] & & V_1 \times V_2 \arrow[ddd, "\psi_1 \times
		\psi_2"'] \arrow[rru, "\pi_M"] \arrow[rd, "\pi_N"] & & \\
			& & & V_1  & \\
			& & & & \psi_2(V_2) \\
		\phi(U) \arrow[rr, "(\psi_1 \times \psi_2) \circ g \circ \phi^{-1}"] & & \psi_1(V_1)
		\times \psi_2(V_2) \arrow[dr, "\pi_1"] \arrow[rru, near start, "\pi_2"] & & \\
			& & & \psi_1(V_1) \arrow[from=uuu, crossing over, near start,"\psi_1"] &
	\end{tikzcd}
\end{equation*}
Si \( Z \) tiene dimensión \( r \) y \( N \) y \( M \) dimensión \( n \) y \( m \)
respectivamente, entonces \( \psi_1(V_1) \) es un abierto de \( \R^n \) y \( \psi_2(V_2)
\) es un abierto de \( \R^m \). Por lo tanto \( \pi_1 \) y \( \pi_2 \) son las
proyecciones de \( \R^n \times \R^m \) sobre \( \R^n \) y \( \R^m \) respectivamente.

Del diagrama leemos que
\begin{equation*}
	\psi_2 \circ \pi_N \circ g \circ \phi^{-1} = \pi_1 \circ (\psi_1 \times \psi_2) \circ g
	\circ \phi^{-1},
\end{equation*}
es decir que la lectura en cartas de la primera componente de \( g \), \( \pi_N \circ g \)
es precisamente la primera componente de la lectura en cartas de \( g \), \( \tilde{g}
\defeq (\psi_1 \times \psi_2) \circ g \circ \phi^{-1} \). De la misma forma tenemos
\begin{equation*}
	\psi_2 \circ \pi_M \circ g \circ \phi^{-1} = \pi_2 \circ (\psi_1 \times \psi_2) \circ g
	\circ \phi^{-1} = \pi_2 \circ \tilde{g}.
\end{equation*}

La aplicación \( \tilde{g} \) es una función real de un abierto de \( \R^r \) a un abierto
de \( \R^n \times \R^m \) por lo que es de clase \( C^\infty \) si y solo si lo son sus
dos componentes. Por definición, \( g \) es lisa si y solo si su lectura en cualquier carta es de
clase \( C^\infty \). Por lo tanto, \( g \) será lisa si y solo si las lecturas en
cartas de sus dos componentes son de clase \( C^{\infty} \), y por lo que hemos visto, si
y solo si sus dos componentes \( \pi_M \circ g \) y \( \pi_N \circ g \) son lisas. 

Con este resultado más general es immediato ver que \( F \) es lisa, puesto que su
primera componente es la identidad y su segundo componente es \( f \), que es lisa por
hipótesis.

\parbreak
 
Podemos entonces hablar de la aplicación tangente a \( F \), \( T_x F \colon T_x N \to
T_{F(x)}(N \times M) \). El espacio tangente de un producto se identifica canónicamente
con el producto de espacios tangentes
\begin{equation}\label{eq:iso entre espacios tangentes}
	\begin{aligned}
		T_{(x,y)}(N \times M) & \longrightarrow T_x N \times T_y M \\
		v & \longmapsto (T_{(x,y)} \pi_N(v), T_{(x,y)}\pi_M (v)).
	\end{aligned}
\end{equation}
Para \( v \in T_x N \) calculamos, usando la regla de la cadena
\begin{equation*}
	(T_{F(x)} \pi_N)(T_xF(v)) = (T_{F(x)}\pi_N \circ T_xF)(v) = T_x(\pi_N \circ F)(v) =
	T_x\id_N(v) = v
\end{equation*}
y
\begin{equation*}
	(T_{F(x)} \pi_M)(T_xF(v)) = (T_{F(x)}\pi_M \circ T_xF)(v) = T_x(\pi_M \circ F)(v) =
	T_xf(v).
\end{equation*}
Por lo que, mediante el isomorfismo en (\ref{eq:iso entre espacios tangentes}), tenemos
\begin{equation*}
	T_xF(v) = (v, T_xf(v)).
\end{equation*}

\parbreak

Definimos el gráfico de la función \( f \) como
\begin{equation*}
	\Gamma_f \defeq \{(x,y) \in N \times M \mid y = f(x)\}
\end{equation*}
y de hecho se tiene \( \Gamma_f = F(N) \).

A continuación comprovamos que \( \Gamma_f \) es una subvariedad de \( N \times M \)
difeomorfa a \( N \). Sea \( (V, \psi) \) una carta de \(
M\) alrededor de un punto de \( f(N) \), \( f(x) \). Entonces \( f^{-1}(V) \) es un
entorno abierto de \( x \) en \( N \). Por lo tanto existe una carta de \( N \) alrededor
de \( x \), \( (U, \phi) \) tal que \( f(U) \subseteq V \) ---si \( U \) es demasiado
grande simplemente lo intersecamos con \( f^{-1}(V) \) y obtenemos una carta que satisface
la condición que queremos---. Entonces \( U \times V \) es un abierto de \( N \times M \)
que contiene \( (x, f(x)) \). Definimos, si \( \dim N = n \) y \( \dim M = m \),
\begin{align*}
	\chi \colon U \times V & \longrightarrow \xi(U \times V) \subseteq \R^n \times \R^m \\
	(x, y) & \longmapsto (\phi(x), \psi(y) - \psi(f(x)))
\end{align*}
Tenemos que \( \xi \) es una carta de \( N \times M \). En efecto, está claro que es contínua, y
además tiene una inversa,
\begin{align*}
	\chi^{-1} \colon U \times V & \longrightarrow \xi(U \times V) \subseteq \R^n \times \R^m \\
	(a, b) & \longmapsto (\phi^{-1}(a), \psi^{-1}(b + \psi(f(\phi^{-1}(a)))))
\end{align*}
que también es contínua, por lo que \( \xi \) es un homeomorfismo sobre su imagen, \( W
\defeq \xi(U \times V) \) que es un abierto de \( \R^{n+m} \) y por lo tanto una
carta.

En particular, \( \xi \) es una carta linealizante para \( \Gamma_f \) alrededor de \( (x,
f(x))\). Por un lado, si \( (x, f(x)) \in (U \times V) \cap
\Gamma_f \) entonces
\begin{equation*}
	\chi(x, f(x)) = (\phi(x), \psi(f(x)) - \psi(f(x))) = (\phi(x), 0) \in \phi(U) \times
	0 = W \cap \R^n \times 0.
\end{equation*}
De manera inversa, si \( (a,0) \in W \cap \R^n \times 0 \) entonces existe \( (x,y) \in U
\times V \) tal que 
\begin{equation*}
	(a,0) = \chi(x,y) = (\phi(x), \psi(y) - \psi(f(x)))
\end{equation*}
por lo que, como	\( \psi \) es un homeomorfismo y en particular biyectiva, \( y = f(x) \)
y deducimos \( (a,0) = (\phi(x), 0) \in \chi(\Gamma_f \cap (U \times V)) \). Es decir,
\( \chi \) nos da una linearización local de \( \Gamma_f \) alrededor de \( (x, f(x))
\), por lo que \( \Gamma_f \) es una subvariedad de \( N \times M \).

La aplicación \( F \), con su codominio restringido a \( \Gamma_f \), realiza un
difeomorfismo entre \( N \) y \( \Gamma_f \). Puesto que \( F(N) = \Gamma_f \) tenemos que es exhaustiva. También es inyectiva porque si \( F(x) =
F(y) \) entonces \( (x, f(x)) = (y, f(y)) \) luego \( x = y \). Hemos comprobado
inicialmente que es lisa. Por último, la inversa de \( F \) es la restricción de \( \pi_N
\) sobre \( \Gamma_f \), que es también lisa. Por lo tanto \( F \) es un difeomorfismo.

En particular, la aplicación tangente \( T_xF \) nos da un isomorfismo entre \( T_xN \) y
\( T_{F(x)}\Gamma_f \):
\begin{align*}
	T_x F \colon T_xN & \longrightarrow T_{F(x)} \Gamma_f \\
	v & \longmapsto (v, T_xf(v))
\end{align*}
donde estamos identificando \( T_{(x,f(x))}(N \times M) \) con \( T_x N \times T_{f(x)}M
\). Por lo tanto todo elemento de \( T_{F(x)} \Gamma_f \) es de la forma \( (v, T_xf(v))
\), es decir, \( T_{F(x)}\Gamma_f \) es el gráfico de \( T_xf \).



\end{document}
