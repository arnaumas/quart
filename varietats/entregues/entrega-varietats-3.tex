\documentclass[12pt]{article}
\usepackage[utf8]{inputenc}
\usepackage[T1]{fontenc}
\usepackage[spanish]{babel}
\usepackage{lmodern}
\usepackage{geometry}
\usepackage{hyperref}
\usepackage[dvipsnames]{xcolor}
\usepackage[bf,sf,small,pagestyles]{titlesec}
\usepackage{titling}
\usepackage[font={footnotesize, sf}, labelfont=bf]{caption} 
\usepackage{siunitx}
\usepackage{graphicx}
\usepackage{tikz-cd}
\usetikzlibrary{babel}
\usepackage{booktabs}
\usepackage{amsmath,amssymb}
\usepackage[spanish,sort]{cleveref}
\usepackage{enumitem}

\geometry{
	a4paper,
	right = 2.5cm,
	left = 2.5cm,
	bottom = 3cm,
	top = 3cm
}
\renewcommand{\baselinestretch}{1.3}

\hypersetup{
	colorlinks,
	linkcolor = {red!50!blue},
	linktoc = page
}

\crefname{figure}{figura}{figures}
\crefname{table}{taula}{taules}
\numberwithin{table}{section}
\numberwithin{figure}{section}

\graphicspath{{./figs/}}


\renewcommand{\arraystretch}{1.4}


% CUSTOM COMMANDS FOR ALGEBRAIC TOPOLOGY
% ----------------------------------------------------------

% Restriction of a function
\newcommand{\rest}[1]{\raisebox{-.5ex}{$|$}_{#1}}

% Real numbers
\newcommand{\R}{\mathbb{R}}
\newcommand{\PR}{\mathbb{PR}}

% Rational numbers
\newcommand{\Q}{\mathbb{Q}}

% Complex numbers
\newcommand{\C}{\mathbb{C}}

% Natural numbers
\newcommand{\N}{\mathbb{N}}

% Integers
\newcommand{\Z}{\mathbb{Z}}

% Vector bold
\renewcommand{\vec}[1]{\mathbf{#1}}

% Span
\newcommand{\gen}[1]{\langle #1 \rangle}

% Set
\newcommand{\set}[1]{\{ #1 \}}

% Script A, B, M, P
\newcommand{\A}{\mathcal{A}}
\newcommand{\B}{\mathcal{B}}
\newcommand{\M}{\mathcal{M}}
\renewcommand{\P}{\mathcal{P}}
\renewcommand{\S}{\mathfrak{S}}

% Identity
\newcommand{\id}{\mathrm{id}}

% Kernel and image
\DeclareMathOperator{\im}{im}
\DeclareMathOperator{\coker}{coker}

% Absolute value
\newcommand{\abs}[1]{\lvert #1 \rvert}

% Norm
\newcommand{\norm}[1]{\lVert #1 \rVert}

% Category of Vector Spaces
\newcommand{\Vect}{\mathsf{Vect}}
\newcommand{\VectK}{\Vect_{K}}
\newcommand{\VectR}{\Vect_{\R}}
\DeclareMathOperator{\Hom}{Hom}
\DeclareMathOperator{\Bil}{Bil}
\newcommand{\dual}{^{\vee}}
\DeclareMathOperator{\tr}{tr}

% Category of Manifolds
\DeclareMathOperator{\Diff}{Diff}

% Epi and monomorphisms
\newcommand{\onto}{\twoheadrightarrow}
\newcommand{\into}{\tailrightarrow}

\newcommand{\parbreak}{
	\begin{center}
		--- $\ast$ ---
	\end{center} 
}

% Defined as
\makeatletter
\newcommand*{\defeq}{\mathrel{\rlap{%
    \raisebox{0.3ex}{$\m@th\cdot$}}%
  \raisebox{-0.3ex}{$\m@th\cdot$}}%
	=
}
\makeatother

% Support
\DeclareMathOperator{\supp}{supp}

% Categories
\newcommand{\Top}{\mathsf{Top}}


\declaretheorem[name=Lema de la Serpiente, refname={lema de la serpiente},
Refname={Lema de la serpiente}, numbered=no]{serpiente}

\newpagestyle{pagina}{
	\headrule
	\sethead*{\sffamily {\bfseries Topología de variedades:} Entrega 3}{}{\theauthor}
	\footrule
	\setfoot*{}{}{\sffamily \thepage}
}
\renewpagestyle{plain}{
	\footrule
	\setfoot*{}{}{\sffamily \thepage}
}
\pagestyle{pagina}

\title{\sffamily {\bfseries Entrega 3:} Lema de la serpiente}
\author{\sffamily Arnau Mas}
\date{\sffamily 19 de enero de 2019}


\begin{document}
\maketitle

A continuación demostraremos el siguiente resultado, útil para cçálculos de álgebra
hommológica.

\begin{serpiente}
Supongamos que se tiene, en la categoría \( \mathsf{Vect}_K \) de \( K \)-espacios
vectoriales ---aunque el resultado es válido en cualquier categoría abeliana---, el
diagrama commutativo
\begin{equation*}
	\begin{tikzcd}
		0 \arrow[r] & A_1 \arrow[r, "f_1"] \arrow[d, "a"] & B_1 \arrow[r, "g_1"] \arrow[d, "b"] & C_1
		\arrow[r] \arrow[d, "c"] & 0 \\
		0 \arrow[r] & A_2 \arrow[r, "f_2"] & B_2 \arrow[r, "g_2"] & C_2 \arrow[r] & 0 
	\end{tikzcd}
\end{equation*}
donde ambas filas son sucesiones exactas. Entonces existe una sucesión exacta
\begin{equation*}
	\begin{tikzcd}
		0 \arrow[r] & \ker{a} \arrow[r] & \ker{b} \arrow[r] & \ker{c} \arrow[dll, "\delta"
		description, out=0, in=180, looseness=2, overlay] \\
			& \coker{a} \arrow[r] & \coker{b} \arrow[r] & \coker{c} \arrow[r] & 0
	\end{tikzcd}
\end{equation*}
\end{serpiente}

En primer lugar, recordemos que el conúcleo de un morfismo \( f \colon A \to B \) es el
cociente \( B/\im{f} \). Consideremos la siguiente sucesión para el morfismo \( a \),
\begin{equation*}
	\begin{tikzcd}
		\ker{a} \arrow[r, "\iota_a", tail] & A_1 \arrow[r, "a"] & A_2 \arrow[r,
		"\pi_a", two heads] & \coker{a}
	\end{tikzcd}
\end{equation*}
donde \( \iota_a \) es la inclusión de \( \ker{a} \) dentro de \( A_1 \) y \( \pi_a \) es
la proyección de \( A_2 \) sobre \( \coker{a} \). Esta sucesión es de hecho exacta.
Efectivamente, \( \im{\iota_a} = \ker{a} \) esencialmente por definición, que es una de
las condiciones de exactitud. Similarmente, \( \pi_a \) manda \( \im{a} \) a 0, por lo que
\( \ker{\pi_a} = \im{a} \), que es el último requerimiento para la exactitud. Este mismo
argumento funciona con \( b \) y \( c \), por lo que podemos expandir el diagrama original
verticalmente resultando en el diagrama commutativo
\begin{equation*}
	\begin{tikzcd}
		& \ker{a} \arrow[d, "\iota_a", tail] & \ker{b} \arrow[d, "\iota_b", tail] & \ker{c}
		\arrow[d, "\iota_c", tail] \\
		0 \arrow[r] & A_1 \arrow[r, "f_1"] \arrow[d, "a"] & B_1 \arrow[r, "g_1"] \arrow[d, "b"] & C_1
		\arrow[r] \arrow[d, "c"] & 0 \\
		0 \arrow[r] & A_2 \arrow[r, "f_2"] \arrow[d, "\pi_a", two heads] & B_2 \arrow[r,
		"g_2"] \arrow[d, "\pi_b", two heads] & C_2 \arrow[r] \arrow[d, "\pi_c", two heads] & 0
		\\
								& \coker{a} & \coker{b} & \coker{c}.
	\end{tikzcd}
\end{equation*}
Tenemos que conectar los núcleos y los conúcleos de manera exacta, y después definir el
morfismo \( \delta \) que conecta \( \ker{c} \) con \( \coker{a} \). Empezamos con los
núcleos.

Sea \( x \in \ker{a} \). Entonces, usando que \( b \circ f_1 = f_2 \circ a \), se tiene
\begin{equation*}
	(b \circ f_1)(x) = (f_2 \circ a)(x) = f_2(0) = 0
\end{equation*}
por lo que \( f_1(x) \in \ker{b} \). Dicho de otra forma, la imagen de la restricción de
\( f_1 \) a \( \ker{a} \), \( f_1 \circ \iota_a \), está dentro de \( \ker{b} \), por lo
que obtenemos una factorización
\begin{equation*}
	f_1 \circ \iota_a = \iota_b \circ \hat{f}_1
\end{equation*}
donde \( \hat{f}_1 \) es la restricción de \( f_1 \) a \( \ker{a} \) y \( \ker{b} \). Por
el mismo argumento, usando que \( c \circ g_1 = g_2 \circ b \) obtenemos la factorización
\begin{equation*}
	g_1 \circ \iota_b = \iota_c \circ \hat{g}_2.
\end{equation*}
Todo esto nos da el diagrama commutativo
\begin{equation}\label{eq:kernels}
	\begin{tikzcd}
		\ker{a} \arrow[d, "\iota_a", tail] \arrow[r, "\hat{f}_1"] & \ker{b} \arrow[d,
		"\iota_b", tail] \arrow[r, "\hat{g}_1"] & \ker{c}
		\arrow[d, "\iota_c", tail] \\
		A_1 \arrow[r, "f_1"] & B_1 \arrow[r, "g_1"]  & C_1 \\
	\end{tikzcd}
\end{equation}
Tenemos que comprovar que la sucesión \( 0 \to \ker{a} \xrightarrow{\hat{f}_1} \ker{b}
\xrightarrow{\hat{g}_1} \ker{c} \) es exacta. 

Si \( \hat{f}_1(x) = 0 \) entonces \( (\iota_b \circ \hat{f}_1)(x) = 0 \). Pero por
\eqref{eq:kernels}, \( \iota_b \circ \hat{f}_1 = f_1 \circ \iota_a \), y este último
morfismo es composición de morfismos inyectivos, por lo tanto inyectivo. Entonces
deducimos que \( x = 0 \) y que \( \hat{f}_1 \) es inyectivo. Tenemos que comprovar
también que \( \im{\hat{f}_1} = \ker{\hat{g}_1} \).  Si \( x \in \im{\hat{f}_1} \)
entonces podemos escribir \( x = \hat{f}_1(y) \) para \( y \in \ker{a} \) y entonces,
usando la commutatividad de \eqref{eq:kernels}
\begin{equation*}
	(\iota_c \circ \hat{g}_1)(x) = (\iota_c \circ \hat{g}_1 \circ \hat{f}_1)(y) = (g_1 \circ
	f_1 \circ \iota_a)(y) = 0
\end{equation*}
donde usamos la exactitud de la sucesión inicial, que implica \( g_1 \circ f_1 = 0 \).
Entonces, como \( \iota_c \) es inyectiva, se concluye \( \hat{g}_1(x) = 0 \). Tenemos,
pues, \( \im{\hat{f}_1} \subseteq \ker{\hat{g}_1} \). En el otro sentido, sea \( x \in
\ker{\hat{g}_1} \). Entonces \( 0 = (\iota_c \circ \hat{g}_1)(x) = (g_1 \circ \iota_b)(x)
\). Por exactitud, se tiene que \( \iota_b(x) \in \ker{g_1} = \im{f_1} \), por lo que
podemos escribir \( \iota_b(x) = f_1(y) \) para \( y \in A_1 \). Usamos que \( b \circ f_1
= f_2 \circ a \) y encontramos
\begin{align*}
	(f_2 \circ a)(y) & = (b \circ f_1)(y) \\
									 & = (b \circ \iota_b)(x) \\
									 & = 0.
\end{align*}
Como \( f_2 \) es inyectivo, \( a(y) = 0 \), por lo que \( y \in \ker{a} \). Por lo tanto
\begin{equation*}
	\iota_b(x) = f_1(y) = f_1(\iota_a(y)) = \iota_b(\hat{f}_1(y))
\end{equation*}
y como \( \iota_b \) es inyectiva, \( x = \hat{f}_1(y) \). Es decir, \( x \) tiene una
preimagen por \( \hat{f}_1 \) en \( \ker{a} \), por lo tanto \( x \in \im{\hat{f}_1} \) y
tenemos la igualdad que buscábamos, \( \im{\hat{f}_1} = \ker{\hat{g}_1} \).

De manera análoga se obtiene la sucesión exacta \( \coker{a} \to \coker{b}
\to \coker{c} \to 0 \). Construimos primero los morfismos entre los conúcleos.  

Sea \( x \in \im{a} \), por lo que se tiene \( x = a(y) \) para \( y \in A_2 \). Entonces 
\begin{align*}
	(\pi_b \circ f_2)(x) & = (\pi_b \circ f_2)(a(y)) \\
											 & = \pi_b\big((f_2 \circ a)(y)\big) \\
											 & = \pi_b\big((b \circ f_2)(y)\big) \\
											 & = 0.
\end{align*}
La última igualdad es porque \( (b \circ f_1)(y) = b(f_1(y)) \) está en la imagen de \( b
\) y la proyección \( \pi_b \) manda \( \im{b} \) a 0. Como \( \pi_b \circ f_2 \) es nula
en \( \im{a} \), debe factorizar a través de \( A_2/\im a = \coker{a} \). Esto significa
que existe \( \bar{f}_2 \colon \coker{a} \to \coker{b} \) tal que \( \bar{f}_2 \circ \pi_a
= \pi_b \circ f_2 \). Por el mismo argumento obtenemos otro morfismo \( \bar{g}_2 \colon
\coker{b} \to \coker{c} \), y por lo tanto el diagrama commutativo
\begin{equation}\label{eq:cokernels}
	\begin{tikzcd}
		A_2 \arrow[d, "\pi_a", two heads] \arrow[r, "f_2"] & B_2 \arrow[d, "\pi_b", two heads]
		\arrow[r, "g_2"] & C_2 \arrow[d, "\pi_c", two heads] \\
		\coker{a} \arrow[r, "\bar{f}_2"] & \coker{b} \arrow[r, "\bar{g}_2"]  & \coker{c} \\
	\end{tikzcd}
\end{equation}

Queda por verificar que la sucesión \( \coker{a} \xrightarrow{\bar{f}_2} \coker{b}
\xrightarrow{\bar{g}_2} \coker{c} \to 0 \) es exacta. Veamos primero que \( \bar{g}_2 \)
es exhaustiva. Sea \( x \in \coker{c} \). Puesto que tanto \( \pi_c \) como \( g_2 \) son
exhaustivas, \( x \) tiene en \( B_2 \) una preimagen por \( \pi_c \circ g_1 \), \( y \).
Usando la commutatividad de \eqref{eq:cokernels} obtenemos
\begin{equation*}
	x = (\pi_c \circ g_2)(y) = (\bar{g}_2 \circ \pi_b)(y)
\end{equation*}
luego \( \pi_b(y) \) es una preimagen de \( x \) por \( \bar{g}_2 \), que demuestra que \(
\bar{g}_2 \) es exhaustiva.

Queda por ver que \( \ker{\bar{g}_2} = \im{\bar{f}_2} \). Sea \( x \in \im{\bar{f}_1} \),
es decir, que \( x = \bar{f}_1(y) \) para \( y \in \coker{b} \). Por la exhaustividad de
\( \pi_a \), \( y = \pi_a(z) \) para algun \( z \in A_2 \). Entonces, usando la
commutatividad de \eqref{eq:cokernels},
\begin{align*}
	\bar{g}_2 (x) = (\bar{g}_2 \circ \bar{f}_2 \circ \pi_a)(z) = (\pi_c \circ g_2 \circ
	f_2)(z) = 0
\end{align*}
puesto que \( g_2 \circ f_2 = 0 \) por exactitud. En el otro sentido, sea \( x \in
\ker{\bar{g}_2} \). Tenemos que demostrar que \( x \) tiene una preimagen por \(
\bar{f}_2 \). Podemos poner \( x = \pi_b(y) \) para algún \( y \in B_2 \). Entonces \(
0 = \bar{g}_2(x) = (\bar{g}_2 \circ \pi_b)(y) = (\pi_c \circ g_2)(y) \). Es decir, \( g_2(y) \in
\ker{\pi_c} \) y \( \ker{\pi_c} = \im{c} \) por exactitud. Entonces existe \( z \in C_1 \)
tal que \( c(z) = g_2(y) \). Y como \( g_1 \) es exhaustivo, \( z = g_1(w) \) para algún
\( w \in B_1 \). Entonces \( g_2(y) = (c \circ g_1)(w) = (g_2 \circ b)(w) \). Por lo tanto
\( g_2(y - b(w)) = 0 \), por lo que, por exactitud, existe \( y' \in A_2 \) tal que \(
f_2(y') = y - b(w) \). Y entonces
\begin{equation*}
	(\pi_b \circ f_2)(y') = \pi_b(y - b(w)) = \pi_b(y) = x.
\end{equation*}
Pero \( \pi_b \circ f_2 = \bar{f}_2 \circ \pi_a \), luego \( \pi_a(y') \in \coker{a} \) es
una preimagen de \( x \) por \( \bar{f}_2 \), lo que completa la prueba de que \(
\ker{\bar{g}_2} = \im{\bar{f}_2} \).

Podemos resumir todo esto en el diagrama commutativo
\begin{equation*}
	\begin{tikzcd}
		0 \arrow[r] & \ker{a} \arrow[d, "\iota_a", tail] \arrow[r, "\hat{f}_1"] & \ker{b} \arrow[d,
		"\iota_b", tail] \arrow[r, "\hat{g}_1"] & \ker{c}
		\arrow[d, "\iota_c", tail] \\
		0 \arrow[r] & A_1 \arrow[r, "f_1"] \arrow[d, "a"] & B_1 \arrow[r, "g_1"] \arrow[d, "b"] & C_1
		\arrow[r] \arrow[d, "c"] & 0 \\
		0 \arrow[r] & A_2 \arrow[r, "f_2"] \arrow[d, "\pi_a", two heads] & B_2 \arrow[r,
		"g_2"] \arrow[d, "\pi_b", two heads] & C_2 \arrow[r] \arrow[d, "\pi_c", two heads] & 0
		\\
																				 & \coker{a} \arrow[r, "\bar{f}_2"] &
		\coker{b} \arrow[r, "\bar{g}_2"]  & \coker{c} \arrow[r] & 0 \\
	\end{tikzcd}
\end{equation*}
donde todas las filas y columnas son exactas.

Por último tenemos que construir el morfismo de conexión \( \delta \colon \ker{c} \to
\coker{a} \) que haga que la secuencia cojunta sea exacta. Sea \( x \in \ker{c} \). Por
ser \( g_1 \) exhaustiva, existe \( y \in B_1 \) tal que \( g_1(y) = \iota_c(x) \), luego
\begin{equation*}
	0 = (c \circ \iota_c)(x) = (c \circ g_1)(y) = (g_2 \circ b)(y).
\end{equation*}
Entonces \( b(y) \) está en el núcleo de \( g_2 \), que por exactitud es lo mismo que la
imagen de \( f_2 \). Entonces existe \( z \in A_2 \) tal que \( f_2(z) = b(y) \). Pero por
la inyectividad de \( f_2 \), cada elemento de \( \im{f_2} \) tiene una única preimagen,
por lo que \( z \) es único. Entonces \( \pi_a(z) \in \coker{a} \), por lo que podemos
definir \( \delta(x) = \pi_a(z) \). Tenemos que ver que así definido \( \delta \) es un
morfismo. La elección de \( z \) no es ambigua, tal y como hemos argumentado. Sí lo es, en
cambio, al elección de \( y \). Si \( y' \) es otra preimagen de \( \iota_c(x) \) por \(
g_1 \) se tiene que \( y' - y \) está en el núcleo de \( g_1 \). Y por exactitud podemos
escribir
\begin{equation*}
	y' = y + f_1(w)
\end{equation*}
para algún \( w \in A_1 \). Sea \( z' \) la preimagen de \( y' \) por \( f_2 \). Tenemos
\begin{equation*}
	f_2(z - z') = b(y - y') = (b \circ f_1)(w) = (f_2 \circ a)(w). 
\end{equation*}
Como \( f_2 \) es inyectiva se concluye \( z - z' = a(w) \), por lo que \( \pi_a(z - z') =
0 \) y \( \pi_a(z) = \pi_a(z') \). Por lo tanto \( \delta \) está bien definido. También
es lineal, pues si \( \delta(x_1) = \pi_a(z_1) \) y \( \delta(x_2) = \pi_a(z_2) \)
entonces para todo \( \lambda \in K \) 
\begin{equation*}
	f_2(z_1 + \lambda z_2) = f_2(z_1) + \lambda f_2(z_2) = b(y_1) + \lambda b(y_2)
\end{equation*}
donde \( y_1 \) y \( y_2 \) son una preimagen de \( \iota_c(x_1) \) y \( \iota_c(x_2) \)
por \( b \), respectivamente. Por lo que \( f_2(z_1 + \lambda z_2) = \iota_c(x_1) +
\lambda \iota_c(x_2) = \iota_c(x_1 + \lambda x_2) \) y entonces \( \delta(x_1 + \lambda
x_2) = \pi_a(z_1 + \lambda z_2) = \pi_a(z_1) + \lambda \pi_a(z_2) = \delta(x_1) + \lambda
\delta(x_1) + \lambda \delta(x_2). \)

Tenemos la sucesión
\begin{equation*}
	\begin{tikzcd}
		0 \arrow[r] & \ker{a} \arrow[r, "\hat{f}_1"] & \ker{b} \arrow[r, "\hat{g}_1"] &
		\ker{c} \arrow[dll, "\delta" description, out=0, in=180, looseness=2, overlay] \\ &
		\coker{a} \arrow[r, "\bar{f}_2"] & \coker{b} \arrow[r, "\bar{g}_2"] & \coker{c}
		\arrow[r] & 0
	\end{tikzcd}
\end{equation*}
Solo queda demostrar que es exacta, y para ello es suficiente comprovar que es exacta en
\( \ker{c} \) y \( \coker{a} \), puesto que ya hemos visto que es exacta en el resto de
puntos. 

Demostremos que \( \ker{\delta} = \im{\hat{g}_1} \). Si \( x \in \im{\hat{g}_1} \)
entonces \( x = \hat{g}_1(y) \) para \( y \in \ker{b} \). Entonces
\begin{equation*}
	(g_1 \circ \iota_b)(y) = (\iota_c \circ \hat{g}_1)(y) = \iota_c(x)
\end{equation*}
por lo que \( \iota_b(y) \) es una preimagen de \( \iota_c(x) \) por \( g_1 \). Entonces
\( (b \circ \iota_b)(y) = 0 \), luego su preimagen por \( f_2 \) es 0 y \( \delta(x) =
\pi_a(0) = 0 \). En el otro sentido, supongamos \( \delta(x) = 0 \). Tenemos que ver que
\( x \) es la imagen de algún elemento de \( \ker{b} \). Sea \( y \) una preimagen de \(
\iota_c(x) \) por \( g_1 \) y \( z \) la preimagen de \( b(y) \) por \( f_2 \). Por
la construcción de \( \delta \), \( \pi_a(z) = 0 \), por lo que \( z = a(w) \) para \( w
\in A_1 \). Entonces
\begin{equation*}
	b(y) = f_2(z) = (f_2 \circ a)(w) = (b \circ f_1)(w)
\end{equation*}
y \( b(y - f_1(w)) = 0 \). Escribamos \( y - f_1(w) = \iota_b(u) \) para \( u \in \ker{b}
\), con lo que resulta
\begin{equation*}
	(\iota_c \circ \hat{g}_1)(u) = (g_1 \circ \iota_b)(u) = g_1(y) - (g_1 \circ f_1)(w) =
	g_1(y) = \iota_c(x)
\end{equation*}
con lo que, por la inyectividad de \( \iota_c \), \( \hat{g_1}(u) = x \).

La demostración de exactitud en \( \coker{a} \) es muy parecida. Sea \( x \in \im{\delta}
\), por lo que \( x = \delta(y) \) para algún \( y \in \ker{c} \). Por la construcción de
\( \delta \), existe \( z \in A_2 \) tal que \( \delta(y) = \pi_a(z) \). Aplicando \(
\bar{f}_2 \) se obtiene
\begin{equation*}
	\bar{f}_2(x) = (\bar{f}_2 \circ \pi_a)(z) = (\pi_b \circ f_2)(z).
\end{equation*}
Recordemos que, en la construcción de \( \delta \), \( f_2(z) \) estaba en la imagen de	\(
b \), por lo que \( \bar{f}_2(x) = \pi_b(f_2(z)) = 0 \). Por lo tanto \( \im{\delta} \subseteq
\ker{\bar{f}_2} \).

En el otro sentido, si \( x \in \coker{a} \) es tal que \( \bar{f}_2(x) = 0 \) tenemos que
ver que existe algún elemento en \( \ker{c} \) cuya imagen por \( \delta \) es \( x \).
Existe \( z \in A_2 \) tal que \( x = \pi_a(z) \), luego
\begin{equation*}
	0 = \bar{f}_2(x) = (\bar{f}_2 \circ \pi_a)(z) = (\pi_b \circ f_2)(z)
\end{equation*}
lo que nos dice que \( f_2(z) \) está en la imagen de \( b \). Es decir, existe \( y \in
B_1 \) tal que \( b(y) = f_2(z) \). Queremos saber si \( c(g_1(y)) = 0 \), puesto que si
lo es tendremos que \( g_1(y) \in \ker{c} \) y entonces \( \delta(g_1(y)) = \pi_a(z) = x
\). Y efectivamente
\begin{equation*}
	(c \circ g_1)(y) = (g_2 \circ b)(y) = (g_2 \circ f_2)(z) = 0
\end{equation*}
por exactitud. Entonces \( x \in \im{\delta} \), lo que termina la prueba de la exactitud
de la sucesión en \( \coker{a} \), y por lo tanto de la exactitud de la sucesión entera.
\end{document}
