\documentclass[12pt]{article}

\usepackage[utf8]{inputenc}
\usepackage[T1]{fontenc}
\usepackage[spanish]{babel}
\usepackage{lmodern}
\usepackage{geometry}
\usepackage{hyperref}
\usepackage[dvipsnames]{xcolor}
\usepackage[bf,sf,small,pagestyles]{titlesec}
\usepackage{titling}
\usepackage[font={footnotesize, sf}, labelfont=bf]{caption} 
\usepackage{siunitx}
\usepackage{graphicx}
\usepackage{booktabs}
\usepackage{amsmath,amssymb}
\usepackage[spanish,sort]{cleveref}
\usepackage{enumitem}

\geometry{
	a4paper,
	right = 2.5cm,
	left = 2.5cm,
	bottom = 3cm,
	top = 3cm
}

\hypersetup{
	colorlinks,
	linkcolor = {red!50!blue},
	linktoc = page
}

\crefname{figure}{figura}{figures}
\crefname{table}{taula}{taules}
\numberwithin{table}{section}
\numberwithin{figure}{section}

\graphicspath{{./figs/}}

% Unitats
\sisetup{
	inter-unit-product = \ensuremath{ \cdot },
	allow-number-unit-breaks = true,
	detect-family = true,
	list-final-separator = { i },
	list-units = single
}

\renewcommand{\arraystretch}{1.4}

\newcommand{\down}{\downarrow}
\newcommand{\Z}{\mathbb{Z}}
\renewcommand{\vec}[1]{\mathbf{#1}}
\newcommand{\N}{\mathbb{N}}
\newcommand{\R}{\mathbb{R}}
\newcommand{\C}{\mathbb{C}}
\newcommand{\Ry}{\mathit{Ry}}
\let\Im\relax
\DeclareMathOperator{\Im}{Im}
\DeclareMathOperator{\trace}{tr}
\newcommand{\abs}[1]{\lvert #1 \rvert}
\newcommand{\ket}[1]{\vert {#1} \rangle}
\newcommand{\bra}[1]{\langle #1 \vert}
\newcommand{\braket}[2]{\langle {#1} \vert {#2} \rangle}
\newcommand{\parbreak}{
	\begin{center}
		--- $\ast$ ---
	\end{center} 
}
\makeatletter
\newcommand*{\defeq}{\mathrel{\rlap{%
    \raisebox{0.3ex}{$\m@th\cdot$}}%
  \raisebox{-0.3ex}{$\m@th\cdot$}}%
	=
}
\makeatother

\newpagestyle{pagina}{
	\headrule
	\sethead*{\sffamily {\bfseries Topología de variedades:} Entrega 1}{}{\theauthor}
	\footrule
	\setfoot*{}{}{\sffamily \thepage}
}
\renewpagestyle{plain}{
	\footrule
	\setfoot*{}{}{\sffamily \thepage}
}
\pagestyle{pagina}

\title{\sffamily {\bfseries Topología de variedades:} Entrega 1}
\author{\sffamily Arnau Mas}
\date{\sffamily 4 de octubre de 2019}


\begin{document}
\maketitle

El espacio de matrices reales \( n \) por \( n \), es una variedad de dimensión \( n^2 \) difeomorfa a \( \R^{n\times n} \). Sobre \( M_n(\R) \) está definida la función determinante,
\begin{align*}
	\det \colon M_n(\R) \to \R
\end{align*}
que es una función lisa por ser polinomial. Por lo tanto existe su diferencial en una matriz \( A \), es decir la aplicación lineal
\begin{align*}
	T_A \det \colon T_A M_n(\R) \to T_{\det A} \R. 
\end{align*}
Calcularemos su diferencial en la matriz identidad, que denotaremos simplemente por \( 1 \). 

En primer lugar, el espacio tangente a \( M_n(\R) \) en cualquier punto es difeomorfo a \( M_n(\R) \) y igualmente, el espacio tangente a \( \R \) en cualquier punto es también \( \R \). Por lo tanto la diferencial del determinante en la identidad es una aplicación lineal \( T_1 \det \colon M_n(\R) \to \R \). Puesto que es lineal queda determinada por su acción sobre una base del espacio de salida \( M_n(\R) \). Las matrices con ceros en todas sus entradas salvo un 1 en la posición \( i,j \), \( E_{ij} \) son una base de \( M_n(\R) \). Por definición de la aplicación diferencial se tiene
\begin{equation*}
	T_1 \det(E_{ij}) = \lim_{t \to 0} \frac{\det(1 + tE_{ij}) - \det(1)}{t} = \lim_{t \to 0} \frac{\det(1 + tE_{ij}) - 1}{t}.
\end{equation*}

Distinguimos ahora dos casos. Cuando \( i = j \) la matriz \( 1 + tE_{ii} \) es diagonal y todas entradas son 1 salvo la de la posición \( i,i \) que es \( 1 + t \). El determinante de esta matriz es \( 1 + t \) por lo que
\begin{equation*}
	T_1 \det(E_{ii}) = \lim_{t \to 0} \frac{1 + t - 1}{t} = 1.
\end{equation*}

Por otro lado, cuando \( i \neq j \) la matriz \( 1 + tE_{ij} \) es la que resulta de sumar \( t \) veces la fila \( j \)-ésima de la matriz identidad a su fila \( i \)-ésima. Puesto que el determinante es multilineal y alternado en las filas, esta operación no altera el valor del determinante de una matriz, por lo que \( \det(1 + tE_{ij}) = \det(1) = 1 \). Luego
\begin{equation*}
	T_1 \det(E_{ij}) = \lim_{t \to 0} \frac{1 - 1}{t} = 0.
\end{equation*}
Es decir, \( T_1 \det(E_{ij}) \) es 1 cuando \( i = j \) y 0 en caso contrario. Así podemos escribir \( T_1 \det(E_{ij} = \delta_{ij} \) donde \( \delta_{ij} \) es la delta de Kronecker. 

Por linealidad podemos calcular \( T_1\det \) para cualquier matriz. Si \( A = \sum_{i = 1}^{n} \sum_{j = 1}^{n} a_{ij}E_{ij} \) entonces
\begin{equation*}
	T_1 \det(A) =	\sum_{i = 1}^{n} \sum_{j = 1}^{n} a_{ij} T_1 \det(E_{ij}) = \sum_{i = 1}^{n} \sum_{j = 1}^{n} a_{ij} \delta_{ij} = \sum_{i = 1}^{n} a_{ii} = \trace A. 
\end{equation*}
Es decir, la diferencial del determinante en la identidad es precisamente la traza.

\end{document}
