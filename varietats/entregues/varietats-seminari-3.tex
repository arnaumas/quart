\documentclass[12pt]{article}
\usepackage[utf8]{inputenc}
\usepackage[T1]{fontenc}
\usepackage[spanish]{babel}
\usepackage{lmodern}
\usepackage{geometry}
\usepackage{hyperref}
\usepackage[dvipsnames]{xcolor}
\usepackage[bf,sf,small,pagestyles]{titlesec}
\usepackage{titling}
\usepackage[font={footnotesize, sf}, labelfont=bf]{caption} 
\usepackage{siunitx}
\usepackage{graphicx}
\usepackage{tikz-cd}
\usetikzlibrary{babel}
\usepackage{booktabs}
\usepackage{amsmath,amssymb}
\usepackage[spanish,sort]{cleveref}
\usepackage{enumitem}

\geometry{
	a4paper,
	right = 2.5cm,
	left = 2.5cm,
	bottom = 3cm,
	top = 3cm
}
\renewcommand{\baselinestretch}{1.3}

\hypersetup{
	colorlinks,
	linkcolor = {red!50!blue},
	linktoc = page
}

\crefname{figure}{figura}{figures}
\crefname{table}{taula}{taules}
\numberwithin{table}{section}
\numberwithin{figure}{section}

\graphicspath{{./figs/}}


\renewcommand{\arraystretch}{1.4}


% CUSTOM COMMANDS FOR ALGEBRAIC TOPOLOGY
% ----------------------------------------------------------

% Restriction of a function
\newcommand{\rest}[1]{\raisebox{-.5ex}{$|$}_{#1}}

% Real numbers
\newcommand{\R}{\mathbb{R}}
\newcommand{\PR}{\mathbb{PR}}

% Rational numbers
\newcommand{\Q}{\mathbb{Q}}

% Complex numbers
\newcommand{\C}{\mathbb{C}}

% Natural numbers
\newcommand{\N}{\mathbb{N}}

% Integers
\newcommand{\Z}{\mathbb{Z}}

% Vector bold
\renewcommand{\vec}[1]{\mathbf{#1}}

% Span
\newcommand{\gen}[1]{\langle #1 \rangle}

% Set
\newcommand{\set}[1]{\{ #1 \}}

% Script A, B, M, P
\newcommand{\A}{\mathcal{A}}
\newcommand{\B}{\mathcal{B}}
\newcommand{\M}{\mathcal{M}}
\renewcommand{\P}{\mathcal{P}}
\renewcommand{\S}{\mathfrak{S}}

% Identity
\newcommand{\id}{\mathrm{id}}

% Kernel and image
\DeclareMathOperator{\im}{im}
\DeclareMathOperator{\coker}{coker}

% Absolute value
\newcommand{\abs}[1]{\lvert #1 \rvert}

% Norm
\newcommand{\norm}[1]{\lVert #1 \rVert}

% Category of Vector Spaces
\newcommand{\Vect}{\mathsf{Vect}}
\newcommand{\VectK}{\Vect_{K}}
\newcommand{\VectR}{\Vect_{\R}}
\DeclareMathOperator{\Hom}{Hom}
\DeclareMathOperator{\Bil}{Bil}
\newcommand{\dual}{^{\vee}}
\DeclareMathOperator{\tr}{tr}

% Category of Manifolds
\DeclareMathOperator{\Diff}{Diff}

% Epi and monomorphisms
\newcommand{\onto}{\twoheadrightarrow}
\newcommand{\into}{\tailrightarrow}

\newcommand{\parbreak}{
	\begin{center}
		--- $\ast$ ---
	\end{center} 
}

% Defined as
\makeatletter
\newcommand*{\defeq}{\mathrel{\rlap{%
    \raisebox{0.3ex}{$\m@th\cdot$}}%
  \raisebox{-0.3ex}{$\m@th\cdot$}}%
	=
}
\makeatother

% Support
\DeclareMathOperator{\supp}{supp}

% Categories
\newcommand{\Top}{\mathsf{Top}}


\newpagestyle{pagina}{
	\headrule
	\sethead*{\sffamily {\bfseries Topología de variedades:} Seminario 3}{}{\theauthor}
	\footrule
	\setfoot*{}{}{\sffamily \thepage}
}
\renewpagestyle{plain}{
	\footrule
	\setfoot*{}{}{\sffamily \thepage}
}
\pagestyle{pagina}

\title{\sffamily {\bfseries Seminario 3:} Cálculo de la cohomología de \( \PR^n \)}
\author{\sffamily Arnau Mas}
\date{\sffamily 10 de enero de 2019}


\begin{document}
\maketitle

Recordamos que el espacio proyectivo \( \PR^n \) \( n \)-dimensional es el espacio que se
obtiene al identificar los puntos antipodales de la \( n \)-esfera \( S^n \). Podemos
entender esto como el cociente de \( S^n \) por la acción de un grupo. En efecto,
consideremos la aplicación antipodal \( a \colon S^n \to S^n \), que es un difeomorfismo
de la esfera. Esta aplicación es una involución: \( a^2 = \id \), por lo que podemos
definir una acción de \( \Z/2\Z \) sobre \( S^n \) mediante \( 0 \mapsto \id \) y \( 1
\mapsto a \). La ación es libre puesto que \( a \) no tiene puntos fijos y también es
propia porque el grupo con el que actúamos es finito. El cociente de \( S^n \) por esta
acción es justamente \( \PR^n \) puesto que la órbita de cada punto está formada por él
mismo y su antipodal. En particular esto justifica que \( \PR^n \) es una variedad. 

\parbreak

Tenemos, pues, una proyección lisa \( p \colon S^n \to \PR^n \) que de hecho es una
submersión ---puesto que la acción es libre y propia---. Consideramos la sucesión
\begin{equation}\label{eq:sucesion exacta corta}
	\begin{tikzcd}
		0 \arrow[r] & \Omega^k(\PR^n) \arrow[r, "p^\ast"] & \Omega^k(S^n) \arrow[r, "\id -
		a^\ast"] & E^k \arrow[r] & 0
	\end{tikzcd}
\end{equation}
donde \( E^k = \im(\id - a^\ast) \subseteq \Omega^k(S^n) \). Comprovar la exactitud de
esta sucesión equivale a demostrar que \( p^\ast \) es inyectiva, que \( \id - a^\ast \)
es exhaustiva sobre \( E^k \) y que \( \im p^\ast = \ker(\id - a^\ast) \). Que \( \id -
a^\ast \) es exhaustiva sobre \( E^k \) es evidente por la definición de \( E^k \). Las
otras dos condiciones tienen más contenido.

Demostramos primero que \( \im p\ast = \ker(\id - a^\ast) \). Observemos que \( \omega \in
\ker(\id - a^\ast) \) si y solo si \( \omega = a^\ast \omega \). Sea \( \omega \im p^\ast
\). Entonces existe \( \theta \in \Omega^k(\PR^n) \) tal que \( \omega = p^\ast \theta \).
Luego
\begin{equation*}
	a^\ast \omega = a^\ast (p^\ast \theta) = (p \circ a)^\ast \theta = p^\ast \theta =
	\omega.
\end{equation*}
En la penúltima igualdad se utiliza que \( p \circ a = p \). Esto es porque la proyección
sobre el espacio proyectivo de dos puntos antipodales de la esfera es, por definición, la
misma. Tenemos, pues, \( \im p^\ast \subseteq \ker(\id - a^\ast) \).

En el otro sentido, sea \( \omega \in \Omega^k(S^n) \) una forma invariante por (el
pullback) de la aplicación antipodal. Tenemos que demostrar que existe \( \theta \in
\Omega^k(S^n) \) tal que \( p^\ast \theta = \omega \). Para poder construir esta \( \theta
\) primero observamos que la preimagen de cualquier punto de \( \PR^n \) por \( p \) está
formada por dos puntos de \( S^n \) que son antipodales. Este hecho se transfiere a la
aplicación tangente de \( p \), \( Tp \colon TS^n \to T\PR^n \):  como \( p \) es una
submersión, \( T_xp \) es exhaustiva en cada punto \( x \in S^n \), pero como \( S^n \)
y \( \PR^n \) tienen la misma dimensión se tiene que\( T_xp : T_x S^n \to T_{p(x)}\PR^n \)
es un isomorfismo para todo \( x \) de \( S^n \). Por ello existen, para todo \( y = p(x) \in
\PR^n \), isomorfismos \( T_xp \colon T_xS^n \to T_{p(x)}\PR^n \) y \( T_{a(x)}p
\colon T_{a(x)}S^n \to T_{p(x)}\PR^n \), donde usamos que \( p(a(x)) = p(x) \).

La invariancia de \( \omega \) por \( a^\ast \) se lee localmente como
\begin{equation*}
	\omega_x(v_1, \dots, v_k) = a^\ast\omega_{x}(v_1, \dots, v_k) = \omega_{a(x)}(T_xa(v_1),
	\dots, T_xa(v_k))
\end{equation*}
para todo \( x \in S^n \) y \( v_1, \dots, v_k \in T_xS^n \). Esto, junto con el
isomorfismo \( T_xp \), nos permite definir una forma \( \theta \in \Omega^k(\PR^n) \)
mediante
\begin{equation*}
	\theta_{[x]} (v_1, \dots, v_k) = \omega_x(T_xp^{-1}(v_1), \dots, T_xp^{-1}(v_k)).
\end{equation*}
Es justamente el hecho que \( a^\ast \omega = \omega \) lo que garantiza que esta es una
buena definición. En efecto, el otro representante de la clase \( [x] \) es \( a(x)
\), entonces
\begin{align*}
	\theta_{[a(x)]}(v_1, \dots, v_k) & = \omega_{a(x)}(T_{a(x)}p^{-1}(v_1), \dots,
	T_{a(x)}p^{-1}(v_k)) \\
																	 & = a^\ast\omega_{a(x)}(T_{a(x)}p^{-1}(v_1), \dots,
																	 T_{a(x)}p^{-1}(v_k)) \\
																	 & = \omega_x\left( (T_{a(x)}a \circ T_{a(x)}p^{-1})(v_1),
																	 \dots, (T_{a(x)}a \circ T_{a(x)}p^{-1})(v_k) \right).
\end{align*}
Ahora, de la identidad \( p \circ a = p \) se desprende \( Tp \circ Ta = Tp \). Por lo
tanto
\begin{equation*}
	T_{x} p \circ T_{a(x)} a = T_{a(x)} p
\end{equation*}
y puesto que tanto \( T_xp \) como \( T_{a(x)}p \) son isomorfismos
\begin{equation*}
	T_{a(x)}a \circ T_{a(x)}p^{-1} = T_xp^{-1}.
\end{equation*}
Luego
\begin{equation*}
	\theta_{[a(x)]}(v_1, \dots, v_k) = \omega_x(T_xp^{-1}(v_1), \dots, T_xp^{-1}(v_k)) =
	\theta_{[x]}(v_1, \dots, v_k).
\end{equation*}
Es decir, la definición de \( \theta \) no depende del representante, por lo que está bien
definida como \( k \)-forma de \( \PR^n \). Además, por construcción, el pullback de \(
\theta \) por \( p \) es \( \omega \). Efectivamente:
\begin{align*}
	p^\ast \theta_x(v_1, \dots, v_k) & = \theta_{p(x)}(T_xp(v_1), \dots, T_xp(v_k)) \\
																	 & = \theta_{[x]}(T_xp(v_1), \dots, T_xp(v_k)) \\
																	 & = \omega_x\left((T_xp^{-1}\circ T_xp)(v_1), \dots,
																	 (T_xp^{-1}\circ T_xp)(v_k)\right) \\
																	 & = \omega_x(v_1, \dots, v_k).
\end{align*}
Esto nos da la inclusión en el otro sentido, \( \ker(\id - a^\ast) \subseteq \im p^\ast
\), y por lo tanto \( \ker(\id - a^\ast) = \im p^\ast \).

Por último se tiene que comprovar que \( p^\ast \) es inyectiva, es decir, que \( \ker
p^\ast = 0 \). Sea, pues, \( \theta \in \ker p^\ast \). Entonces, para todo \( x \in S^n
\) y \( v_1, \dots, v_k \in T_xS^n \) se tiene
\begin{equation*}
	0 = p^\ast \theta_x(v_1, \dots, v_k) = \theta_{p(x)}(T_xp(v_1), \dots, T_xp(v_k)).
\end{equation*}
Como \( p \) es exhaustiva y para todo \( x \in S^1 \) \( T_xp \) es un isomorfismo,
podemos concluir que \( \theta \) también es la forma nula. Por lo tanto \( \ker(p^\ast) =
0 \) como queríamos.

\parbreak

Queremos ver como actúa la diferencial \( d \) sobre los espacios \( E^k \). Puesto que \(
E^k \subseteq \Omega^k(S^n) \) está claro que para todo \( \omega \in E^k \) se tiene \(
d\omega \in \Omega^{k+1}(S^n) \). En realidad se tiene algo más fino: \( d\omega \) está
en \( E^{k+1} \), no solamente en \( \Omega^{k+1}(S^n) \). Efectivamente, si \( \omega \in
E^k \) existe \( \sigma \in \Omega^k(S^n) \) tal que \( \omega = \sigma - a^\ast \sigma
\). Entonces
\begin{equation*}
	d\omega = d\sigma - d(a^\ast \sigma) = d\sigma - a^\ast d\sigma
\end{equation*}
porque la diferencial commuta con cualquier pullback. Entonces \( d\omega \) es la imagen
de \( d\sigma \) por \( \id - a^\ast \) por lo que \( d\omega \in E^{k+1} \). Entonces el
complejo \( \Omega^{\bullet}(S^n) \) se restringe al complejo \( E^\bullet \). Puesto que
la diferencial \( d \colon E^k \to E^{k+1} \) es una restricción de \( d \colon
\Omega^{k}(S^n) \to \Omega^{k+1}(S^n) \), se sigue cumpliendo la condición \( d^2 = 0 \),
es decir, \( E^\bullet \) es efectivamente un complejo de cocadenas.

\parbreak

La sucesión exacta corta en \eqref{eq:sucesion exacta corta} se transporta a una sucesión
exacta corta entre complejos,
\begin{equation*}
	\begin{tikzcd}
		0 \arrow[r] & \Omega^\bullet(\PR^n) \arrow[r, "p^\ast"] & \Omega^\bullet(S^n) \arrow[r, "\id -
		a^\ast"] & E^\bullet \arrow[r] & 0.
	\end{tikzcd}
\end{equation*}
Entonces, usando el lema del zig-zag, podemos escribir una sucesión exacta larga entre las
correspondientes cohomologías:
\begin{equation}\label{eq:sucesion exacta larga}
	\begin{tikzcd}
		0 \arrow[r] & H^0(\PR^n) \arrow[r, "p^{\ast \ast}"] & H^0(S^n) \arrow[r, "\id -
		a^{\ast \ast}"] & \makebox[\widthof{$H^{k+1}(E)$}][c]{$H^0(E)$}  \arrow[dll,
		"\qquad\cdots\qquad" {font=\normalsize, description, sloped}, out=0, in=180, looseness=1.6, overlay] \\
						 & \makebox[\widthof{$H^{k+1}(\PR^n)$}][c]{$H^k(\PR^n)$} \arrow[r, "p^{\ast \ast}"] & H^k(S^n) \arrow[r, "\id -
		a^{\ast \ast}"] & \makebox[\widthof{$H^{k+1}(E)$}][c]{$H^k(E)$} \arrow[dll, "\delta"
		description, out=0, in=180, looseness=1.6,
		overlay] \\
						 & H^{k+1}(\PR^n) \arrow[r, "p^{\ast \ast}"] & H^{k+1}(S^n) \arrow[r, "\id -
		a^{\ast \ast}"] & H^{k+1}(E) \arrow[dll, "\qquad\cdots\qquad" {font=\normalsize, description,
		sloped}, out=0, in=180, looseness=1.6, overlay] \\
						 & \makebox[\widthof{$H^{k+1}(\PR^n)$}][c]{$H^n(\PR^n)$} \arrow[r, "p^{\ast \ast}"] & H^n(S^n) \arrow[r,
			"\id - a^{\ast \ast}"] & \makebox[\widthof{$H^{k+1}(E)$}][c]{$H^n(E)$}  \arrow[r] & 0 \\		 
		\end{tikzcd}
\end{equation}
Los morfismos de conexión \( \delta \) vienen inducidos por la sucesión exacta.



\parbreak

La aplicación antipodal da lugar de forma natural a dos clases de formas diferenciales
sobre \( S^n \). Por un lado, las formas \emph{invariantes}, que cumplen \( a^\ast \omega
= \omega \). El conjunto de estas formas es \( \ker(\id - a^\ast) \), que, por lo que
hemos visto previamente, es igual al conjunto de formas que vienen inducidas por una forma
de \( \PR^n \), \( \im p^\ast \). Por otro lado existen las formas antiinvariantes, que
cumplen \( a^\ast \omega = -\omega \). Se tiene que el conjunto de \( k \)-formas antiinvariantes
es justamente \( E^k \). Recordemos que \( E^k = \im(\id - a^\ast) \), por lo que si \(
\omega \in E^k \) entonces existe \( \sigma \in \Omega^k(S^n) \) tal que \( \omega =
\sigma - a^\ast \sigma \) y
\begin{equation*}
	a^\ast \omega = a^\ast \sigma - a^\ast a^\ast \sigma = a^\ast \sigma - \sigma =
	-\omega.
\end{equation*}
Y en el otro sentido, si \( \omega \) es antiinvariante tenemos
\begin{equation*}
	(\id - a^\ast)\frac{\omega}{2} = \frac{\omega}{2} - a^\ast \frac{\omega}{2} =
	\frac{\omega}{2} + \frac{\omega}{2} =\omega
\end{equation*}
luego \( \omega \in E_k \).

Podemos descomponer toda forma de \( \Omega^k(S^n) \) en la suma de una forma invariante y
otra antiinvariante definiendo \( \omega_i = \frac{\omega + a^\ast \omega}{2} \) y \(
\omega_a = \frac{\omega - a^\ast \omega}{2} \). Entonces \( \omega = \omega_i + \omega_a
\) y se comprueba fácilmente que \( \omega_i \) es invariante mientras que \( \omega_a \)
es antiinvariante. Esta descomposición es única, pues si una forma descompusiese de dos
formas distintas, \( \omega = \omega_i + \omega_a = \tilde{\omega}_i + \tilde{\omega}_a \)
entonces tendríamos \( \tau = \omega_i - \tilde{\omega}_i = \tilde{\omega}_a - \omega_a
\). La forma \( \tau \) sería invariante y antiinvariante, lo que significa que tiene que
ser la forma nula, puesto que \( \tau = -\tau \). Así \( \omega_i = \tilde{\omega}_i \) y
\( \omega_a = \tilde{\omega}_a \). Esta descomposición tambien se puede
expresar, por las observaciones que hemos hecho previamente como
\begin{equation*}
	\Omega^k(S^n) = \ker(\id - a^\ast) \oplus \im(\id - a^\ast).
\end{equation*}

\parbreak

Consideremos la cohomología de las formas antiinvariantes de \( S^n \), es decir, la
cohomología del complejo \( E^\bullet \). Sabemos que los únicos grupos de cohomología no
triviales de la \( n \)-esfera son \( H^0(S^n) \) y \( H^n(S^n) \), ambos isomorfos a \(
\R \). Podemos utilizar esto para calcular \( H^k(E) \). Sea \( \omega \in Z^k(E) \), es
decir, una forma cerrada antiinvariante. Puesto que \( \omega \) es en particular una
forma cerrada de
\( S^n \) y \( H^k(S^n) \) es trivial, existe \( \eta \in \Omega^{k-1}(S^n) \) tal que \(
d\eta = \omega \). Si podemos demostrar que \( \eta \) es antiinvariante ---esto es \( \eta \in
E^{k-1} \)--- entonces tendremos que todo forma cerrada antiinvariante es una forma
exacta
antiinvariante. Es decir, tendremos \( Z^k(E) = B^k(E) \) y por lo tanto \( H^k(E) = 0 \).  

La descomposición de \( \eta \) en sus partes invariante e antiinvariante es 
\begin{equation*}
	\eta = \frac{\eta + a^\ast \eta}{2} + \frac{\eta - a^\ast \eta}{2},
\end{equation*}
donde el primer término es la parte invariante, \( \eta_i \), y el segundo la
antiinvariante, \( \eta_a \). Si aplicamos \( d \) obtenemos
\begin{align*}
	d\eta & = d\left(\frac{\eta + a^\ast \eta}{2}\right) + d\left(\frac{\eta - a^\ast
	\eta}{2}\right) \\
				& =	\frac{d\eta + d(a^\ast \eta)}{2} + \frac{d\eta - d(a^\ast \eta)}{2} \\
				& = \frac{d\eta + a^\ast d\eta}{2} + \frac{d\eta - a^\ast d\eta}{2} \\
\end{align*}
es decir, \( (d\eta)_i = d(\eta_i) \) y \( (d\eta)_a = d(\eta_a) \). Pero como \( d\eta =
\omega \) deducimos \( d(\eta_i) = \omega_i \) y \( d(\eta_a) = \omega_a \). Al ser \(
\omega \) antiinvariante, \( \omega_i = 0 \) y \( \omega_a = \omega \). Entonces \( \omega
= d(\eta_a) \): hemos escrito \( \omega \) como la imagen por \( d \) de una forma
antiinvariante, \( \eta_a \). Es decir, \( \omega \) es cerrada. Por lo tanto \( Z^k(E)
= B^k(E) \) y \( H^k(E) = 0 \).

\parbreak

A continuación calculamos la cohomología de de Rham de \( \PR^n \) usando los resultados
anteriores. \( \PR^n \)es conexo puesto que \( \PR^n \) es un cociente de \( S^n \), que
es también conexo. Entonces su primer grupo de cohomología es 
\begin{equation*}
	H^0(\PR^n) \cong \R.
\end{equation*}
Para los órdenes intermedios \( 0 < k < n \) ``recortamos'' la sucesión exacta larga de
\eqref{eq:sucesion exacta larga} y obtenemos las sucesions exactas
\begin{equation*}
	\begin{tikzcd}
		H^{k-1}(E) \arrow[r, "\delta"] & H^{k}(\PR^n) \arrow[r, "p^{\ast \ast}"] & H^k(S^n)
	\end{tikzcd}
\end{equation*}
Para \( 1 < k < n \) hemos calculado \( H^{k-1}(E) = H^{k}(S^n) = 0 \), por lo que
deducimos que \( H^k(\PR^n) = 0 \) para \( 1 < k < n \). En el caso \( k = 1 \) se obtiene
el mismo resultado, pero no de forma immediata. Recordemos que las formas cerradas de \(
\Omega^0(S^n) \) son las formas constantes, mientras que la única forma exacta es la forma
nula, que es invariante y antiinvariante a la vez. Consideremos una forma cerrada
antiinvariante, \( f \). Por ser cerrada debe ser constante, pongamos \( f(x) = C \) para
todo \( x \in S^n \). Pero por ser antiinvariante se tiene
\begin{equation*}
	C = f(x) = -f(a(x)) = -C
\end{equation*}
por lo que \( C = 0 \). Es decir, la única 0-forma cerrada antiinvariante es la aplicación
nula. En consecuencia \( H^0(E) = 0 \), por lo que también podemos concluir \( H^1(\PR^n)
= 0 \).

Queda por calcular el grupo \( H^n(\PR^n) \). Para ello tomamos el final de
\eqref{eq:sucesion exacta larga} y obtenemos la sucesión exacta corta
\begin{equation*}
	\begin{tikzcd}
		0 \arrow[r] & H^n(\PR^n) \arrow[r, "p^{\ast \ast}"] & H^n(S^n) \cong \R \arrow[r, "\id -
		a^{\ast \ast}"] & H^n(E) \arrow[r] & 0.
	\end{tikzcd}
\end{equation*}
Por exactitud deducimos
\begin{equation*}
	H^n(\PR^n) \cong \im p^{\ast \ast}= \ker(\id - a^{\ast \ast}),
\end{equation*}
es decir, la cohomología en dimensión \( n \) de \( \PR^n \) es la cohomología de las \(
n \)-formas invariantes de \( S^n \). 

La aplicación antipodal preserva la orientación para \( n \) impar y la invierte para \( n
\) par. Efectivamente, el determinante de la lectura de \( T_xa \) en las cartas inducidas
por la inclusión \( S^n \rightarrowtail \PR^n \) es \( (-1)^{n+1} \) en todo \( x \in S^n
\). De esto deducimos, cuando \( n \) es impar, que para una forma antiinvariante \( \omega \) 
\begin{equation*}
	\int_{S^1} \omega = \int_{a(S^1)} a^\ast \omega = \int_{S^1} -\omega = -\int_{S^1}
	\omega
\end{equation*}
por lo que las formas antiinvariantes tienen integral 0 en dimensión impar. En cambio, en
dimensión par son las formas invariantes las que tienen integral nula. En efecto, puesto
que en este caso \( a \) invierte la orientación,
\begin{equation*}
	\int_{S^1} \omega = -\int_{a(S^1)} a^\ast \omega = - \int_{S^1} \omega.
\end{equation*}

Por otro lado, puesto que \( S^n \) es orientable existe una forma volumen \( \omega_0 \in
\Omega^n(S^n) \) cuya integral sobre \( S^n \) no puede ser 0, por lo que, por el teoerema
de Stokes, usando que \( S^n \) no tiene frontera, \( \omega_0 \) no puede ser exacta.  Al
estar en la dimensión máxima, toda forma de \( S^n \) es cerrada.  Y como \( H^n(S^n)
\cong \R \) se deduce que \( H^n(S^n) = \gen{[\omega_0]} \). Descompongamos \( \omega_0 \)
en sus partes invariante y antiinvariante:
\begin{equation*}
	\omega_0 = \omega_0^i + \omega_0^a,
\end{equation*}
por lo que
\begin{equation*}
	\int_{S^1} = \int_{S^1} \omega_0^i + \int_{S^1} \omega_0^a.
\end{equation*}
En dimensión impar tenemos que \( \int_{S^1} \omega_0^a = 0 \) luego \( \int_{S^1}
\omega_0^i = \int_{S^1} \omega_0 \neq 0 \), por lo que \( \omega_0^i \) es una forma
invariante cerrada pero no exacta, luego \( H^n(\PR^n) \cong \ker(\id - a^\ast) =
\gen{[\omega_o^i]} \cong \R \).

En dimensión par, en cambio no hay formas invariantes no exactas. Efectivamente, como \(
H^n(S^n) = \gen{\omega_0} \), podemos escribir, para cualquier forma \( \omega \in
\Omega^n(\PR^n) \), \( \omega = \lambda \omega_0 + d\alpha \) para  \( \lambda \in \R \) y
\( \alpha \in \Omega^{n-1}(S^n) \). Entones, si \( \omega \) es invariante
\begin{equation*}
	0 = \int_{S^1} \omega = \lambda \int_{S^1} \omega_0 + \int_{S^1} d\alpha = \lambda
	\int_{S^1} \omega_0,
\end{equation*}
usando el teorema de Stokes para concluir \( \int_{S^1} d\alpha = 0 \). Entonces, como	\(
\int_{S^1} \omega_0 \neq 0 \) se sigue \( \lambda = 0 \). Es decir, toda \( n \)-forma
invariante es exacta si \( n \) es par. \( \alpha \) no necesariamente es una forma
invariante, pero vimos anteriormente que \( (d\alpha)_i = d(\alpha_i) \) por lo que \(
\omega = d(\alpha_i) \) y \( \omega \) es una forma exacta de \( \ker(\id - a^{\ast
\ast}) \). Por lo tanto \( H^n(\PR^n) \cong \ker(\id - a^{\ast \ast}) = 0 \). 

Todo esto queda resumido como
\begin{equation*}
	H^k(\PR^n) \cong 
	\begin{cases}
		\R, & \text{cuando } k = 0 \\
		0, & \text{cuando } 0 < k < n \\
		\R, & \text{cuando } k = n \text{ y } n \text{ impar} \\
		0, & \text{cuando } k = n \text{ y } n \text{ par}
	\end{cases}
\end{equation*}

\end{document}
