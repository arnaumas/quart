\documentclass[12pt,twoside]{book}

% PREAMBLE FOR RIEMANNIAN GEOMETRY NOTES

% ----------------------------------------------------------
% Packages
\usepackage[utf8]{inputenc}
\usepackage[T1]{fontenc}
\usepackage[english]{babel}
\usepackage{lmodern}
\usepackage{geometry}
\usepackage{hyperref}
\usepackage[dvipsnames]{xcolor}
\usepackage[bf,sf,small,pagestyles]{titlesec}
\usepackage{titling}
\usepackage[font={footnotesize, sf}, labelfont=bf]{caption} 
\usepackage{siunitx}
\usepackage{graphicx}
\usepackage{tikz-cd}
\usepackage{booktabs}
\usepackage{amsmath,amssymb}
\usepackage[sort]{cleveref}
\usepackage{amsthm,thmtools}
\usepackage[shortlabels]{enumitem}

% ----------------------------------------------------------
% Geometry setup
\geometry{
	a4paper,
	right = 3cm,
	left = 3cm,
	bottom = 3cm,
	top = 3cm
}
% Wider space between lines
\renewcommand{\baselinestretch}{1.3}

% ----------------------------------------------------------
% Hyperref setup
\hypersetup{
	colorlinks,
	linkcolor = {red!50!blue},
	linktoc = page
}
\numberwithin{table}{section}
\numberwithin{equation}{section}
\numberwithin{figure}{section}

% ----------------------------------------------------------
% Definition of theorem, example, etc environments
\newcommand{\qedtriangle}{\ensuremath{\triangle}}
\newcommand{\qedtriangledown}{\ensuremath{\bigtriangledown}}
\declaretheoremstyle[spaceabove=6pt, spacebelow=6pt, headfont=\bfseries, notefont=\normalfont, notebraces={(}{)}, qed=\qedtriangle]{definition}
\declaretheoremstyle[spaceabove=6pt, spacebelow=6pt, headfont=\bfseries, notefont=\normalfont, notebraces={(}{)}, qed=\qedtriangledown]{example}

\declaretheorem[name=Theorem, refname={theorem,theorems}, Refname={Theorem,Theorem},
numberwithin=chapter]{theorem}
\declaretheorem[name=Proposition, refname={proposition,propositions},
Refname={Proposition,Propositions}, numberlike=theorem]{proposition}
\declaretheorem[name=Lemma, refname={lemma,lemmas},
Refname={Lemma,Lemmas}, numberlike=theorem]{lemma}
\declaretheorem[name=Definition, style=definition, refname={definition,definitions},
Refname={Definitio,Definitions}, numberwithin=chapter]{definition}
\declaretheorem[name=Example, style=example, refname={example,examples},
Refname={Example,Examples}, numberwithin=chapter]{example}

% ----------------------------------------------------------
% Definition of custom list style
\newlist{points}{enumerate}{1}
\setlist[points,1]{label=\textup{(}{\itshape \roman*}\textup{)}, wide}

% ----------------------------------------------------------
% Definition of page styles
\newpagestyle{page}[\sffamily \footnotesize]{
	\headrule
	\sethead*{\ifthesection{{\bfseries \thesection} \sectiontitle}{}}{}{{\bfseries Chapter \thechapter.} \chaptertitle}
	\footrule
	\setfoot*{}{}{\thepage}
}
\renewpagestyle{plain}[\sffamily \footnotesize]{
	\footrule
	\setfoot*{}{}{\thepage}
}
\pagestyle{page}

% ----------------------------------------------------------
% Format of chapter titles
\titleformat{\chapter}[block]{\sffamily \bfseries \Huge}{\filleft \large Chapter \Huge \thechapter\\}{0pt}{\Huge \titlerule[1pt] \vspace{1ex} \filleft}



% CUSTOM COMMANDS FOR ALGEBRAIC TOPOLOGY
% ----------------------------------------------------------

% Restriction of a function
\newcommand{\rest}[1]{\raisebox{-.5ex}{$|$}_{#1}}

% Real numbers
\newcommand{\R}{\mathbb{R}}
\newcommand{\PR}{\mathbb{PR}}

% Rational numbers
\newcommand{\Q}{\mathbb{Q}}

% Complex numbers
\newcommand{\C}{\mathbb{C}}

% Natural numbers
\newcommand{\N}{\mathbb{N}}

% Integers
\newcommand{\Z}{\mathbb{Z}}

% Vector bold
\renewcommand{\vec}[1]{\mathbf{#1}}

% Span
\newcommand{\gen}[1]{\langle #1 \rangle}

% Set
\newcommand{\set}[1]{\{ #1 \}}

% Script A, B, M, P
\newcommand{\A}{\mathcal{A}}
\newcommand{\B}{\mathcal{B}}
\newcommand{\M}{\mathcal{M}}
\renewcommand{\P}{\mathcal{P}}
\renewcommand{\S}{\mathfrak{S}}

% Identity
\newcommand{\id}{\mathrm{id}}

% Kernel and image
\DeclareMathOperator{\im}{im}
\DeclareMathOperator{\coker}{coker}

% Absolute value
\newcommand{\abs}[1]{\lvert #1 \rvert}


% Category of Vector Spaces
\newcommand{\Vect}{\mathsf{Vect}}
\newcommand{\VectK}{\Vect_{K}}
\newcommand{\VectR}{\Vect_{\R}}
\DeclareMathOperator{\Hom}{Hom}
\DeclareMathOperator{\Bil}{Bil}
\newcommand{\dual}{^{\vee}}
\DeclareMathOperator{\tr}{tr}
\DeclareMathOperator{\Sym}{Sym}

% Category of Manifolds
\DeclareMathOperator{\Diff}{Diff}

% Epi and monomorphisms
\newcommand{\onto}{\twoheadrightarrow}
\newcommand{\into}{\tailrightarrow}

\newcommand{\parbreak}{
	\begin{center}
		--- $\ast$ ---
	\end{center} 
}

% Defined as
\makeatletter
\newcommand*{\defeq}{\mathrel{\rlap{%
    \raisebox{0.3ex}{$\m@th\cdot$}}%
  \raisebox{-0.3ex}{$\m@th\cdot$}}%
	=
}
\makeatother

% Support
\DeclareMathOperator{\supp}{supp}

% Categories
\newcommand{\Top}{\mathsf{Top}}

% LINEAR ALGEBRA
% ----------------------------------------------------------
\newcommand{\inn}[2]{\langle #1, #2 \rangle}
\newcommand{\norm}[1]{\left\lVert #1 \right\rVert}

\newcommand{\coord}[2]{\frac{\partial}{\partial #1}\rest{#2}}

% Standard metric
\newcommand{\gst}{g_\text{st}}

% Integration d
\newcommand{\ud}{\, \mathrm{d}}



\graphicspath{{./figs/}}
\newcommand{\dummyfig}[1]{
  \centering
  \fbox{
    \begin{minipage}[c][0.33\textheight][c]{0.5\textwidth}
      \centering{\ttfamily #1}
    \end{minipage}
  }
}

\title{Riemannian Geometry}
\author{Arnau Mas}
\date{2020}

\begin{document}
\maketitle

\frontmatter
\pagestyle{plain}
These are notes gathered during the subject \emph{Riemannian Geometry} as taught by
Florent Balacheff between February 2020 and June 2020.

\mainmatter
\chapter{Riemannian Manifolds}
\section{Inner products}
Recall that an inner product on a real vector space \( V \) is a symmetric bilinear form
\( \phi \colon V \times V \to \R \) which is positive definite, meaning that for all \( v
\in V \) one has \( \phi(v,v) > 0 \). An inner product is also often written as \(
\inn{u}{v} \). Equipping a vector space with an inner product is, in some sense, giving it
a notion of geometry. Following the usual euclidean case, we define the \emph{length} or
\emph{norm} of a vector as \( \norm{v} = \sqrt{\inn{v}{v}} \) and the \emph{angle} between
two vectors by 
\begin{equation*}
	\cos{\theta} = \frac{\inn{u}{v}}{\norm{u}\norm{v}}.
\end{equation*}
Positive-definiteness is essential for both of these definitions. In fact, as a
consequence of Sylvester's theorem, the geometry of a vector space with an
inner product isn't much different from that of ordinary euclidean space. Indeed,
for any inner product there exists an \emph{orthonormal} basis, i.e. a basis \( e_1,
\dots, e_n \) such that
\begin{equation*}
	\inn{e_i}{e_j} = \delta_{ij}.
\end{equation*}

\section{Riemannian Metrics}
Riemannian manifolds are smooth manifolds with extra structure which allows us to speak
of, not only their topology, but also their geometry. Think of the difference between a
balloon which can be deformed freely and a sphere with a rigid shape. The first one is \(
S^2 \) as a naked smooth manifold and the second one is \( S^2 \) as a Riemannian
manifold. What is this extra structure? A Riemannian metric. Informally, a Riemannian
metric is an inner product at every tangent space of the manifold which varies smoothly
along the manifold. 

\begin{definition}[Riemannian metric]
	A Riemannian metric \( g \) on a manifold \( M \) is a smooth section of the bundle \(
	\Sym^2 M \) of the symmetric bilinear forms on \( M \) such that for every \( x \in M \)
	\( g_x \) is positive definite. 
\end{definition}

\subsection{Components of a metric in local coordinates}
Given a local chart \( (U, \phi) \)

\subsection{The pullback of a metric}
Given an immersion \( f \colon M \to N \) and a 

The pullback gives a succint characterisation of local isometries: a local diffeomorphism
\( f \colon (M,g) \to (N,h) \) is a local isometry if and only if \( f^\ast h = g \). 

\begin{example}
	By considering manifolds embedded into \( \R^n \) and pulling back the euclidean metric
	onto them we obtain many examples of Riemannian manifolds. 
	Consider the \( n \)-sphere \( S^n \) embedded into \( \R^{n+1} \) as 
	\begin{equation*}
		S^n = \set{x \in \R^n \mid \norm{x} = 1}
	\end{equation*}
	The pullback of the euclidean metric along this inclusion gives \( S^n \) a Riemannian
	metric which we will refer to as its \emph{standard} or \emph{canonical} metric. 
\end{example}

This also shows that any manifold admits a metric. By Whitney's Embedding Theorem, we know
that any manifold can be immersed into \( \R^n \) for a sufficiently large \( n \), and
then we just pull back the euclidean metric along this immersion. We can give an
alternative proof of this fact which does not rely on Whitney's Theorem. 

\begin{theorem}
	 Any smooth manifold admits a Riemannian metric. 
\end{theorem}
\begin{proof}
	Consider a smooth manifold \( M \) and a local chart \( (U, \phi) \). It is clear that
	\( \phi \) gives an immersion of \( U \) into \( \R^n \), so we can pull back the
	euclidean metric onto \( U \). What we need, however, is a way of turning this into a
	global metric. The tool for this job are partitions of unity. Then let \( \set{(U_\alpha,
	\phi_\alpha)}_{\alpha \in I} \) be a locally finite atlas of \( M \) and \( \rho_\alpha
	\) a partition of unity subordinate to this atlas. Then we define \( g \) as
	\begin{equation*}
		g_x(X,Y) = \sum_{\alpha \in I} \rho_\alpha(x) (\phi_\alpha^{\ast}\gst)_x(X,Y)
	\end{equation*}
	for all \( x \in M \) and \( X,Y \in T_xM \). This is a well-defined sum since \( \rho
	\) is a partition of unity. Since every \( \phi_\alpha^\ast \gst \) is a symmetric
	bilinear form then so is \( g \). And \( g \) is also positive definite. Indeed, for
	every non-zero \( X \in T_xM \) there exists at least \( \alpha_0 \in I \) such that \(
	(\phi^\ast_{\alpha_0} \gst)_x(X,X) > 0 \) which means
	\begin{equation*}
		g_x(X,X) \geq \rho_{\alpha_0}(x) (\phi^\ast_{\alpha_0} \gst)_x(X,X) > 0.
	\end{equation*}

	It remains to be shown that \( g \) is globally smooth. Consider at the point \( x \)
	the coordinate vectors \( \coord{x_i}{x} \) corresponding to the chart \(
	\phi_{\alpha_0} \). Then we have
	\begin{equation*}
		g_x\left(\coord{x_i}{x}, \coord{x_j}{x}\right) = \sum_{\alpha \in I} \rho_\alpha(x)
		(\phi_\alpha^\ast \gst)_x\left(\coord{x_i}{x}, \coord{x_j}{x}\right)
	\end{equation*}
\end{proof}

\subsection{The product metric}	
\begin{definition}[Product metric]
	Given two Riemannian manifolds \( (M,g) \) and \( (N,h) \), we define the \emph{product
	metric} on the product manifold \( M \times N \) by
	\begin{equation*}
		(g \times h)_{(x,y)}\big((X_1, Y_1), (X_2,Y_2)\big) \defeq g_x(X_1, X_2) + h_y(Y_1,Y_2)
	\end{equation*}
	where \( x \in M \), \( y \in N \), \( X_1,X_2 \in T_xM \) and \( Y_1,Y_2 \in T_yN \). 
\end{definition}	
\begin{proposition}
	The product metric is indeed a Riemannian metric. 
\end{proposition}	
\begin{proof}
	Binearity and symmetry
\end{proof}

\begin{example}
	Consider the cylinder as submanifold of \( \R^3 \), so the set
	\begin{equation*}
		\set{(x,y,z) \in \R^3 \mid x^2 + y^2 = 1} \subset \R^3
	\end{equation*}
	with the corresponding induced metric. Another representation of the cylinder is as the
	product \( S^1 \times \R \). The product metric for this manifold can be written as \(
	dg_0 + dt^2 \) where \( dg_0 \) is the standard metric for \( S^1 \) and \( dt^2 \) is
	the euclidean metric on \( \R \). These two representations are isometric. 
\end{example}

\subsection{Quotient metrics}

\begin{definition}[Riemannian covering map]
	Given two Riemannian manifolds \( (M,g) \) and \( (N,h) \), a covering map \( p \colon M
	\to N \) is called \emph{Riemannian} provided
	\begin{points}
	\item \( p \) is a smooth covering map, meaning \( p \) is a surjection and for every \( y \in
		N \) there is an open neighbourhood \( U \) of \( y \) such that \( p^{-1}(U) \) is a
		disjoint union of open subsets of \( M \) all diffeomorphic to \( U \), i.e.
		\begin{equation*}
			p^{-1}(U) = \bigsqcup_{\alpha \in I} U_\alpha
		\end{equation*}
		and \( p\rest{U_\alpha} \colon U_\alpha \to U \) is a diffeomorphism for every \( \alpha \in I \) 
	\item and more than that, the \( U_\alpha \) are all isometric to \( U \), which can be
		written ad \( p^\ast h = g \). 
	\end{points}
\end{definition}

Typically, covering map arise through the action of a group on a manifold, as the following
result shows.
\begin{proposition}
	Let \( (M,g) \) be a Riemannian manifold and \( G \) a subgroup of the isometry group of
	\( M \) such that its action is 
	\begin{points}
	\item free, so that if \( g \in G \) has a fixed point then \( g \) must be the
		identity,
	\item proper\footnote{This condition guarantees that the quotient will be
		a Hausdorff space}, meaning for every \( x, y \in M \) there is an open
		neighbourhood \( V \) of \( x \) and \( W \) of \( y \) such that for every \( g \in
		G \)
		\begin{equation*}
			gV \cap W = \emptyset.  
		\end{equation*}
	\end{points}
	Then the quotient manifold \( M/G \) admits a unique Riemannian metric \( g_G \) such that
	the projection map
	\begin{equation*}
		p \colon (M,g) \to (M/G, g_G)
	\end{equation*}
	is a Riemannian covering map. 
\end{proposition}
\begin{proof}
	It is a result from differential topology that \( p \) is indeed a smooth covering map.
	We will show the construction of the induced metric. Take a point \( x \in M/G \). Then
	it has a preimage \( y \in M \) and there is a neighbourhood \( U \) of \( y \) such
	that \( p\rest{U} \) is a diffeomorphism. Then define a metric on \( x \) by \(
	p^{-1}\rest{U}^\ast g \) where \( g \) is the metric on \( M \). This construction does
	not depend on the choice of the preimage \( y \). Indeed, 


	We know that the map \( p \) is a smooth covering. Consider \( x \in M/G \) and two of
	its preimages, \( y_1 \in U_1 \subseteq M \) and \( y_2 \in U_2 \subseteq M \). Since \(
	p(y_1) = p(y_2) = x \), these two points are 

	SKETCH:
	
	1. Define the metric at \( x \in M/G \) as the metric at one of its preimages.
	 
	2. The fact that every preimage of \( x \) has isometric neighbourhoods shows
	well-definition. 
\end{proof}

\begin{example}[The metric of \( \PR^n \)]
	Recall that the projective space \( \PR^n \) can be constructed as the quotient of \(
	S^n \) by the action of the group \( \gen{a} \) where \( a \) is the antipodal map. If
	we give \( S^n \) its standard metric then it is easy to show that \( a \) is not only a
	diffeomorphism but also an isometry. The action is free and proper so \( \PR^n =
	S^n/\gen{a} \) is a smooth manifold. The induced metric on \( \PR^n \) is called the
	standard or canonical metric on \( \PR^n \). 
\end{example}

\begin{example}[Flat tori]
	The \( n \)-dimensional torus can be given its standard metric as the product metric of
	the spheres. 

	Translations are isometries of euclidean space ---so \( \R^n \) with the standard
	metric---. If \( v_1, \dots, v_n \) is a basis of  \( \R^n \) then the group of
	translations \( \Gamma = \gen{T_{v_1}, \dots, T_{v_n}} \) is a group of isometries of \(
	\R^n \). Furthermore, it is easy to show that its action is free and proper, so that \(
	\R^n/\Gamma \) is a smooth manifold which we equip with its induced metric.
	Topologically, this space is homeomorphic to an \( n \)-torus. 
\end{example}

\section{Lengths in a Riemannian manifold}
Once we have a Riemannian metric on a manifold we can begin to speak of length and volume.

\subsection{Length of a curve}
\begin{definition}[Length of a curve]
	Let \( (M,g) \) be a connected Riemannian metric and \( \gamma \colon [a,b] \to M \) a a
	piecewise differentiable path\footnote{Meaning \( \gamma \) is continous and there
	exists a partition \( a = a_0 < a_1 < \dots < a_n = b \) of the interval \( [a,b] \)
such that \( \gamma\rest{[a_i, a_{i+1}} \) for \( 0 \leq i \leq n-1 \)}. Then the length
of \( \gamma \) is
	\begin{equation*}
		L(\gamma) \defeq \int_a^b \sqrt{g_{\gamma(t)}(\dot{\gamma}(t), \dot{\gamma}(t))} \ud t
	\end{equation*}
	where \( \dot{\gamma}(t) \) is \( T_t \gamma (1) \), the vector tangent to the curve at
	the point \( t \). The quantity \( \sqrt{g_x(X,X)} \) is of course the norm of the
	vector \( X \) induced by the metric \( g \) and is usually written \( \norm{X}_g \).
\end{definition}
In the case where \( M = \R^n \) and the metric is the euclidean metric this coincides
with the usual notion of the length of a curve obtained by taking the supremum of the
length of every rectification of the curve.  

\begin{proposition}
	The length of a curve is independent of its parametrisation.
\end{proposition}
\begin{proof}
	Take a curve \( \gamma \colon [a,b] \to M \). A reparemetrisation of \( \gamma \) is
	another curve \( \gamma \circ h \) where \( h \colon [c,d] \to [a,b] \) is a
	diffemorphism. Then
	\begin{align*}
		L(\gamma \circ h) &= \int_{c}^d \norm{(\gamma \circ h)'(t)}_g \ud t \\
											&= \int_c^d \norm{h'(t) \dot{c}(h(t))}_g \ud t \\
											& = \int_{h^{-1}(a)}^{h^{-1}(b)} \norm{\dot{\gamma}(h(t))}_g
											\abs{h'(t)} \ud t \\
											& = \int_a^b \norm{\dot{\gamma}(t)}_g \ud t \\
											& = L(\gamma)
	\end{align*}
	where in the last step we used the Change of Variable Theorem.
	% TODO: Comentar el cas en el que \( h \) és decreixent
\end{proof}

\begin{definition}[Energy of a curve]
	Following ideas from Physics one defines the \emph{energy} of a curve by
	\begin{equation*}
		E(\gamma) \defeq \int_a^b g_{\gamma(t)}(\dot{\gamma}(t), \dot{\gamma}(t)) \ud t.
	\end{equation*}
	As opposed to the length, the energy does depend on the parametrisation. 
\end{definition}
A quick application of the Cauchy-Schwarz inequality gives the result
\begin{equation*}
	L(\gamma)^2 \leq E(\gamma)(b - a)
\end{equation*}
with equality if a nd only if \( \norm{\dot{\gamma}(t)}_g \) is constant along the curve.

\subsection{The distance defined by a metric}
\begin{theorem}
	Let \( (M,g) \) be a connected Riemannian manifold. The map 
	\begin{align*}
		d_g \colon M \times M & \longrightarrow \R \\
		(x,y) & \longmapsto \inf \set{L(\gamma) \mid \forall \gamma \colon [a,b] \to M \colon
		\gamma(a) = x, \gamma(b) = y}
	\end{align*}
	is a distance and the metric space \( (M, d_g) \) has the same topology as the original
	topology of \( M \). 
\end{theorem}
\begin{proof}
	Let's first show that \( d_g \) is well-defined, meaning it is finite for any pair \(
	x,y \in M \).
	
	Consider the set \( U_x \) of all points of \( M \) such that there exists a piecewise
	differentiable curve between it and \( x \). We will show that it is both open and
	closed and therefore equal to \( M \) since \( M \) is connected\footnote{This is
	essentially a proof that a connected manifold is path-connected}. 
	
	We first show \( U_x
	\) is open. Take \( y \in U_x \). Then there is a chart \( (U, \phi) \) around \( y \)
	such that \( \phi(U) \) is connected, therefore path-connected. Then for any \( y' \in U
	\) there is a path \( \gamma \colon \colon [b, b+\epsilon] \colon \phi(U) \) such that
	\( \gamma(b) = \phi(y) \) and \( \gamma(b+\epsilon) = \phi(y') \). Then \( \phi^{-1}
	\circ \gamma \) is a path in \( U \) which connects \( y \) and \( y' \). Then by
	concatenating the path that connects \( x \) and \( y \) with \( \phi^{-1} \circ \gamma
	\) we get a path connecting \( x \) to \( y' \). This means \( U \subseteq U_x \) and so
	\( U \) is an open neighbourhood of \( y \) in \( U_x \), therefore \( U_x \) is open. 

	Now 
	\begin{equation*}
		M - U_x = \bigcup_{y \notin U_x} U_y
	\end{equation*}
	which is open, therefore \( U_x \) is closed. Therefore it is both open and closed and
	since it is not empty (\( x \in U_x \)) it must be equal to \( M \). 

	Since \( M \) is path connected, there is a piecewise smooth path between any two points
	\( x \) and \( y \) in \( M \), call it \( \gamma \). Then
	\begin{equation*}
		d_g(x,y) \leq L(\gamma) < \infty. 
	\end{equation*}

	We now show that \( d_g \) is indeed a distance.  For
	any pair \( x \) and \( y \), if \( \gamma \colon [a,b] \to M \) is a path from \( x \)
	to \( y \), then the path \( -\gamma \colon [-b,-a] \to M \)  defined by \( -\gamma(t)
	\defeq \gamma(-t) \) satisfies \( -\gamma(-b) = y \) and \( -\gamma(-a) = x \) and of
	course has the same length as \( \gamma \) since it is just a reparametrisation of it.
	Then, by the definition of \( d_g \), for every \( \epsilon > 0 \), there exists a path
	\( \gamma \) from \( x \) to \( y \) such that \( L(\gamma) \leq d_g(x,y) + \epsilon \).
	But then 
	\begin{equation*}
		d_g(y,x) \leq L(-\gamma) = L(\gamma) \leq d_g(x,y) + \epsilon
	\end{equation*}
	which implies \( d_g(y,x) = d_g(x,y) \). 

	Let's show the triangle inequality, i.e. for any \( x,y,z \in M \)
	\begin{equation*}
		d_g(x,z) \leq d_g(x,y) + d_g(y,z).
	\end{equation*}
	For any \( \epsilon > 0 \) there exists paths \( \gamma_1 \) from \( x \) to \( y \) and
	\( \gamma_2 \) from \( y  \) to \( z \) such that
	\begin{equation*}
		L(\gamma_1) \leq d(x,y) + \epsilon
	\end{equation*}
	and
	\begin{equation*}
		L(\gamma_2) \leq d(y,z) + \epsilon. 
	\end{equation*}
	Consider the concatenation \( \gamma_1 + \gamma_2 \) of these two paths\footnote{For
	this to work we may need to potentially alter the parametrisations of the paths, but
	this will not alter their lengths}. Then this is a piecewise differentiable path from
	\( x \) to \( z \) and its length (as can be shown by a simple computation) is \(
	L(\gamma_1 + \gamma_2) = L(\gamma_1) + L(\gamma_2) \). Thus
	\begin{equation*}
		d(x,z) \leq L(\gamma_1 + \gamma_2) \leq d(x,y) + d(y,z) + 2\epsilon
	\end{equation*}
	which implies the triangle inequality since it works for arbitrarily small \( \epsilon
	\).
	Finally we prove that if \( x \) and \( y \) are not equal, then \( d(x,y) > 0 \). Take
	a chart \( (U, \phi) \) around \( x \). Then, since \( M \) is a Hausdorff space, there
	exists \( \epsilon > 0 \) such that \( y \notin
	\phi^{-1}\left(\bar{B}_\epsilon(\phi(x))\right) \). 

	% TODO: Falten detalls
	Let \( \gamma \colon [a,b] \to M \)	be a piecewise smooth path from \( x \) to \( y
	\). Then there exists \( t_0 \in (a,b] \) such that \( \gamma(t) \in
	\phi^{-1}\left(\bar{B}_\epsilon(\phi(x))\right) \) for all \( t \in [a,t_0] \) and \(
	\norm{\phi(\gamma(t_0)) - \phi(x)} = \epsilon \) (so the first time the path intersects
	the preimage of the ball). Then
	\begin{align*}
		L(\gamma) & \geq L(\gamma\rest{[a,t_0]} = \int_a^{t_0} \norm{\dot{\gamma}(t)}_g \ud t \\
							& = \int_a^{t_0} \sqrt{g_{\gamma(t)}(\dot{\gamma}(t), \dot{\gamma}(t))} \ud
							t \\
							& = \int_a^{t_0} \sqrt{g_{ij}(\phi(\gamma(t))) \dot{x}^i(t) \dot{x}^j(t) }
							\\
							& \geq \int_a^{t_0} \sqrt{\lambda} \norm{\dot{x}^i(t) e_i} \ud t \\
							& = \sqrt{\lambda} \int_a^{t_0} \norm{\partial_t (\phi \circ \gamma)(t)} \ud
							t \\
							& \geq \sqrt{\lambda} \norm{\int_{a}^{t_0} \partial_t (\phi \circ
							\gamma)(t)} \ud t \\
							& = \sqrt{\lambda} \norm{\phi(\gamma(t_0) - \phi(x)} = \sqrt{\lambda}
							\epsilon > 0. 
	\end{align*}
	
	So we have \( (M,d_g) \) is a metric space, and consequently a topological space. Lastly
	we show this topology agrees with the topology \( M \) carries by the fact it is a
	manifold. We just showd that for any \( \epsilon > 0 \) there is a \( \lambda > 0 \)
	such that
	\begin{equation*}
		\set{y \in M \mid d_g(x,y) < \sqrt{\lambda} \epsilon} \subseteq
		\phi^{-a}\left(B_\epsilon(\phi(x))\right)
	\end{equation*}
	The sets of the form on the left are a basis for the topology of \( M \) as a metric
	space, whereas those on the right are a basis for the topology of \( M \) as a manifold.
	We need the other inclusion. For any \( y \in \phi^{-1}\left(B_\epsilon(\phi(x))\right)
	\), define the path
	\begin{align*}
		\gamma_y \colon [0,1] & \longrightarrow M \\
		t & \longmapsto \phi^{-1}\left(t\phi(y) + (1-t)\phi(x)\right)
	\end{align*}
	so the preimage of the straight line segment between \( \phi(x) \) and \( \phi(y) \).
	Then
	\begin{align*}
		d_g(x,y) & \leq L(\gamma_y) = \int_0^1 \sqrt{g_{ij}(\phi(\gamma_y(t)))
		\dot{x}^i(t)\dot{x}^j(t)} \ud t \\
						 & \leq \int_0^1 \sqrt{\mu} \norm{\dot{\phi_\gamma(t)}} \ud t \\
						 & \sqrt{\mu} \norm{\phi(y) - \phi(x)} < \sqrt{\mu} \epsilon
	\end{align*}
	% La derivada és constant perquè la corba és la interpolació
\end{proof}

The computation we did in the previous proof can be used to show that the distance induced
by the Euclidean metric is exactly the same as the distance given by the Euclidean norm,
for for any path \( \gamma \colon [a,b] \to \R^n \) from \( x \) to \( y \) we have
\begin{align*}
	L(\gamma) = \int_a^b \norm{\dot{\gamma}(t)}\ud t \geq \norm{\int_a^b \dot{\gamma}(t) \ud
	t} = \norm{\gamma(b) - \gamma(b)} = \norm{x - y}
\end{align*}
which means \( d(x,y) \geq \norm{x - y} \). But the segment between \( x \) and \( y \) is
a path between \( x \) and \( y \) with length \( \norm{x- y} \) which shows we have
equality. In particular, we just showed that in euclidean space there always exists a path
of minimum length between two points, the segment (duh). 

\section{Volume in a Riemannian manifold}
\subsection{Orientability}
Let's recall the definition of an orientable manifold.
\begin{definition}[Orientable manifold]
	A smooth manifold \( M \) is called \emph{orientable} if it admits an orientable atlas,
	that is an atlas whose every transition map is orientation preserving (so that the sign
	of its jacobian is positive). If no such atlas exists then the manifold is called
	\emph{non orientable}.

	If \( M \) is orientable, choosing an orientable atlas determines an orientation for \(
	M\). If the union of two such atlases is orientable then we say they determine the same
	orientation. 
\end{definition}


\end{document}

