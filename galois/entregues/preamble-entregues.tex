% PREAMBLE FOR GALOIS THEORY NOTES

% ----------------------------------------------------------
% Packages
\usepackage[utf8]{inputenc}
\usepackage[T1]{fontenc}
\usepackage[catalan]{babel}
\usepackage{lmodern}
\usepackage{geometry}
\usepackage{hyperref}
\usepackage[dvipsnames]{xcolor}
\usepackage[bf,sf,small,pagestyles]{titlesec}
\usepackage{titling}
\usepackage[font={footnotesize, sf}, labelfont=bf]{caption} 
\usepackage{siunitx}
\usepackage{graphicx}
\usepackage{tikz-cd}
\usepackage{booktabs}
\usepackage{amsmath,amssymb}
\usepackage[sort]{cleveref}
\usepackage{amsthm,thmtools}
\usepackage[shortlabels]{enumitem}
\usepackage{todonotes}

% ----------------------------------------------------------
% Geometry setup
\geometry{
	a4paper,
	right = 3cm,
	left = 3cm,
	bottom = 3cm,
	top = 3cm
}
% Wider space between lines
\renewcommand{\baselinestretch}{1.3}

% ----------------------------------------------------------
% Hyperref setup
\hypersetup{
	colorlinks,
	linkcolor = {red!50!blue},
	linktoc = page
}
\numberwithin{table}{section}
\numberwithin{equation}{section}
\numberwithin{figure}{section}

\newcommand{\locallabel}[1]{\label{\currentprefix:#1}}
\newcommand{\localref}[1]{\ref{\currentprefix:#1}}
\newcommand{\localcref}[1]{\cref{\currentprefix:#1}}


% ----------------------------------------------------------
% Definition of theorem, example, etc environments
\newcommand{\qedtriangle}{\ensuremath{\triangle}}
\newcommand{\qedtriangledown}{\ensuremath{\bigtriangledown}}
\declaretheoremstyle[spaceabove=6pt, spacebelow=6pt, headfont=\bfseries,
notefont=\normalfont, notebraces={(}{)}, qed=\qedtriangle]{definition}

\declaretheorem[name=Teorema, refname={teorema,teoremes}, Refname={Teorema,Teoremes},
]{teorema}

\declaretheorem[name=Lema, refname={lema,lemes},
Refname={Lema,Lemes}, numberlike=teorema]{lema}

\declaretheorem[name=Coro\l.lari, refname={coro\l.lari,coro\l.laris},
Refname={Coro\l.lari,Coro\l.laris}, numberlike=teorema]{corollari}

\declaretheorem[name=Definició, style=definition, refname={definicio,definicions},
Refname={Definició,Definicions}]{definicio}

% ----------------------------------------------------------
% Definition of custom list style
\newlist{points}{enumerate}{1}
\setlist[points,1]{label=\textup{(}{\itshape \roman*}\textup{)}, wide}
