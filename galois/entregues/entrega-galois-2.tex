\documentclass[12pt]{article}

% PREAMBLE FOR GALOIS THEORY NOTES

% ----------------------------------------------------------
% Packages
\usepackage[utf8]{inputenc}
\usepackage[T1]{fontenc}
\usepackage[catalan]{babel}
\usepackage{lmodern}
\usepackage{geometry}
\usepackage{hyperref}
\usepackage[dvipsnames]{xcolor}
\usepackage[bf,sf,small,pagestyles]{titlesec}
\usepackage{titling}
\usepackage[font={footnotesize, sf}, labelfont=bf]{caption} 
\usepackage{siunitx}
\usepackage{graphicx}
\usepackage{tikz-cd}
\usetikzlibrary{babel}
\usepackage{booktabs}
\usepackage{amsmath,amssymb}
\usepackage{mathtools}
\usepackage[sort]{cleveref}
\usepackage{amsthm,thmtools}
\usepackage[shortlabels]{enumitem}
\usepackage{todonotes}

% ----------------------------------------------------------
% Geometry setup
\geometry{
	a4paper,
	right = 3cm,
	left = 3cm,
	bottom = 3cm,
	top = 3cm
}
% Wider space between lines
\renewcommand{\baselinestretch}{1.3}

% ----------------------------------------------------------
% Hyperref setup
\hypersetup{
	colorlinks,
	linkcolor = {red!50!blue},
	linktoc = page
}
\numberwithin{table}{section}
\numberwithin{equation}{section}
\numberwithin{figure}{section}

\newcommand{\locallabel}[1]{\label{\currentprefix:#1}}
\newcommand{\localref}[1]{\ref{\currentprefix:#1}}
\newcommand{\localcref}[1]{\cref{\currentprefix:#1}}


% ----------------------------------------------------------
% Definition of theorem, example, etc environments
\newcommand{\qedtriangle}{\ensuremath{\triangle}}
\newcommand{\qedtriangledown}{\ensuremath{\bigtriangledown}}
\declaretheoremstyle[spaceabove=6pt, spacebelow=6pt, headfont=\bfseries,
notefont=\normalfont, notebraces={(}{)}, qed=\qedtriangle]{definition}

\declaretheorem[name=Teorema, refname={teorema,teoremes}, Refname={Teorema,Teoremes},
]{teorema}

\declaretheorem[name=Lema, refname={lema,lemes},
Refname={Lema,Lemes}, numberlike=teorema]{lema}

\declaretheorem[name=Coro\l.lari, refname={coro\l.lari,coro\l.laris},
Refname={Coro\l.lari,Coro\l.laris}, numberlike=teorema]{corollari}

\declaretheorem[name=Definició, style=definition, refname={definicio,definicions},
Refname={Definició,Definicions}]{definicio}

% ----------------------------------------------------------
% Definition of custom list style
\newlist{points}{enumerate}{1}
\setlist[points,1]{label=\textup{(}{\itshape \roman*}\textup{)}, wide}

% CUSTOM COMMANDS FOR ALGEBRAIC TOPOLOGY
% ----------------------------------------------------------

% Restriction of a function
\newcommand{\rest}[1]{\raisebox{-.5ex}{$|$}_{#1}}

% Real numbers
\newcommand{\R}{\mathbb{R}}

% Rational numbers
\newcommand{\Q}{\mathbb{Q}}

% Complex numbers
\newcommand{\C}{\mathbb{C}}

% Natural numbers
\newcommand{\N}{\mathbb{N}}

% Integers
\newcommand{\Z}{\mathbb{Z}}

% Finite field
\newcommand{\F}{\mathbb{F}}

% Symmetric group
\renewcommand{\S}{\mathfrak{S}}

% Vector bold
\renewcommand{\vec}[1]{\mathbf{#1}}

% Script A, B, M, P
\newcommand{\A}{\mathcal{A}}
\newcommand{\B}{\mathcal{B}}
\newcommand{\M}{\mathcal{M}}
\renewcommand{\P}{\mathcal{P}}

% Identity
\newcommand{\id}{\mathrm{id}}

% Injection
\newcommand{\into}{\hookrightarrow}

% Surjection
\newcommand{\onto}{\twoheadrightarrow}

% Evaluation map
\newcommand{\ev}[1]{\mathrm{ev}_{#1}}

% Span
\newcommand{\gen}[1]{\left\langle #1 \right\rangle}

% Absolute value
\newcommand{\abs}[1]{\left\lvert #1 \right\rvert}

% Set
\newcommand{\set}[1]{ \{ #1 \} }

% mcd
\DeclareMathOperator{\mcd}{mcd}

% Characteristic of a field
\DeclareMathOperator{\ch}{ch}

% Minimal polymomial
\newcommand{\irr}[2]{m_{#1, #2}(x)}
\newcommand{\irre}[1]{m_{#1}(x)}

% Degree of an extension
\newcommand{\ext}[2]{[#1(#2) : #1]}
\newcommand{\exte}[2]{[#1 : #2]}

% Image
\DeclareMathOperator{\im}{im}

% Galois group
\DeclareMathOperator{\Gal}{Gal}

% Galois group over rationals
\DeclareMathOperator{\GalQ}{Gal_{\Q}}

% Norm
\newcommand{\norm}[1]{\lVert #1 \rVert}

\newcommand{\parbreak}{
	\begin{center}
		--- $\ast$ ---
	\end{center} 
}

% Defined as
\makeatletter
\newcommand*{\defeq}{\mathrel{\rlap{%
    \raisebox{0.3ex}{$\m@th\cdot$}}%
  \raisebox{-0.3ex}{$\m@th\cdot$}}%
	=
}
\makeatother


% Pagestyles
\newpagestyle{pagina}{
	\headrule
	\sethead*{\sffamily {\bfseries Teoria de Galois}}{}{Entrega 2}
	\footrule
	\setfoot*{}{}{\sffamily \thepage}
}
\renewpagestyle{plain}{
	\footrule
	\setfoot*{}{}{\sffamily \thepage}
}
\pagestyle{pagina}

\title{\sffamily {\bfseries Entrega 2}}
\author{\sffamily Arnau Mas}
\date{\sffamily 9 de gener de 2019}

\begin{document}
\maketitle

\section*{Problema 1}
Sigui \( F \) el cos \( \Q(\sqrt[4]{2}, i) \). Hem de veure que \( F \) és una extensió de
Galois de \( \Q \). \( F \) és una extensió separable perquè \( \Q \) té característica 0
i \( F \) és una extensió algebraica ---tant \( i \) com \( \sqrt[4]{2} \) són algebraics
sobre \( \Q \). Només hem de veure, doncs, que \( F \) és una extensió normal. I com que
es tracta d'una extensió finita, és equivalent a comprovar que és el cos de descomposició
d'algun polinomi. Considerem el polinomi \( x^4 - 2 \). Les seves arrels a \( \C \) són \(
\sqrt[4]{2} \), \( -\sqrt[4]{2} \), \( i \sqrt[4]{2} \) i \( -i \sqrt[4]{2} \), per tant
el seu cos de descomposició sobre \( \Q \) és
\begin{equation*}
	\Q(x^2 - 4) = \Q(\sqrt[4]{2}, -\sqrt[4]{2}, i\sqrt[4]{2}, -i\sqrt[4]{2}) =
	\Q(\sqrt[4]{2}, i\sqrt[4]{2}).
\end{equation*}
I de fet \( \Q(x^2 - 4) = F \). D'una banda, és car que \( F \) conté \( \Q(x^2 - 4) \), i
com que
\begin{equation*}
	i = \frac{i \sqrt[4]{2}}{\sqrt[4]{2}}
\end{equation*}
\( F \) està contingut dins de \( \Q(x^2 - 4) \). Per tant \( F \) és un cos de
descomposició sobre \( \Q \) i per tant n'és una extensió normal.

\parbreak

Com que \( F \) és una extensió de Galois tenim la igualtat
\begin{equation*}
	\abs{\GalQ{F}} = \exte{F}{\Q}. 
\end{equation*}
Per a calcular el grau \( \exte{F}{\Q} \) observem que \( F \) no pot ser igual a \(
\Q(\sqrt[4]{2}) \) donat que \( \Q(\sqrt[4]{2}) \) està contingut dins de \( \R \) però \(
F \) no. Això implica que \( \exte{F}{\Q(\sqrt[4]{2}} = 2 \). Efectivament,
l'irreductible de \( i \) a qualsevol extensió de \( \Q \) ha de dividir \( x^2 + 1 \).
I si fos de grau 1 sobre \( \Q(\sqrt[4]{2}) \) voldria dir que \( i \in \Q(\sqrt[4]{2})
\), que no és el cas. Pel lema de les torres, i fent servir que l'irreductible de \(
\sqrt[4]{2} \) sobre \( \Q \) és \( x^4 - 2 \) es té
\begin{equation*}
	\exte{F}{\Q} = \exte{F}{\Q(\sqrt[4]{2})} \ext{\Q}{\sqrt[4]{2}} = 4 \cdot 2 = 8. 
\end{equation*}
Això és suficient per a concloure que \( \GalQ{F} \) és isomorf al grup dihedral del
quadrat, \( D_{2 \times 4} \), ja que és l'únic subgrup d'ordre 8 de \( \S_4 \), i \(
\GalQ{F} \) és isomorf a un subgrup de \( \S_4 \).

\parbreak

Per a trobar el reticle de cossos intermitjos de l'extensió \( F/\Q \) fem servir la
correspondència de Galois. Abans, però, determinem l'estructura de \( \GalQ{F} \) amb una
mica més de detall. Com que \( F \) és el cos de descomposició de \( x^2 - 4 \), tenim que
\( \GalQ{F} = \GalQ{x^2 - 4} \), per tant els elements de \( \GalQ{F} \) permuten les
arrels de \( x^2 - 4 \), que són \( \sqrt[4]{2}, i\sqrt[4]{2}, -\sqrt[4]{2}, -i\sqrt[4]{2}
\). Denotem-les, en aquest ordre, per \( 1, 2, 3 \) i \( 4 \). D'altra banda, un element
de \( \GalQ{F} \) queda determinat per on envia \( \sqrt[4]{2} \) i \( i \). Les opcions
per a \( \sqrt[4]{2} \) són les quatre arrels del seu irreductible sobre \( \Q \), \( x^4
- 2 \). I les possibilitats per a \( i \) són les dues arrels de \( x^2 + 1 \), \( i \) i
\( -i \). Així, hi ha quatre elements de \( \GalQ{F} \) que fixen \( i \) i quatre que no.
Això correspon al fet de que de les 8 simetries d'un quadrat, és a dir, elements de \(
D_{2 \times 4} \), n'hi ha 4 que no involucren una reflexió i 4 que si. 

Els quatre automorfismes que fixen \( i \) són les quatre potències del morfisme que envia
\( \sqrt[4]{2} \) a \( i\sqrt[4]{2} \), és a dir, de la permutació \( (1 \, 2 \, 3 \, 4)
\). Els quatre morfismes restants són la composició d'un d'aquests pel morfisme
conjugació, és a dir, el que envia \( i \) a \( -i \), i que per tant queda representat
per a \( (2\,4) \). Els elements de \( D_{2 \times 4} \) són, doncs
\begin{table}[h]
	\center
	\begin{tabular}{cc}
		\toprule
		Rotacions & Reflexions \\
		\midrule
		id & (2 4) \\
		(1 2 3 4) & (1 2)(3 4) \\
		(1 3)(2 4) & (1 3) \\
		(1 4 3 2) & (1 4)(2 3) \\
		\bottomrule
	\end{tabular}
\end{table}

Anem a determinar el reticle de subgrups de \( D_{2 \times 4} \). El únics ordres
possibles per a subgrups no trivials de \( D_{2 \times 4} \) són 2 i 4. Els subgrups
d'ordre 2 estan generats per elements d'ordre 2. Els elements d'ordre 2 de \( D_{2 \times
4} \) són tots tret dels dos 4-cicles i la identitat, per tant hi ha 5 subgrups d'ordre 2.
Els dos 4-cicles generen el mateix subgrup d'ordre 4, que és isomorf a \( \Z/4\Z \). Dos
elements d'ordre 2 diferents poden generar un subgrup d'ordre 4 isomorf al grup de Klein \( V_4
\) ---equivalentment \( \Z/2\Z \times \Z/2\Z \)---. A priori tenim \( \binom{5}{2} = 10 \)
possibilitats. Tenim el subgrup
\begin{equation*}
	\gen{(1\,3)(2\,4)} = \gen{(1\,3), (1\,3)(2\,4)} = \gen{(2\,4), (1\,3)(2\,4)} = \set{\id,
		(1\,3), (2\,4), (1\,3)(2\,4)}.
\end{equation*}
Un altre subgrup és
\begin{align*}
	\gen{(1\,2)(3\,4), (1\,4)(2\,3)} & = \gen{(1\,2)(3\,4), (1\,3)(2\,4)} =
	\gen{(1\,4)(2\,3), (1\,3)(2\,4)} \\
																	 & = \set{\id, (1\,2)(3\,4), (1\,4)(2\,3),
																	 (1\,3)(2\,4)}.
\end{align*}
AMb tot això hem exhaurit 6 possibilitats. Les altres 4 generen tot el grup. Efectivament,
com que \( (1\,3)(1\,2)(3\,4) = (1\,2\,3\,4) \), el subgrup \( \gen{(1\,3), (1\,2)(3\,4)}
\) té ordre superior a 4, i per tant és el total. Conjugant per les potències de \(
(1\,2\,3\,4) \) arribem a la conclusió de que passa el mateix per a les altres tres
possibilitats. El reticle de subgrups és
\begin{equation*}
	\begin{tikzcd}[column sep = 0.8]
		& & D_{2 \times 4} \arrow[dl, no head] \arrow[d, no head] \arrow[dr, no head] \\
		& \makebox[\widthof{$\gen{(1\,2)(3\,4), (1\,4)(2\,3)}$}][c]{$\gen{(1\,3),(2\,4)}$} & \gen{(1\,2\,3\,4)} & \gen{(1\,2)(3\,4), (1\,4)(2\,3)} \\
		\gen{(1\,3)} \arrow[ur, no head] & \gen{(2\,4)} \arrow[u, no head] &
		\gen{(1\,3)(2\,4)} \arrow[ul, no head] \arrow[u, no head] \arrow[ur, no head] &
		\gen{(1\,2)(3\,4)} \arrow[u, no head] & \gen{(1\,4)(2\,3)} \arrow[ul, no head] \\
																					& & \gen{\id} \arrow[ull, no head] \arrow[ul, no
																					head] \arrow[u, no head] \arrow[ur, no head]
																					\arrow[urr, no head] 
	\end{tikzcd}
\end{equation*}

Un cop coneixem l'estructura de \( \GalQ{F} \cong D_{2\times 4} \) podem determinar
el reticles de cossos intermitjos de \( F/\Q \), fent servir la correspondència de Galois,
\( H \mapsto F^H \) per a \( H \leq \GalQ{F} \).  Com que l'extensió és de Galois, la
correspondència és bijectiva. En particular es dedueix que hi ha tres cossos intermitjos
de grau 2 sobre \( \Q \) i cinc cossos intermitjos de grau 4 sobre \( \Q \), fent servir
que \( \exte{F^H}{\Q} = \abs{\GalQ{F}}/\abs{H} \).

% Com que \( F = \Q(\sqrt[4]{2})(i) \), tot element de \( F \) es pot escriure com \(
% \alpha + \beta i \) on \( \alpha, \beta \in \Q(\sqrt[4]{2}) \). I \( \Q(\sqrt[4]{2}) \) és
% una extensió de grau 4 sobre \( \Q \) per tant podem escriure'n un element qualsevol com
% \begin{equation*}
% 	\alpha = \alpha_1 + \alpha_2 \sqrt[4]{2} + \alpha_3 \sqrt{2} + \alpha_4 2^{3/4}. 
% \end{equation*}
% Amb \( \alpha_i \in \Q \). Per a poder determinar els cossos fixos per subgrups de
% \( \GalQ{F} \) ens cal saber com actuen els elements de \( \GalQ{F} \) sobre els sis
% elements no racionals de la \( \Q \)-base de \( F \) que acabem de donar.

Comencem pel subgrup \( H_1 = \gen{(1\,3),(2\,4)} \). Com que \( H_1 \) té ordre 4 \(
F^{H_1} \) ha de tenir ordre 2 sobre \( \Q \). Recordem que \( (2\,4) \) representa
el morfisme conjugació, pel que tots els elements de \( F^{H_1} \) han de ser reals. A
més
\begin{equation*}
	(1\,3)(\sqrt{2}) = \left((1\,3)(\sqrt[4]{2})\right)^2 = (-\sqrt[4]{2})^2 = \sqrt{2}
\end{equation*}
per tant \( \Q(\sqrt{2}) \subseteq F^H_1 \). Però de fet es té igualtat perquè \(
\Q(\sqrt{2}) \) té grau 2 sobre \( \Q \).

Considerem el subgrup \( H_2 = \gen{(1\,2\,3\,4)} \). Com abans, té ordre 2 per tant el
corresponent cos \( F^{H_2} \) ha de tenir ordre 2 sobre \( \Q \). Recordem que \(
(1\,2\,3\,4) \) permuta les arrels de \( x^4 - 2 \) cíclicament però fixa \( i \), per
tant \( \Q(i) \subseteq F^{H_2} \), però de fet es té igualtat perquè \( \Q(i) \) té grau
2 sobre \( \Q \). 

L'últim dels subgrups d'ordre 4 és \( H_3 = \gen{(1\,2)(3\,4), (1\,4)(2\,3)} \). Tenim 
\begin{gather*}
	(1\,2)(3\,4)(i \sqrt{2}) = \left((1\,2)(3\,4)(\sqrt[4]{2})\right)
	\left((1\,2)(3\,4)(i\sqrt[4]{2})\right) = (i \sqrt[4]{2})(\sqrt[4]{2}) = i \sqrt{2} \\
	(1\,4)(2\,3)(i \sqrt{2}) = \left((1\,4)(2\,3)(\sqrt[4]{2})\right)
	\left((1\,4)(2\,3)(i\sqrt[4]{2})\right) = (-i \sqrt[4]{2})(-\sqrt[4]{2}) = i \sqrt{2}
\end{gather*}
per tant \( \Q(i\sqrt{2}) = F^{H_3} \) perquè \( \Q(i\sqrt{2}) \) té grau 2 sobre \( \Q
\).

Fem el mateix per als cincs subgrups d'ordre 2, que donen lloc a extensions de grau 4.
Ràpidament, com que \( (1\,3)(i \sqrt[4]{2}) = i \sqrt[4]{2} \) i \( (2\,4)(\sqrt[4]{2}) =
\sqrt[4]{2} \) concloem
\begin{equation*}
	F^{\gen{(1\,3)}} = \Q(i\sqrt[4]{2}) \text{ i } F^{\gen{(2\,4)}} = \Q(\sqrt[4]{2})
\end{equation*}
perquè tant \( \sqrt[4]{2} \) com \( i\sqrt[4]{2} \) tenen grau sobre \( \Q \), perquè són
arrels de \( x^4 - 2 \). 

Considerem ara el subgrup \( \gen{(1\,3)(2\,4)} \). Ja hem vist que \( \sqrt{2} \) és fix
tant per \( (1\,3) \) com per \( (2\,4) \), per tant també ho és per la seva composició.
També ho és \( i \):
\begin{equation*}
	(1\,3)(2\,4)(i) = \frac{(1\,3)(2\,4)(i\sqrt[4]{2})}{(1\,3)(2\,4)(\sqrt[4]{2})} =
	\frac{-i\sqrt[4]{2}}{-\sqrt[4]{2}} = i.
\end{equation*}
Aleshores, com que \( \Q(i,\sqrt{2}) \) té grau 2 sobre \( \Q \), és \(
F^{\gen{(1\,3)(2\,4)}} \). Mirem de trobar un element primitiu d'aquesta extensió.
Considerem \( \gamma = i + \sqrt{2} \). Aleshores \( \gamma^3 = 7i + \sqrt{2} \), per tant
\begin{equation*}
	i = \frac{\gamma^3 - \gamma}{6}
\end{equation*}
és a dir, \( i \in \Q(\gamma) \), i per tant \( \sqrt{2} = \gamma - i \in \Q(\gamma)
\). Això ens dona \( \Q(\gamma) = \Q(i, \sqrt{2}) \). 

Queden dos extensions de grau 4. Per a la que correspon a \( \gen{(1\,2)(3\,4)} \), com
que \( (1\,2)(3\,4) \) intercanvia \( \sqrt[4]{2} \) amb \( i\sqrt[4]{2} \), \(
\beta = \sqrt[4]{2} + i \sqrt[4]{2} \) és fix per la seva acció, per tant \( \Q(\beta)
\subseteq F^{\gen{(1\,2)(3\,4)}} \). Tenim que \( \beta^2 = 2(i\sqrt{2}) \), per tant \(
\Q(i\sqrt{2}) \subseteq \Q(\beta) \). Però la inclusió és estricta perquè \( \beta \) no
és fix per \( (1\,4)(2\,3) \), a diferència de tots els elements de \( \Q(i\sqrt{2}) \).
Per tant \( \Q(\beta) \) té grau almenys 4 sobre \( \Q \). I de fet exactament 4 perquè \(
\beta^4 = -8 \). 

Per un argument molt similar, \( \Q(\alpha) = F^{\gen{(1\,4)(2\,3)}} \) amb \( \alpha =
\sqrt[4]{2} - i\sqrt[4]{2} \). Així doncs, el reticle de cossos intermitjos queda
\begin{equation*}
	\begin{tikzcd}
		& & \Q(i, \sqrt[4]{2}) \arrow[dll, no head] \arrow[dl, no head] \arrow[d, no head]
		\arrow[dr, no head] \arrow[drr, no head] & & \\
		\Q(i \sqrt[4]{2}) & \Q(\sqrt[4]{2}) & \Q(\sqrt{2} + i)  
											&	\makebox[\widthof{$\Q(\sqrt[4]{2})$}][c]{$\Q(\sqrt[4]{2} +
											i\sqrt[4]{2})$}
											&	\Q(\sqrt[4]{2} - i\sqrt[4]{2}) \\
		& \Q(\sqrt{2}) \arrow[ul, no head] \arrow[u, no head] \arrow[ur, no head] & \Q(i)
		\arrow[u, no head] & \Q(i\sqrt{2}) \arrow[ul, no head] \arrow[u, no head] \arrow[ur, no head]\\
		& & \Q \arrow[ul, no head] \arrow[u, no head] \arrow[ur, no head] & &
	\end{tikzcd}
\end{equation*}

% TODO: trobar element primitiu de F

\section*{Problema 2}
El conjunt d'automorfismes d'un grup \( G \), \( \Aut(G) \) és un grup amb la composició.
Tenim que la composició d'automorfismes és automorfisme: en efecte, la composició de
morfismes és morfisme, i si \( \phi, \psi \in \Aut(G) \) aleshores
\begin{equation*}
	(\phi \circ \psi)^{-1} = \psi^{-1} \circ \phi^{-1} 
\end{equation*}
per tant \( \phi \circ \psi \) és un automorfisme i no només un morfisme. A més la
composició d'aplicacions és sempre associativa. El morfisme identitat fa el paper
d'element neutre. També hi ha inversos perquè la inversa d'un automorfisme també és un
morfisme. Així doncs \( (\Aut(G), \circ) \) és un grup. 

\parbreak

Considerem una acció d'un grup \( H \) sobre un grup \( G \), \( \phi \colon H \to
\Aut(G) \). Es defineix el producte semidirecte \( G \rtimes_\phi H \) o simplement \( G
\rtimes H \) com \( G \times H \) amb l'operació
\begin{equation*}
	(g_1, h_1)(g_2, h_2) = (g_1 \phi(h_1)(g_2), h_1 h_2).
\end{equation*}
Aquesta operació dóna a \( G \rtimes H \) estructura de grup. Efectivament, és clar que
defineix una operació a \( G\rtimes H \), és a dir, \( (g_1, h_1)(g_2, h_2) \in G\rtimes H
\). L'element neutre és \( (e_G, e_H) \):
\begin{gather*}
	(g,h)(e_G, e_H) = (g \phi(h)(e_G), he_H) = (g e_G, h) = (g,h) \\
	(e_G, e_H)(g,h) = (e_G \phi(e_H)(g), e_H h) = (\id_G(g), h) = (g,h).
\end{gather*}
L'invers de \( (g,h) \) és \( (\phi(h^{-1})(g^{-1}), h^{-1}) \):
\begin{gather*}
	(g,h)(\phi(h^{-1})(g^{-1}), h^{-1}) = \left(g \left(\phi(h) \circ \phi(h^{-1})\right)(g^{-1}),
	hh^{-1}\right) = (gg^{-1}, e_H) = (e_G, e_H) \\
	\begin{aligned}
		(\phi(h^{-1})(g^{-1}), h^{-1})(g,h) & = (\phi(h^{-1})(g^{-1}) \phi(h^{-1})(g),
		h^{-1}h) \\
																				& =	\left(\left(\phi(h^{-1})(g)\right)^{-1}
																				\phi(h^{-1})(g), e_H\right) = (e_G, e_H).
	\end{aligned}
\end{gather*}
Per últim cal comprovar l'associativitat:
\begin{align*}
	(g_1,h_1)\big((g_2, h_2)(g_3, h_3)\big) & = (g_1,h_1) \big(g_2 \phi(h_2)(g_3), h_2
	h_3)\big) \\
																					& = \Big(g_1 \phi(h_1)\big(g_2
																					\phi(h_2)(g_3)\big), h_1(h_2h_3)\Big) \\
																					& = \Big(g_1 \phi(h_1)(g_2)
																					\big(\phi(h_1h_2)(g_3)\big), (h_1h_2)h_3\Big)
																				\\
																					& = \big(g_1\phi(h_1)(g_2), h_1h_2\big)(g_3,h_3) \\
																					& =
																					\big((g_1,h_1)(g_2,h_2)\big)(g_3,h_3).
\end{align*}

Observem que amb l'acció trivial \( e \colon H \to \Aut(G) \) on \( e(h) = \id_G \) per
tot \( g \in G \) recuperem el producte directe de grups estàndard.

\parbreak

Considerem un producte directe \( G = G_1 \rtimes G_2 \). Considerem el subconjunt
\begin{equation*}
	K = \set{(g,h) \in G \mid g = e_{G_1}}.
\end{equation*}
\( K \) és un subgrup. Si \( (e_{G_1},h_1) \) i \( (e_{G_1},h_2) \) són a \( K \)
aleshores
\begin{align*}
	(e_{G_1}, h_1)(e_{G_1}, h_2)^{-1} & = (e_{G_1}, h_1)(\phi(h_2^{-1})(e_{G_1}), h_2^{-1})
	\\
																		& = (e_{G_1},h_1)(e_{G_1}, h_2^{-1}) = (e_{G_1}
																		\phi(h_1)(e_{G_1}), h_1h_2^{-1}) \\
																		& = (e_{G_1}, h_1h_2^{-1}) \in K.
\end{align*}

Similarment tenim el subgrup \( H = \set{(g,h) \in G \mid h = e_{G_2}} \). Si \(
(g_1,e_{G_2}) \) i \( (g_2,e_{G_2}) \) són a \( H \) aleshores 
\begin{align*}
	(g_1,e_{G_2})(g_2, e_{G_2})^{-1} & = (g_1,e_{G_2})(\phi(e_{G_2})(g_2^{-1}), e_{G_2}) \\
																	 & = (g_1,e_{G_2})(g_2^{-1}, e_{G_2}) = (g_1
																	 \phi(e_{G_2})(g_2^{-1}), e_{G_2}) \\
																	 & = (g_1 g_2^{-1}, e_{G_2}) \in H.
\end{align*}
A més \( H \) és normal a \( G \): per tot \( (x,y) \in G \) i \( g \in G_1 \) aleshores 
\begin{align*}
	(x,y)(g,e_{G_2})(x,y)^{-1} &= (x \phi(y)(g), y)\left(\phi(y^{-1})(x^{-1}), y^{-1}\right) \\
														 & = \left(x \phi(y)(g) \left(\phi(y) \circ
														 \phi(y^{-1})(x^{-1})\right), yy^{-1}\right) \\
														 & = (x \phi(y)(g) x^{-1}, e_{G_2}) \in H.
\end{align*}

És clar que \( H \cap K = \gen{(e_1, e_2)} \)	i que \( H \cong G_1 \) i \( K \cong G_2 \)
mitjançant les projeccions sobre el primer factor i el segon, respectivament. I per tot \(
(g,h) \in G \) es té
\begin{equation*}
	(g,e_{G_2})(e_{G_1}, h) = (g \phi(e_{G_2})(e_{G_1}), e_{G_2}h) = (g,h),
\end{equation*}
és a dir, \( HK = G \). 

A continuació veiem el recíproc del que acabem de provar: si un grup \( G \) té dos
subgrups \( H \) i \( K \) tals que \( HK = G \), \( H \cap K = \gen{e} \) i \( H \) és
normal a \( G \) aleshores \( G \) és isomorf a \( H \rtimes K \). 

En general, la conjugació per un element fix \( y \in G \) és un automorfisme.
Efectivament, si la denotem per \( \phi_y \) es té
\begin{equation*}
	\phi_y(x_1 x_2) = yx_1 x_2y^{-1} = yx_1y^{-1} yx_2y^{-1} =
	\phi_y(x_1)\phi_y(x_2),
\end{equation*}
i la conjugació per \( y^{-1} \) n'és l'invers. Com que \( H \) és normal a \( G \),
\( \phi_y(h) \in H \) si \( h \in H \), per qualsevol \( y \in G \). Així, \( \phi_y \) es
pot restringir a un automorfisme d'\( H \). Això ens permet definir una acció de \( K \)
sobre \( H \) mitjançant \( k \mapsto \phi_k \). Això funciona perquè 
\begin{equation*}
	(k_1k_2)h(k_1k_2)^{-1} = k_1 (k_2hk_2^{-1}) k_1^{-1},
\end{equation*}
és a dir, \( \phi_{k_1 k_2} = \phi_{k_1} \circ \phi_{k_2} \). Tenim, doncs, el producte
semidirecte \( H \rtimes K \) mitjançant aquesta acció. 

Definim \( \Phi \colon H \rtimes K \to G \) com \( \Phi(h,k) = hk \). A continuació veiem
que \( \Phi \) és un isomorfisme. Tenim
\begin{align*}
	\Phi\big((h_1,k_1)(h_2,k_2)\big) & = \Phi(h_1 \phi_{k_1}(h_2), k_1 k_2) \\
																	 & = \Phi(h_1k_1h_2 k_1^{-1}, k_1k_2) \\
																	 & = h_1k_1h_2k_2 = \Phi(h_1,k_1) \Phi(h_2,k_2).
\end{align*}
Que \( \Phi \) és exhaustiva és perquè \( HK = G \). I que és injectiva és conseqüència de
que \( H \cap K = \gen{e} \). En efecte, si \( (h,k) \in \ker \Phi \) aleshores \( hk = e
\), per tant \( h = k^{-1} \in H\cap K \), és a dir \( h = k = e \). Així \( \Phi \) té
nucli trivial i per tant és injectiva. Tenim que \( \Phi \) és un morfisme bijectiu, és a
dir un isomorfisme.

\section*{Problema 3}
Considerem el polinomi \( p(x) = x^5 - 2 \in \Q[x] \). Hem de veure que el seu cos de
descomposició sobre \( \Q \) és \( F = \Q(\sqrt[5]{2}, \zeta) \) on \( \zeta =
e^{\frac{2\pi i}{5}} \). És clar que \( \zeta^k \sqrt[5]{2} \) amb \( 0 \leq k \leq 4 \)
són cinc arrels de \( p(x) \) diferents a \( \C \). Pel Teorema Fonamental de l'Àlgebra,
de fet són totes les arrels. Per tant el cos de descomposició de \( p(x) \) és
\begin{equation*}
	\Q(\sqrt[5]{2}, \zeta \sqrt[5]{2}, \zeta^2\sqrt[5]{2}, \zeta^3 \sqrt[5]{2}, \zeta^4
	\sqrt[5]{2}).
\end{equation*}
Certament aquest cos està contingut a \( F \), i també el conté perquè \( \zeta =
\frac{\zeta \sqrt[5]{2}}{\sqrt[5]{2}} \). Per tant són iguals. 

L'extensió \( \Q(\sqrt[5]{2}) \) és de grau 5 sobre \( \Q \) perquè \( \sqrt[5]{2} \) és
arrel de \( x^5 - 2 \), que és irreductible per Eisenstein. D'altra banda, \( \Q(\zeta) \)
és una extensió de grau 4 sobre \( \Q \) perquè \( \zeta \) és arrel de \( x^4 + x^3 + x^2
+ x + 1 \), que és irreductible perquè és un polinomi ciclotòmic. Tenim el següent
diagrama d'inclusions
\begin{equation*}
	\begin{tikzcd}[column sep = 0.5]
		& \Q(\sqrt[5]{2},\zeta) \arrow[dl, no head], \arrow[dr, no head] & \\
		\Q(\sqrt[5]{2}) & & \Q(\zeta) \\
									 & \Q \arrow[ur, "4", no head, swap] \arrow[ul, "5", no head] & 
	\end{tikzcd}
\end{equation*}
\( \zeta \) continua sent arrel de \( x^4 + x^3 + x^2 + x + 1 \) a \( F \), així
com \( \sqrt[2]{4} \) ho és de \( x^5 - 2 \),  el que vol dir que \(
\exte{F}{\Q(\sqrt[5]{2})} \) no pot ser més gran que 4 i \( \exte{F}{\Q(\zeta)} \) no pot
ser més gran que 5. Això, combinat amb el lema de les torres i amb el fet que 4 i 5 són
coprimers ens permet concloure que \( \exte{F}{\Q} = 4 \times 5 = 20 \).

\parbreak

Sigui \( G = \GalQ(F) \). Per a determinar els elements de \( G \) considerem el següent
diagrama
\begin{equation*}
	\begin{tikzcd}
		\Q(\zeta)(\sqrt[5]{2}) \arrow[r, "\sigma"] & \Q(\zeta)(\sqrt[5]{2}) \\
		\Q(\zeta) \arrow[u, tail] \arrow[ur, "\iota"] \\
		\Q \arrow[u, tail] \arrow[uur, tail]
	\end{tikzcd}
\end{equation*}
El morfisme \( \iota \) és la inclusió que ve determinada per \( \iota(\zeta) = \zeta \),
que existeix perquè \( \zeta \) és arrel de \( x^4 + x^3 + x^2 + x + 1 \) a \( F \). Pel
lema d'extensió de morfismes, \( \iota \) es pot extendre a \( \sigma \) tal que \(
\sigma(\sqrt[5]{2}) = \zeta \sqrt[5]{2} \) perquè \( \zeta \sqrt[5]{2} \) és una arrel de
\( x^5 - 2 \) a \( F \). Aleshores \( \sigma \in G \). 

Similarment considerem 
\begin{equation*}
	\begin{tikzcd}
		\Q(\zeta)(\sqrt[5]{2}) \arrow[r, "\tau"] & \Q(\zeta)(\sqrt[5]{2}) \\
		\Q(\zeta) \arrow[u, tail] \arrow[ur, "\bar{\tau}"] \\
		\Q \arrow[u, tail] \arrow[uur, tail]
	\end{tikzcd}
\end{equation*}
on \( \bar{\tau}(\zeta) = \zeta^2 \), que és un morfisme perquè \( \zeta^2 \) és arrel de
\( x^4 + x^3 + x^3 + x + 1 \) a \( F \), i \( \tau(\sqrt[5]{2}) = \sqrt[5]{2} \). Pel lema
d'extensió de morfismes, \( \tau \) és un element de \( G \). 

\parbreak

Com que \( F \) és un cos de descomposició, pel matiex argument que al problema 1, és una
extensió e Galois de \( \Q \) i per tant \( \abs{G} = \exte{F}{\Q} = 20 \).

\( \sigma \) té ordre 5 perquè \( \sigma^5(\sqrt[5]{2}) = \zeta^5 \sqrt[5]{2} =
\sqrt[5]{2} \) i \( \tau \) té ordre 4:
\begin{equation*}
	\zeta \xmapsto{\tau} \zeta^2 \xmapsto{\tau} \zeta^4 \xmapsto{\tau} \zeta^8 = \zeta^3
	\xmapsto{\tau} \zeta^6 = \zeta.
\end{equation*}
Aleshores \( \gen{\sigma} \) és un subgrup de \( G \) d'ordre 5 i \( \gen{\tau} \) és un
subgrup de \( G \) d'ordre 4. Per qüestions de grau, \( \gen{\sigma} \cap \gen{\tau} =
\gen{\id} \), ja que l'ordre d'un element de la intersecció ha de dividir 5 i 4, per tant
només pot ser 1. En particular, si \( \sigma^n \tau^m = \sigma^k \tau^l \) aleshores \(
\sigma^n \sigma^{-k} = \tau^l \tau^{-m} = \id \), pel que \( \sigma^n = \sigma^k \) i
\( \tau^m = \tau^l \). Això vol dir, en particular, que \( \gen{\sigma}\gen{\tau} \) conté
almenys 20 elements diferents, els 20 possibles productes d'elements de \( \gen{\sigma} \)
amb elements de \( \gen{\tau} \). Però \( \abs{G} = 20 \), el que vol dir \(
\gen{\sigma}\gen{\tau} = G \). 

Observem que \( \gen{\sigma} \) és un 5-subgrup de Sylow de \( G \). El nombre de
5-subgrups de Sylow de \( G \), pel tercer teorema de Sylow, divideix 4 i és congruent a
1 mòdul 5, per tant només pot ser 1. Això vol dir que \( \gen{\sigma} \) és l'únic
5-subgrup de Sylow de G, i per tant és normal, pel segon teorema de Sylow. Estem en totes
les hipòtesis per a concloure, fent servir el problema anterior, que 
\begin{equation*}
	G \cong \gen{\sigma} \rtimes \gen{\tau}
\end{equation*}
i com que \( \gen{\sigma} \) i \( \gen{\tau} \) són grups cíclics de 5 i 4 elements
respectivament
\begin{equation*}
	G \cong \Z/5\Z \rtimes \Z/4\Z.
\end{equation*}



\end{document}

