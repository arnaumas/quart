\documentclass[12pt]{article}

% Packages
\usepackage[utf8]{inputenc}
\usepackage[T1]{fontenc}
\usepackage[catalan]{babel}
\usepackage{lmodern}
\usepackage{geometry}
\usepackage{hyperref}
\usepackage[dvipsnames]{xcolor}
\usepackage[bf,sf,small,pagestyles]{titlesec}
\usepackage{titling}
\usepackage[font={footnotesize, sf}, labelfont=bf]{caption} 
\usepackage{siunitx}
\usepackage{graphicx}
\usepackage{booktabs}
\usepackage{amsmath,amssymb}
\usepackage[catalan,sort]{cleveref}
\usepackage{enumitem}

% Geometry setup
\geometry{
	a4paper,
	right = 2.5cm,
	left = 2.5cm,
	bottom = 3cm,
	top = 3cm
}
% Wider space between lines
\renewcommand{\baselinestretch}{1.3}

\hypersetup{
	colorlinks,
	linkcolor = {red!50!blue},
	linktoc = page
}

% Cref names in catalan
\crefname{figure}{figura}{figures}
\crefname{table}{taula}{taules}
\numberwithin{table}{section}
\numberwithin{figure}{section}

\graphicspath{{./figs/}}


% COMMANDS FOR REAL ANALYSIS
% ----------------------------------------------------------

% Restriction of a function
\newcommand{\rest}[1]{\raisebox{-.5ex}{$|$}_{#1}}

% Real numbers
\newcommand{\R}{\mathbb{R}}

% Rational numbers
\newcommand{\Q}{\mathbb{Q}}

% Complex numbers
\newcommand{\C}{\mathbb{C}}

% Natural numbers
\newcommand{\N}{\mathbb{N}}

% Integers
\newcommand{\Z}{\mathbb{Z}}

% Script A, B, M, P, L
\newcommand{\A}{\mathcal{A}}
\newcommand{\B}{\mathcal{B}}
\newcommand{\M}{\mathcal{M}}
\renewcommand{\L}{\mathcal{L}}
\renewcommand{\P}{\mathcal{P}}

\newcommand{\class}[2]{C^{#1}(#2)}
\newcommand{\cont}[1]{\class{0}{#1}}

% Identity
\newcommand{\id}{\mathrm{id}}

% Almost everywhere
\renewcommand{\ae}{\text{ a.e.}}

% Almost everywhere on
\newcommand{\aeon}[1]{\text{ a.e. on } #1}

% Absolute value
\newcommand{\abs}[1]{\left\lvert #1 \right\rvert}

% Norm
\newcommand{\norm}[1]{\left\lVert #1 \right\rVert}
\newcommand{\normu}[1]{\norm{#1}_1}
\newcommand{\normd}[1]{\norm{#1}_2}
\newcommand{\normp}[1]{\norm{#1}_p}
\newcommand{\normi}[1]{\norm{#1}_\infty}

% Inner product
\newcommand{\inn}[2]{\left\langle #1\, , #2 \right\rangle}

% Exterior measure
\newcommand{\ext}[1]{m^* \! \left( #1 \right)}  

% Integral differential
\renewcommand{\d}{\, \mathrm{d}}

% Defined as
\makeatletter
\newcommand*{\defeq}{\mathrel{\rlap{%
    \raisebox{0.3ex}{$\m@th\cdot$}}%
  \raisebox{-0.3ex}{$\m@th\cdot$}}%
	=
}
\makeatother

% Set
\newcommand{\set}[1]{ \{ #1 \} }

% Break between paragraphs
\newcommand{\parbreak}{
	\begin{center}
		--- $\ast$ ---
	\end{center} 
}


% Pagestyles
\newpagestyle{pagina}{
	\headrule
	\sethead*{\sffamily {\bfseries Anàlisi real.} Espais normats i de Banach}{}{\theauthor}
	\footrule
	\setfoot*{}{}{\sffamily \thepage}
}
\renewpagestyle{plain}{
	\footrule
	\setfoot*{}{}{\sffamily \thepage}
}
\pagestyle{pagina}

\title{\sffamily {\bfseries Entrega 3:} Espais de Hilbert}
\author{\sffamily Arnau Mas}
\date{\sffamily 10 de gener de 2019}


\begin{document}
\maketitle
\section*{Problema 1}
Sigui \( E \) un espai de Banach real tal que la seva norma satisfà la identitat del
para\l.lelogram, és a dir, per tot \( x,y \in E \)
\begin{equation*}
	2\norm{x}^2 + 2\norm{y}^2 = \norm{x+y}^2 + \norm{x-y}^2.
\end{equation*}

Definim l'aplicació
\begin{align*}
	\inn{\cdot}{\cdot} \colon E \times E & \longrightarrow \R \\
	(x,y) & \longmapsto \frac{\norm{x+y}^2 - \norm{x-y}^2}{4}.
\end{align*}
Veurem que és un producte escalar a \( E \) que indueix la norma \(
\norm{\cdot} \), i per tant que \( E \) té estructura d'Espai de Hilbert.

En primer lloc, per tot \( x \in E \),
\begin{equation*}
	\inn{x}{x} = \frac{\norm{x+x}^2 - \norm{x-x}^2}{4} = \frac{\norm{2x}^2 + \norm{0}^2}{4}
	= \frac{4\norm{x}^2}{4} = \norm{x}^2.
\end{equation*}
Aleshores, si demostrem \( \inn{\cdot}{\cdot} \) és efetivament un producte escalar, la
norma que indueix és \( \norm{\cdot} \). Com a conseqüència, \( \inn{x}{x} = 0 \)
equival a \( \norm{x}^2 = 0 \), i això és equivalent a \( x = 0 \), per propietats de la
norma. L'aplicació \( \inn{\cdot}{\cdot} \) és, doncs, definida positiva.

També es veu ràpidament que és simètrica:
\begin{equation*}
	\inn{x}{y} = \frac{\norm{x+y}^2 - \norm{x-y}^2}{4} = \frac{\norm{y+x}^2 - \norm{y-x}^2}{4} = \inn{y}{x}.
\end{equation*}

Queda per comprovar que \( \inn{\cdot}{\cdot} \) és bilineal. Com que hem provat que és
simètrica només cal veure que és lineal en un dels arguments. Per tot \( x,y, z \in E \)
es té
\begin{align*}
	4\inn{x+y}{z} - 4\inn{x}{z} - 4\inn{y}{z}	& = \norm{x+y+z}^2 - \norm{x+y-z}^2 \\
																						& \quad -	\norm{x+z}^2 + \norm{x-z}^2 -
																						\norm{y+z}^2 + \norm{y-z}^2.
\end{align*}
Apliquem la identitat del para\l.lelogram amb el primer terme i el quart:
\begin{equation*}
	\norm{x+y+z}^2 + \norm{x-z}^2 = \frac{1}{2}\left(\norm{2x + y}^2 + \norm{y +
	2z}^2\right),
\end{equation*}
i amb el segon terme i el tercer:
\begin{equation*}
	-\norm{x+y-z}^2 - \norm{x+z}^2 = -\frac{1}{2}\left(\norm{2x+y}^2 + \norm{y-2z}^2\right).
\end{equation*}
Aleshores queda
\begin{equation*}
	4\inn{x+y}{z} - 4\inn{x}{z} - 4\inn{y}{z}	= \frac{1}{2}\norm{y + 2z}^2 -
	\frac{1}{2}\norm{y-2z} - \norm{y+z}^2 + \norm{y-z}^2.
\end{equation*}
Una altra manera d'escriure la identitat del para\l.lelogram és
\begin{equation*}
	\frac{1}{2}\norm{u+v}^2 = \norm{u}^2 + \norm{v}^2 - \frac{1}{2}\norm{u-v}^2,
\end{equation*}
que ens dóna
\begin{equation*}
	\frac{1}{2}\norm{y+2z}^2 = \norm{y+z}^2 + \norm{z}^2 - \frac{1}{2}\norm{y}^2,
\end{equation*}
i similarment
\begin{equation*}
	\frac{1}{2}\norm{y-2z}^2 = \norm{y-z}^2 +\norm{z}^2 - \frac{1}{2}\norm{y}^2.
\end{equation*}
Per tant
\begin{equation*}
	\frac{1}{2}\norm{y+2z}^2 - \frac{1}{2}\norm{y-2z}^2 = \norm{y+z}^2 - \norm{y-z}^2
\end{equation*}
i \( 4\inn{x+y}{z} - 4\inn{x}{z} - 4\inn{y}{z} = 0 \), és a dir
\begin{equation*}
	\inn{x+y}{z} = \inn{x}{z} + \inn{y}{z}.
\end{equation*}
Combinant-ho amb la simetria, fins aquí tenim que \( \inn{\cdot}{\cdot} \) és additiva en
els dos arguments.

L'additivitat implica immediatament que \( \inn{nx}{y} = n\inn{x}{y} \) per tot \( n \in
\N \), per inducció. A més
\begin{align*}
	\inn{-x}{y} & = \frac{\norm{-x+y}^2 - \norm{-x-y}^2}{4} = \frac{\norm{x-y}^2 -
	\norm{x+y}^2}{4} \\
							& = -\frac{\norm{x+y}^2 - \norm{x-y}^2}{4} = -\inn{x}{y}
\end{align*}
per tant tenim bilinealitat amb coeficients enters. I amb un petit càlcul podem demostrar
que els escalars de la forma \( \frac{1}{n} \) també surten fora:
\begin{align*}
	\inn{\frac{1}{n}x}{y} &= \frac{1}{4}\left(\norm{\frac{1}{n}x + y}^2 - \norm{\frac{1}{n}x -
	y}^2\right) \\
												&= \frac{1}{4n^2}\left(\norm{x+ny}^2 - \norm{x-ny}^2\right) \\
												&= \frac{1}{n^2}\inn{x}{ny} = \frac{1}{n^2}\inn{ny}{x} \\
												& = \frac{n}{n^2}\inn{y}{x} = \frac{1}{n}\inn{y}{x} =
												\frac{1}{n}\inn{x}{y}.
\end{align*}
Tenim, doncs, bilinealitat amb coeficients racionals. Però això és suficient per a
concloure bilinealitat amb coeficients reals. En efecte, per a tot \( \lambda \in \R \),
hi ha una successió de racionals \( (r_n) \) amb límit \( \lambda \). Com que \(
\inn{\cdot}{\cdot} \) és contínua ---com a conseqüència immediata de la continuïtat de la
norma---,
\begin{equation*}
\inn{\lambda x}{y} = \inn{\lim_{n \to \infty}r_n x}{y} = \lim_{n \to \infty} \inn{r_n
x}{y} = \lim_{n \to \infty} r_n \inn{x}{y} = \lambda \inn{x}{y}
\end{equation*}
com volíem.

\section*{Problema 2}
Considerem l'espai de Hilbert \( L^2([-\pi, \pi]) \) amb el producte escalar
\begin{equation*}
	\inn{f}{g} = \int_{[-\pi, \pi]} fg.
\end{equation*}
Dins d'aquest espai hi considerem les funcions donades per \( x \mapsto 1 \), \( x \mapsto
x\) i \( x \mapsto x^2 \), que denotarem simplement per \( 1 \), \( x \) i \( x^2 \)
respectivament. Volem ortonormalitzar-les, és a dir, trobar una base ortonormal del
subespai \( \gen{1, x, x^2} \). Ho farem amb el mètode de Gram-Schmidt.

Considerem les funcions
\begin{gather*}
	f_1 = x - \frac{\inn{1}{x}}{\inn{1}{1}}1 \\
	g_1 = x^2 - \frac{\inn{1}{x^2}}{\inn{1}{1}}1.
\end{gather*}
Tenim
\begin{equation*}
	\inn{1}{f_1} = \inn{x}{1} - \inn{1}{\frac{\inn{1}{x}}{\inn{1}{1}}1} = \inn{x}{1} -
	\inn{1}{x} = 0
\end{equation*}
i pel mateix argument \( \inn{1}{g_1} = 0 \). Calculem exactament \( f_1 \) i \( g_1 \).
Ens fan falta els productes \( \inn{1}{x} \) i \( \inn{1}{x^2} \):
\begin{gather*}
	\inn{1}{x} = \int_{[-\pi, \pi]} x = 0 \\
	\inn{1}{x^2} = \int_{[-\pi, \pi]} x^2 = \frac{2\pi^3}{3}.
\end{gather*}
Per tant, com que \( \inn{1}{1} = m([-\pi,\pi]) = 2\pi \)
\begin{gather*}
	f_1 = x \\
	g_1 = x^2 - \frac{2\pi^3}{6\pi} = x^2 - \frac{\pi^2}{3}.
\end{gather*}
Ara hem de buscar una base ortogonal de l'espai \( \gen{f_1, g_1} \). Podríem tornar a
aplicar Gram-Schmidt, però no cal perquè
\begin{equation*}
	\inn{f_1}{g_1} = \int_{[-\pi,\pi]} x^3 - \frac{\pi^2}{3}x = 0
\end{equation*}
per tant \( f_1 \) i \( g_1 \) ja són ortogonals.

Tenim, doncs, que \( 1 \), \( f_1 \) i \( g_1 \) són una base ortogonal de \(
\gen{1,x,x^2} \). Per a que siguin una base ortonormal les hem de normalitzar. Com que \(
\inn{1}{1} = \norm{1}^2 = \frac{1}{2\pi} \), \( \frac{1}{\sqrt{2\pi}} \)\footnote{Amb això
el que volem dir és la funció 1 multiplicada per \( \frac{1}{\sqrt{2\pi}} \)} té norma 1.
De la mateixa manera, com que
\begin{equation*}
	\norm{f_1}^2 = \inn{f_1}{f_1} = \int_{[-\pi,\pi]} f_1^2 = \int_{[-\pi, \pi]} x^2 =
	\frac{2\pi^3}{3}
\end{equation*}
la funció
\begin{equation*}
	\tilde{f}_1 = \sqrt{\frac{3}{2\pi^3}} x
\end{equation*}
té norma 1.

De la mateixa manera, com que
\begin{equation*}
	\norm{g_1}^2 = \inn{g_1}{g_1} = \int_{[-\pi, \pi]} g_1^2 = \int_{[-\pi,\pi]} x^4 -
	\frac{2\pi^2}{3}x^2 + \frac{\pi^4}{9} = \frac{2\pi^5}{5} -
	\frac{4\pi^5}{9} + \frac{2\pi^5}{9} = \frac{8\pi^5}{45}
\end{equation*}
la funció
\begin{equation*}
	\tilde{g}_1 = \sqrt{\frac{45}{8\pi^5}} g_1 = \frac{3}{2}\sqrt{\frac{5}{2\pi^5}}\left(x^2 -
	\frac{\pi^2}{3}\right) 
\end{equation*}
té norma 1.

Així doncs, \( \frac{1}{\sqrt{2\pi}} \), \( \tilde{f}_1 \) i \( \tilde{g}_1 \) són una
base ortonormal de \( \gen{1,x,x^2} \).

\parbreak

El subespai \( F \defeq \gen{1,x,x^2} \subseteq L^2([-\pi, \pi]) \) és tancat per ser de
dimensió finita, per tant la projecció sobre seu està ben definida. Volem calcular la
projecció de \( x \mapsto \sin{2x} \), que denotarem simplement per \( \sin{2x} \), sobre
\( F \), que és el polinomi \( p \) de grau 2 o menys que és més a prop de \( \sin{2x} \).
És a dir, \( p = P_F \sin{2x} \). Sabem que això és equivalent a dir \( p - \sin{2x} \in
F^\perp \). Per bilinealitat, és suficient requerir que \( p - \sin{2x} \) sigui ortogonal
als elements d'una base de \( F \). A l'apartat anterior hem calculat una base ortonormal
de \( F \), la qual cosa simplifica molt els càlculs. Denotem
\begin{gather*}
	e_0 = \frac{1}{\sqrt{2\pi}} \\
	e_1 = \tilde{f}_1 \\
	e_2 = \tilde{g}_1
\end{gather*}

Les condicions que ha de satisfer \( p \) són
\begin{equation*}
	\inn{p - \sin{2x}}{e_0} = \inn{p - \sin{2x}}{e_1} = \inn{p - \sin{2x}}{e_2} = 0.
\end{equation*}
Si \( p = a_0 + a_1x + a_2x^2 \) aleshores, per l'ortonormalitat de la base que hem triat,
\begin{equation*}
	\inn{e_i}{p} = a_i
\end{equation*}
per \( i = 0,1,2 \). Aleshores les equacions que ha de satisfer \( p \) són
\begin{equation*}
	\inn{p - \sin{2x}}{e_i} = \inn{p}{e_i} - \inn{\sin{2x}}{e_i} = 0
\end{equation*}
per a \( i = 0,1,2 \). Per tant només hem de calcular els tres productes \(
\inn{\sin{2x}}{e_i} \).
\begin{align*}
	\inn{\sin{2x}}{e_0} & = \frac{1}{\sqrt{2\pi}} \int_{[-\pi,\pi]} \sin{2x} = 0 \\
	\inn{\sin{2x}}{e_1} & = \sqrt{\frac{3}{2\pi^3}} \int_{[-\pi,\pi]} x\sin{2x} \\
											& = \sqrt{\frac{3}{2\pi^3}}\left(\frac{-\pi\cos{2\pi} - \pi
											\cos{(-2\pi)}}{2} + \frac{1}{2}\int_{[-\pi,\pi]}\cos{2x} \right) \\
											& = -\sqrt{\frac{3}{8\pi}} \\
	\inn{\sin{2x}}{e_2} & = \frac{3}{2} \sqrt{\frac{5}{2\pi^5}}\left(\int_{[-\pi,\pi]}\left(x^2 -
	\frac{\pi^2}{3}\right)\sin{2x}\right) \\
											& = \frac{3}{2} \sqrt{\frac{5}{2\pi^5}} \int_{[-\pi,\pi]}
											x^2\sin{2x} \\
											& = \frac{3}{2} \sqrt{\frac{5}{2\pi^5}} \\
											& = \frac{3}{2} \sqrt{\frac{5}{2\pi^5}} \left(\frac{-\pi^2 \cos{2\pi} +
											(-\pi)^2 \cos{(-2\pi)}}{2} + \int_{[-\pi,\pi]} x\cos{2x} \right)
											\\
											& = \frac{3}{2} \sqrt{\frac{5}{2\pi^5}} \left(\frac{\pi \sin{2\pi} + \pi
											\sin{(-2\pi)}}{2} - \frac{1}{2}\int_{[-\pi, \pi]} \sin{2x} \right)
											\\
											& = 0.
\end{align*}
Per tant el polinomi que busquem és
\begin{equation*}
	p = -\sqrt{\frac{3}{8\pi}}x.
\end{equation*}




% \begin{equation*}
% 	\inn{p - \sin{2x}}{1} = \inn{p - \sin{2x}}{x} = \inn{p - \sin{2x}}{x^2} = 0. 
% \end{equation*}
% Si \( p = a + bx + cx^2 \)
% \begin{align*}
% 	\inn{p - \sin{2x}}{1} & = \int_{[-\pi,\pi]} p - \sin{2x} \\
% 												& = a \int_{[-\pi,\pi]} 1 + b
% 												\int_{[-\pi,\pi]} x + c \int_{[-\pi,\pi]}x^2 - \int_{[-\pi,\pi]} \sin{2x} \\
% 												& = 2\pi a + \frac{2\pi^3}{3} c.
% \end{align*}
% Per tant la condició \( \inn{p - \sin{2x}}{1} = 0 \) és equivalent a
% \begin{equation*}
% 	2\pi a + 2\pi^3 c = 0
% \end{equation*}
% o, equivalentment
% \begin{equation*}
% 	3a + \pi
% \end{equation*}


\end{document}
