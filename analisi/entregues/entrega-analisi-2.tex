\documentclass[12pt]{article}

\usepackage[utf8]{inputenc}
\usepackage[T1]{fontenc}
\usepackage[spanish]{babel}
\usepackage{lmodern}
\usepackage{geometry}
\usepackage{hyperref}
\usepackage[dvipsnames]{xcolor}
\usepackage[bf,sf,small,pagestyles]{titlesec}
\usepackage{titling}
\usepackage[font={footnotesize, sf}, labelfont=bf]{caption} 
\usepackage{siunitx}
\usepackage{graphicx}
\usepackage{tikz-cd}
\usetikzlibrary{babel}
\usepackage{booktabs}
\usepackage{amsmath,amssymb}
\usepackage[spanish,sort]{cleveref}
\usepackage{enumitem}

\geometry{
	a4paper,
	right = 2.5cm,
	left = 2.5cm,
	bottom = 3cm,
	top = 3cm
}
\renewcommand{\baselinestretch}{1.3}

\hypersetup{
	colorlinks,
	linkcolor = {red!50!blue},
	linktoc = page
}

\crefname{figure}{figura}{figures}
\crefname{table}{taula}{taules}
\numberwithin{table}{section}
\numberwithin{figure}{section}

\graphicspath{{./figs/}}


\renewcommand{\arraystretch}{1.4}


% Commands for real analysis
\renewcommand{\vec}[1]{\mathbf{#1}}

% Restriction of a function
\newcommand{\rest}[1]{\raisebox{-.5ex}{$|$}_{#1}}

% Real numbers
\newcommand{\R}{\mathbb{R}}

% Complex numbers
\newcommand{\C}{\mathbb{C}}

% Natural numbers
\newcommand{\N}{\mathbb{N}}

% Integers
\newcommand{\Z}{\mathbb{Z}}

% Script A, B, M, P
\newcommand{\A}{\mathcal{A}}
\newcommand{\B}{\mathcal{B}}
\newcommand{\M}{\mathcal{M}}
\renewcommand{\P}{\mathcal{P}}

% Identity
\newcommand{\id}{\mathrm{id}}

% Almost everywhere
\renewcommand{\ae}{\text{ a.e. }}

% Absolute value
\newcommand{\abs}[1]{\lvert #1 \rvert}

% Exterior measure
\newcommand{\ext}[1]{m^* \! \left( #1 \right)}  


% Pagestyles
\newpagestyle{pagina}{
	\headrule
	\sethead*{\sffamily {\bfseries Anàlisi real.} Espais normats i de Banach}{}{\theauthor}
	\footrule
	\setfoot*{}{}{\sffamily \thepage}
}
\renewpagestyle{plain}{
	\footrule
	\setfoot*{}{}{\sffamily \thepage}
}
\pagestyle{pagina}

\title{\sffamily {\bfseries Entrega 2:} Espais normats i de Banach}
\author{\sffamily Arnau Mas}
\date{\sffamily 6 de desembre de 2019}


\begin{document}
\maketitle
\section*{Problema 1}
Sigui \( f(x) = x^{-1/2}\chi_{(0,1)}(x) \). Donada una enumeració de \( \Q \), \(
\set{r_n}_{n = 1}^\infty \) definim
\begin{equation*}
	g(x) = \sum_{n = 1}^{\infty} \frac{1}{2^n} f(x - r_n).
\end{equation*}
Si definim 
\begin{equation*}
	f_n(x) \defeq \frac{1}{2^n}f(x - r_n) = \frac{\chi_{(0,1)}(x - r_n)}{2^n\sqrt{x - r_n}}
	= \frac{\chi_{(r_n, r_n + 1)}(x)}{2^n \sqrt{x - r_n}}
\end{equation*}
aleshores les \( f_n \) són totes positives i mesurables. Aleshores, aplicant el Teorema
de Beppo-Levi tenim
\begin{equation*}
	\int_\R \abs{g} = \int_\R g = \int_\R \sum_{n = 1}^{\infty} f_n = \sum_{n = 1}^{\infty} \int f_n.  
\end{equation*}
Calculem, doncs, la integral de \( f_n \). \( f_n \) és no nu\l.l si i només si \( 0 < x -
r_n < 1 \), és a dir, si i només si \( x \in (r_n, r_n + 1) \), per tant
\begin{align*}
	\int_\R f_n & = \int_{(r_n, r_n + 1)} f_n = \frac{1}{2^n}\int_{(r_n, r_n + 1)}
	\frac{1}{\sqrt{x - r_n}} \d{x} \\
							& = \frac{1}{2^n} \int_{(0,1)} \frac{1}{\sqrt{x}} \d{x} = \frac{2}{2^n} =
							\frac{1}{2^{n-1}}.
\end{align*}
I aleshores
\begin{equation*}
	\int_\R \abs{g} = \sum_{n = 1}^{\infty} \frac{1}{2^{n-1}} = 2 < \infty
\end{equation*}
per tant \( g \in L^1(\R) \).

Per a demostrar que \( g \) és finita\( \gpta{x \in \R} \) podem fer servir la desigualtat de Chebyshev, que
ens dóna, per tot \( N \in \N \)
\begin{equation*}
	m(g^{-1}([N,\infty])) \leq \frac{1}{N} \int_\R g = \frac{2}{N}.
\end{equation*}
El conjunt de punts on \( g(x) = \infty \) és la intersecció
\begin{equation*}
	\bigcap_{N = 1}^\infty g^{-1}([N, \infty])	
\end{equation*}
perquè si \( g(x) > \infty \) aleshores per tot \( N \in \N \), \( g(x) > N \). Com que \(
g^{-1}([N+1, \infty]) \subseteq g^{-1}([N, \infty]) \) i \( m(g^{-1}([1,\infty])) \leq 2 \)
es té, per continuïtat de la mesura de Lebesgue
\begin{equation*}
	\leb{\bigcap_{N = 1}^\infty g^{-1}([N, \infty])} = \lim_{N \to \infty} \leb{g^{-1}([N,
	\infty])} \leq \lim_{N \to \infty} \frac{2}{N} < 0.
\end{equation*}
Per tant, el conjunt de punts on \( g \) no és finita és nul, és a dir \( g(x) < \infty \gpta{x
\in \R} \).

\parbreak

A continuació demostrem que \( g \) no està fitada en cap interval d'interior no buit.
Sigui \( I \subseteq \R \) un interval amb \( \mathring{I} \neq \emptyset \). Per la
densitat de \( \Q \) dins de \( \R \) hi ha un racional dins de \( \mathring{I} \), \(
r_N \). I per tant, com que \( \mathring{I} \) conté una bola centrada en \( r_N \), \(
(r_N - \alpha, r_N + \alpha) \), i sense pèrdua de generalitat podem suposar \( \alpha <
1 \). Sigui \( x \in (r_N, r_N + \alpha \). Podem posar \( x = r_N + \epsilon \) amb \(
\epsilon \in (0,\alpha) \). Aleshores 
\begin{equation*}
	g(x) = \sum_{n = 1}^{\infty} f_n(x) \geq f_N(x) = \frac{\chi_{(0,1)}(x - r_N)}{2^N
	\sqrt{x - r_N}} = \frac{1}{2^N \sqrt{\epsilon}}.
\end{equation*}
El que ens diu això és que per tot \( x \in (r_N, r_N + \epsilon) \) es té
\begin{equation*}
	g(x) \geq \frac{1}{2^N \sqrt{\epsilon}},
\end{equation*}
per tant \( g \) no pot estar fitada en un entorn de \( r_N \) perquè sempre hi ha punts
prop de \( r_N \) que superen qualsevol fita. En particular \( g \) no està fitada a \( I
\), com volíem veure.

Com a conseqüència, \( g \) no pot ser contínua en un punt on és finita ja que no està
fitada en cap interval que contingui aquest punt. Com que el conjunt de punts on \( g \)
no pren valors finits és nul, concloem que \( g \) és discontínua gairebé a tot punt de \(
\R \).

\parbreak

Sabem que\( \gpta{x \in \R} \) \( g(x) < \infty \), per tant \( g(x)^2 < \infty \gpta{x
\in \R} \). 

Volem veure que \( g^2 \) no és integrable en cap interval \( I \) amb interior no buit. Tenim
\begin{equation*}
	\int_I \abs{g^2} = \int_I g^2 = \int_I \left(\sum_{n = 1}^{\infty} f_n \right)^2 \geq
	\int_I \sum_{n = 1}^{\infty} f_n^2 = \sum_{n = 1}^{\infty} \int_I f_n^2 
\end{equation*}
fent servir Beppo-Levi a l'última igualtat. Calculem, doncs, la integral de \( f_n^2 \) a
un interval \( I \). Com que
\begin{equation*}
	f_n(x)^2 = \left(\frac{\chi_{(r_n, r_n + 1)}(x)}{2^n \sqrt{x - r_n}}\right)^2 =
	\frac{\chi_{(r_n, r_n + 1)}(x)}{2^{2n} (x - r_n)}
\end{equation*}
es té
\begin{equation*}
	\int_I f_n^2 = \int_I \frac{\chi_{(r_n, r_n + 1)}(x)}{2^{2n} (x - r_n)} \d{x} = \int_{I
	\cap (r_n, r_n + 1)} \frac{1}{2^{2n} (x - r_n)} \d{x}.
\end{equation*}
Per a molts \( n \) aquesta integral serà nu\l.la perquè \( I \) i \( (r_n, r_n + 1) \)
seran disjunts. Ara bé, per la densitat de \( \Q \) dins de \( \R \), sempre trobarem \( N
\in \N \) tal que \( r_N \in \mathring{I} \). Aleshores \( I \cap (r_N, r_N + 1) = (r_N,
\alpha) \) amb \( \alpha > r_N \), i per tant
\begin{equation*}
	\int_I f_N^2 = \int_{(r_N, \alpha)} \frac{1}{2^{2N} (x - r_N)} \d{x} = \frac{1}{2^{2N}}
	\int_{(0, \alpha - r_N)} \frac{1}{x} \d{x} = \infty.
\end{equation*}
Amb això ens és suficient per a provar que la integral de \( g^2 \) sobre \( I \)
divergeix:
\begin{equation*}
	\int_I g^2 \geq \sum_{n = 1}^{\infty} \int_I f_n^2 \geq \int_I f^2_N = \infty. 
\end{equation*}
Per tant \( g^2 \notin L^1(I) \) per a qualsevol interval \( I \) de \( \R \).


\section*{Problema 2}
Sigui \( (E, \norm{\cdot}) \) un espai normat. Volem veure que per tot \( x,y \in E \) es té
la desigualtat
\begin{equation*}
	\norm{x} \leq \max\set{\norm{x - y}, \norm{x + y}}.
\end{equation*}
Per a qualsevol \( t \in [0,1] \) tenim 
\begin{align*}
	\norm{t(x - y) + (1 - t)(x + y)} & \leq t\norm{x - y} + (1 - t)\norm{x + y} \\
																	 & = \norm{x+y} +	(\norm{x-y} - \norm{x+y})t.
\end{align*}
L'aplicació
\begin{align*}
	f \colon [0,1] & \longrightarrow \R \\
	t & \longmapsto \norm{x+y} + (\norm{x-y} - \norm{x+y})t
\end{align*}
és afí, per tant assoleix el seu màxim a un dels extrems de \( [0,1] \). Tenim que \( f(0) =
\norm{x+y} \) i \( f(1) = \norm{x -y} \), per tant, per a tot \( t \in [0,1] \) es té
\begin{equation*}
	\norm{t(x - y) + (1 - t)(x + y)} \leq f(t) \leq \max\set{\norm{x+y}, \norm{x-y}}.
\end{equation*}
Fent \( t = \frac{1}{2} \) arribem a la desigualtat que volíem veure
\begin{equation*}
	\norm{x} = \norm{\frac{1}{2}(x-y) + \frac{1}{2}(x+y)} \leq \max\set{\norm{x+y},
	\norm{x-y}}.
\end{equation*}

\parbreak

Hem de veure que per tot \( x,y \in E \) es té
\begin{equation*}
	\norm{x - y} \geq \frac{1}{2} \max\set{\norm{x}, \norm{y}} \norm{\frac{x}{\norm{x}} -
	\frac{y}{\norm{y}}}.
\end{equation*}
Aixó és equivalent a provar que 
\begin{equation*}
	\norm{x - y} \geq \frac{1}{2} \norm{x} \norm{\frac{x}{\norm{x}} -
	\frac{y}{\norm{y}}}
\end{equation*}
i 
\begin{equation*}
	\norm{x - y} \geq \frac{1}{2} \norm{y} \norm{\frac{x}{\norm{x}} -
	\frac{y}{\norm{y}}}.
\end{equation*}

Demostrem la primera de les desigualtats:
\begin{align*}
	\frac{1}{2} \norm{x} \norm{\frac{x}{\norm{x}} -	\frac{y}{\norm{y}}} 
	& = \frac{1}{2}\norm{x - \frac{\norm{x}}{\norm{y}}y} \\
	& \leq \frac{1}{2}\norm{x - y} + \frac{1}{2}\norm{y - \frac{\norm{x}}{\norm{y}}y} \\
	& = \frac{1}{2}\norm{x - y} + \frac{1}{2}\abs{1 - \frac{\norm{x}}{\norm{y}}}\norm{y} \\
	& = \frac{1}{2}\norm{x - y} + \frac{1}{2}\abs{\norm{y} - \norm{x}} \\
	& \leq \norm{x - y}
\end{align*}
fent servir la desigualtat triangular inversa a l'últim pas.

La segona desigualtat és immediata observant que \( \norm{x - y} = \norm{y - x} \). Així
doncs tenim el resultat que volíem.

\parbreak

Per últim hem de veure que
\begin{equation*}
	\norm{x - y} \geq \frac{1}{4} (\norm{x} + \norm{y}) \norm{\frac{x}{\norm{x}} -
	\frac{y}{\norm{y}}}.
\end{equation*}
Això és una conseqüència immediata de la desigualtat anterior. En efecte, tenim
\begin{equation*}
	\norm{x} + \norm{y} \leq 2 \max\set{\norm{x}, \norm{y}}
\end{equation*}
per tant
\begin{align*}
	\norm{x - y} & \geq \frac{1}{2} \max\set{\norm{x}, \norm{y}} \norm{\frac{x}{\norm{x}} -
	\frac{y}{\norm{y}}} \\
							 & \geq \frac{1}{4} (\norm{x} + \norm{y}) \norm{\frac{x}{\norm{x}} -
	\frac{y}{\norm{y}}}.
\end{align*}

\section*{Problema 3}
Volem veure que l'espai
\begin{equation*}
	\class{s}{[a,b]} \defeq \set{f \in \cont{[a,b]} \mid p_s(f) < \infty}
\end{equation*}
on
\begin{equation*}
	p_s(f) \defeq \sup_{x \neq y} \frac{\abs{f(x) - f(y)}}{\abs{x - y}^s}
\end{equation*}
és un espai de Banach amb la norma
\begin{equation*}
	\norm{f}_s \defeq \abs{f(a)} + p_s(f).
\end{equation*}

És clar que \( \norm{f}_s \geq 0 \) per tot \( f \in \class{s}{[a,b]} \). Comprovem que \(
\norm{\cdot}_s \) satisfà els tres requeriments per a ser una norma.

En primer lloc, hem de veure que si \( \norm{f} = 0 \) aleshores \( f = 0 \). Si \(
\norm{f} = 0 \) aleshores \( \abs{f(a)} = p_s(f) = 0 \). Com que \( p_s(f) = 0 \) es té,
per tot \( x \in (a,b] \) que
\begin{equation*}
	\frac{\abs{f(x) - f(a)}}{\abs{x - a}^s} \leq p_s(f) = 0
\end{equation*}
per tant
\begin{equation*}
	\frac{\abs{f(x) - f(a)}}{\abs{x - a}^s} = 0.
\end{equation*}
Això vol dir que \( \abs{f(x) - f(a)} = 0 \), perquè \( x \neq a \) i per tant \( \abs{x -
a} \neq 0 \). Per tant, per tot \( x \in (a,b] \) es té \( f(x) = f(a) \). I com que \(
\abs{f(a)} \) concloem \( f(x) = f(a) = 0 \), és a dir, \( f = 0 \).

Veiem ara que la norma \( \norm{\cdot}_s \) satisfà la desigualtat triangular. Donades \(
f, g \in \class{s}{[a,b]} \) tenim
\begin{align*}
	p_s(f + g) & = \sup_{x \neq y} \frac{\abs{(f+g)(x) - (f+g)(y)}}{\abs{x-y}^s} \\
						 & = \sup_{x \neq y} \frac{\abs{f(x) - f(y) + g(x) - g(y)}}{\abs{x-y}^s} \\
						 & \leq \sup_{x \neq y} \frac{\abs{f(x) - f(y)} + \abs{g(x) - g(y)}}{\abs{x-y}^s} \\
						 & \leq \sup_{x \neq y} \frac{\abs{f(x) - f(y)}}{\abs{x-y}^s} + \sup_{x \neq
					 y} \frac{\abs{g(x) - g(y)}}{\abs{x-y}} \\
					 	 & = p_s(f) + p_s(g).
\end{align*}
Això de fet demostra que \( \class{s}{[a,b]} \) és un subespai vectorial de \(
\cont{[a,b]} \), afegint que \( p_s(\lambda f) = \abs{\lambda}p_s(f) \). Tenim aleshores
\begin{equation*}
	\norm{f+g}_s = \abs{(f+g)(a)} + p_s(f+g) \leq \abs{f(a)} + \abs{g(a)} + p_s(f) + p_s(g)
	= \norm{f}_s + \norm{g}_s
\end{equation*}

Finalment, per tot \( \lambda \in \R \) tenim
\begin{equation*}
	p_s(\lambda f) = \sup_{x \neq y} \frac{\abs{\lambda f(x) - \lambda f(y)}}{\abs{x - y}^s}
	= \abs{\lambda}\sup_{x \neq y} \frac{\abs{ f(x) -  f(y)}}{\abs{x - y}^s} =
	\abs{\lambda}p_s(f).
\end{equation*}
Per tant és clar que \( \norm{\lambda f} = \abs{\lambda}\norm{f} \).

Fins aquí hem vist que \( \class{s}{[a,b]} \) amb la norma \( \norm{\cdot}_s \)	és un
espai normat. Per veure que és de banach hem de demostrar que és complet, és a dir, que
tota successió de Cauchy a \( \class{s}{[a,b]} \) és convergent. Sigui \( (f_n) \), doncs,
una successió de Cauchy a \( \class{s}{[a,b]} \). Per definició, per qualsevol \( \epsilon
> 0 \) hi ha \( N \in \N \) tal que si \( n, m \geq N \) es té
\begin{equation*}
	\norm{f_n - f_m}_s < \epsilon.
\end{equation*}
Això vol dir que \( p_s(f_n - f_m) < \epsilon \) i \( \abs{(f_n - f_m)(a)} < \epsilon \).
Amb això podem demostrar que a tot \( x \in (a,b] \), la successió \( f_n(x) \) és de
Cauchy i per tant convergent:
\begin{align*}
	\abs{f_n(x) - f_m(x)} & \leq \abs{f_n(x) - f_m(x) - f_n(a) + f_m(a)} + \abs{f_n(a) -
	f_m(a)} \\
												& = \abs{(f_n - f_m)(x) - (f_n - f_m)(a)} + \abs{(f_n - f_m)(a)}
												\\
												& \leq p_s(f_n - f_m)\abs{x - a}^s + \abs{(f_n - f_m)(a)} \\
												& \leq \epsilon \abs{b - a}^s + \epsilon.
\end{align*}
I com que \( \abs{f_n(a) - f_m(a)} < \epsilon \) per a \( n \) i \( m \) prou grans, \(
f_n(a) \) és de Cauchy i per tant també convergent. La successió \( f_n \), doncs, té
límit puntual a \( [a,b] \), diguem-li \( f \). 

Per a demostrar que \( \class{s}{[a,b]} \) és complet hem de comprovar que \( f \in
\class{s}{[a,b]} \) i que \( f_n \to f \) en la norma \( \norm{\cdot}_s \). Com que \(
(f_n) \) és de Cauchy es té, per tot \( x, y \in [a,b] \) amb \( x \neq y \) i \( m,n \)
prou grans
\begin{equation*}
	\frac{\abs{(f_m - f_n)(x) - (f_m - f_n)(y)}}{\abs{x - y}^s} < \epsilon.
\end{equation*}
Prenent el límit puntual quan \( m \to \infty \) tenim, per tot \( x,y \in [a,b] \) amb \(
x \neq y \)
\begin{equation*}
	\frac{\abs{(f - f_n)(x) - (f - f_n)(y)}}{\abs{x - y}^s} < \epsilon,
\end{equation*}
per tant \( p_s(f - f_n) < \epsilon \). Això vol dir que \( f - f_n \in \class{s}{[a,b]}
\). En efecte, si per a una funció \( g \) definida a \( [a,b] \) es té \( p_s(g) \)
aleshores \( g \) és \( s \)-Hölder a \( [a,b] \) i en particular contínua. Per tant, com
que \( f_n \in \class{s}{[a,b]} \) per tot \( n \), tenim
\begin{equation*}
	f = f - f_n + f_n \in \class{s}{[a,b]}.
\end{equation*}

Amb el que acabem de veure és immediat comprovar que \( f_n \to f \) en la norma \(
\norm{\cdot}_s \). D'una banda, com que \( f \) és el límit puntual de les \( f_n \), és
clar que 
\begin{equation*}
	\abs{(f - f_n)(a)} \to 0
\end{equation*}
quan \( n \to \infty \). D'altra banda, pel que hem vist prèviament, per a \( n \) prou
gran \( p_s(f - f_n) < \epsilon \). És a dir, \( p_s(f - f_n) \to 0 \) quan \( n \to
\infty \). I per tant
\begin{equation*}
	\norm{f - f_n}_s \xrightarrow{n \to \infty} 0
\end{equation*}
com volíem.


\end{document}
