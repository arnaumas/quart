\documentclass[12pt]{article}

\input{prelude-entrega.tex}
% COMMANDS FOR REAL ANALYSIS
% ----------------------------------------------------------

% Restriction of a function
\newcommand{\rest}[1]{\raisebox{-.5ex}{$|$}_{#1}}

% Real numbers
\newcommand{\R}{\mathbb{R}}

% Rational numbers
\newcommand{\Q}{\mathbb{Q}}

% Complex numbers
\newcommand{\C}{\mathbb{C}}

% Natural numbers
\newcommand{\N}{\mathbb{N}}

% Integers
\newcommand{\Z}{\mathbb{Z}}

% Script A, B, M, P, L
\newcommand{\A}{\mathcal{A}}
\newcommand{\B}{\mathcal{B}}
\newcommand{\M}{\mathcal{M}}
\renewcommand{\L}{\mathcal{L}}
\renewcommand{\P}{\mathcal{P}}

\newcommand{\class}[2]{C^{#1}(#2)}
\newcommand{\cont}[1]{\class{0}{#1}}

% Identity
\newcommand{\id}{\mathrm{id}}

% Almost everywhere
\renewcommand{\ae}{\text{ a.e.}}

% Almost everywhere on
\newcommand{\aeon}[1]{\text{ a.e. on } #1}

% Absolute value
\newcommand{\abs}[1]{\left\lvert #1 \right\rvert}

% Norm
\newcommand{\norm}[1]{\left\lVert #1 \right\rVert}
\newcommand{\normu}[1]{\norm{#1}_1}
\newcommand{\normd}[1]{\norm{#1}_2}
\newcommand{\normp}[1]{\norm{#1}_p}
\newcommand{\normi}[1]{\norm{#1}_\infty}

% Inner product
\newcommand{\inn}[2]{\left\langle #1\, , #2 \right\rangle}

% Exterior measure
\newcommand{\ext}[1]{m^* \! \left( #1 \right)}  

% Integral differential
\renewcommand{\d}{\, \mathrm{d}}

% Defined as
\makeatletter
\newcommand*{\defeq}{\mathrel{\rlap{%
    \raisebox{0.3ex}{$\m@th\cdot$}}%
  \raisebox{-0.3ex}{$\m@th\cdot$}}%
	=
}
\makeatother

% Set
\newcommand{\set}[1]{ \{ #1 \} }

% Break between paragraphs
\newcommand{\parbreak}{
	\begin{center}
		--- $\ast$ ---
	\end{center} 
}


% Pagestyles
\newpagestyle{pagina}{
	\headrule
	\sethead*{\sffamily {\bfseries Anàlisi real.} Teoria de la mesura}{}{\theauthor}
	\footrule
	\setfoot*{}{}{\sffamily \thepage}
}
\renewpagestyle{plain}{
	\footrule
	\setfoot*{}{}{\sffamily \thepage}
}
\pagestyle{pagina}

\title{\sffamily {\bfseries Entrega 1:} Teoria de la mesura}
\author{\sffamily Arnau Mas}
\date{\sffamily 18 d'octubre de 2019}


\begin{document}
\maketitle

\section*{Problema 1}
El conjunt de Cantor és un subconjunt compacte i completament desconnectat de \( [0,1] \) amb
mesura 0. La seva construcció estàndard consisteix en eliminar el terç central de
l'interval unitat. A continuació s'elimina el terç central dels dos intervals resultants,
i el terç central dels quatre intervals que apareixen. Al pas \( n \) apareixen \( 2^n \)
intervals, la unió dels cuals escrivim \( C_n \). El conjunt de Cantor és la
intersecció de tots els \( C_n \). Per a veure que té mesura nu\l.la, observem que \( C_0
\) és l'interval unitat, un conjunt de mesura 1. Al següent pas s'elimina un conjunt de
mesura \( \frac{1}{3} \) i per tant \( C_1 \) té mesura \( \frac{2}{3} \). A cada iteració
s'elimina un terç de la mesura restant, per tant \( m(C_n) = \left(\frac{2}{3}\right)^n
\). De fet es té \( C_{n+1} \subseteq C_n \). A més, tots els \( C_n \) són mesurables per
ser unió d'intervals. Per tant \( C \) també és mesurable perquè és intersecció de mesurables, i
per continuïtat de la mesura de Lebesgue
\begin{equation*}
	m(C) = m\left(\bigcap_{n = 0}^{\infty} C_n\right) = \lim_{n \to \infty} m(C_n) = \lim_{n \to
	\infty} \left(\frac{2}{3}\right)^n = 0. 
\end{equation*}

De fet, aquest procés es pot repetir treient a cada pas intervals centrats de longitud
en proporció \( \alpha \) amb la longitud dels intervals anteriors, amb \( 0 < \alpha < 1
\) (el cas estàndard del conjunt de Cantor ternari correspon a \( \alpha = \frac{1}{3}
\)). Al primer pas el conjunt que queda té longitud total \( 1 - \alpha \). Al segon pas
queda un conjunt de longitud \( (1 - \alpha) \), i al pas \( n \)-èssim la longitud
restant és \( (1 - \alpha)^n \), per tant en el límit \( n \to \infty \) el conjunt que
obtenim tindrà mesura 0. Així doncs, per a obtenir un conjunt de Cantor de mesura positiva
haurem de procedir de manera diferent. 

Una altra manera de generalitzar la construcció del conjunt ternari de Cantor és observar
que a cada pas els intervals que eliminem tenen mesura \( \frac{1}{3^n} \). En efecte,
després de la primera iteració queden dos intervals de longitud \( \frac{1}{3} \), i de
cada un n'eliminem un terç, per tant un interval de longitud \( \frac{1}{3}\frac{1}{3} =
\frac{1}{9} \). Això dona lloc a quatre intervals de longitud \( \frac{1}{9} \). Procedint
inductivament veiem que a cada pas estem eliminant un terç d'un interval la longitud del
qual és \( \frac{1}{3^n} \). De fet, observant que al pas \( n \)-èssim s'eliminen \(
2^{n-1} \) intervals de longitud \( \frac{1}{3^n} \) tenim una altra prova de que el
conjunt de Cantor té mesura zero. En efecte, si denotem per \( I_n \) la unió dels \(
2^{n-1} \) intervals que eliminem al pas \( n \)-èssim aleshores el conjunt de Cantor
també és
\begin{equation*}
	C = [0,1] - \bigcup_{n = 1}^{\infty} I_n.
\end{equation*}
Pel que acabem de veure, \( m(I_n) = \frac{2^{n-1}}{3^n} \), per tant, com que els \( I_n
\) són dos a dos disjunts
\begin{equation*}
	m(C) = 1 - \sum_{n = 1}^{\infty} \frac{2^{n-1}}{3^n} = 1 - \frac{1}{3}\left(\frac{1}{1 -
	\frac{2}{3}}\right) = 0. 
\end{equation*}

Ara bé, si a cada iteració eliminem una mica menys, intervals de mesura \(
\frac{\lambda}{3^n} \) amb \( 0 < \lambda < 1 \) el que obtenim és un conjunt \(
C_\lambda \) de mesura
\begin{equation*}
	m(C_\lambda) = 1 - \frac{\lambda}{3}\left(\frac{1}{1 - \frac{2}{3}}\right) = 1 -
	\lambda.
\end{equation*}

Comprovem ara que \( C_\lambda \) és un conjunt amb les propietats que volem: un
subconjunt compacte i totalment disconnex de \( [0,1] \). Vegem primer que la construcció
dóna lloc a un conjunt no buit. A cada pas estem eliminant intervals que estan continguts
dins dels intervals que eliminem en la construcció del conjunt de Cantor, per tant el
conjunt que obtenim en el límit no és buit, en particular conté el conjunt de Cantor. Que
és compacte és clar: a cada pas elimnem una unió d'intervals oberts d'un tancat, per tant
el resultat és un tancat. Així doncs la seva intersecció \( C_\lambda \) també es tancada,
i per ser subconjunt de \( [0,1] \) és fitada i per tant compacta. 

Vegem, per últim, que és totalment disconnnex. Siguin \( x, y \in C_\lambda \). 


\section*{Problema 2}
Sigui \( A \subseteq R \) on \( R \) és un rectangle de \( \R^n \) tal que per qualsevol
compacte \( K \subseteq R \) de mesura positiva es té que \( K \cap A \neq \emptyset \).
Volem veure que \( \ext{A} = v(R) \). 

En primer lloc, com que \( A \subseteq R \) aleshores \( \ext{A} \leq \ext{R} = v(R) \),
per tant només hem de veure \( v(R) \leq \ext{A} \).

Com que \( R \) és mesurable tenim que
\begin{equation*}
	v(R) = m(R) = \sup_{K \subseteq R \text{ compacte}} m(K).
\end{equation*}
I com que \( m(R) = m(\bar{R}) \) també és cert
\begin{equation*}
	v(R) = m(\bar{R}) = \sup_{K \subseteq \bar{R} \text{ compacte}} m(K).
\end{equation*}
Sigui, doncs, \( K \subseteq R \) un compacte. Per hipòtesi 

Per qualsevol \( \epsilon > 0 \) existeix \( F \subseteq R \) tancat amb \( \ext{R - F} <
\epsilon \). Com que \( R \) és fitat aleshores \( F \) també ho és i per tant és
compacte. Per hipòtesi, doncs, \( F \cap A \neq \emptyset \).

\section*{Problema 3}
Busquem un conjunt \( B \subseteq \R \) no mesurable tal que \( B \cap \Q^c \) sigui
mesurable. És a dir, tal que el conjunt d'irracionals de \( B \) sigui mesurable. Podem
escriure \( B \) com la unió disjunta entre el conjunt de racionals de \( B \) i el
conjunt d'irracionals de \( B \),
\begin{equation*}
	B = (B \cap \Q) \cup (B \cap \Q^c) 
\end{equation*}
Sigui quin sigui el conjunt \( B \), com que \( B \cap \Q \subseteq \Q \) aleshores \( B
\cap \Q \) és mesurable. Això és perquè els racionals i qualsevol subconjunt seu són
numerables i per tant mesurables (de fet són de mesura 0). Així doncs, si \( B \cap \Q^c
\) és mesurable \( B \) és automàticament mesurable. En particular no pot existir
qualsevol conjunt no mesurable tal que la seva intersecció amb els racionals sigui
mesurable.


\end{document}
