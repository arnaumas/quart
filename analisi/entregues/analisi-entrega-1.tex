\documentclass[12pt]{article}

% Packages
\usepackage[utf8]{inputenc}
\usepackage[T1]{fontenc}
\usepackage[catalan]{babel}
\usepackage{lmodern}
\usepackage{geometry}
\usepackage{hyperref}
\usepackage[dvipsnames]{xcolor}
\usepackage[bf,sf,small,pagestyles]{titlesec}
\usepackage{titling}
\usepackage[font={footnotesize, sf}, labelfont=bf]{caption} 
\usepackage{siunitx}
\usepackage{graphicx}
\usepackage{booktabs}
\usepackage{amsmath,amssymb}
\usepackage[catalan,sort]{cleveref}
\usepackage{enumitem}

% Geometry setup
\geometry{
	a4paper,
	right = 2.5cm,
	left = 2.5cm,
	bottom = 3cm,
	top = 3cm
}
% Wider space between lines
\renewcommand{\baselinestretch}{1.3}

% Hyperref setup
\hypersetup{
	colorlinks,
	linkcolor = {red!50!blue},
	linktoc = page
}

% Cref names in catalan
\crefname{figure}{figura}{figures}
\crefname{table}{taula}{taules}
\numberwithin{table}{section}
\numberwithin{figure}{section}

\graphicspath{{./figs/}}

% Unitats
\sisetup{
	inter-unit-product = \ensuremath{ \cdot },
	allow-number-unit-breaks = true,
	detect-family = true,
	list-final-separator = { i },
	list-units = single
}


% Commands for real analysis
\renewcommand{\vec}[1]{\mathbf{#1}}

% Restriction of a function
\newcommand{\rest}[1]{\raisebox{-.5ex}{$|$}_{#1}}

% Real numbers
\newcommand{\R}{\mathbb{R}}

% Complex numbers
\newcommand{\C}{\mathbb{C}}

% Natural numbers
\newcommand{\N}{\mathbb{N}}

% Integers
\newcommand{\Z}{\mathbb{Z}}

% Script A, B, M, P
\newcommand{\A}{\mathcal{A}}
\newcommand{\B}{\mathcal{B}}
\newcommand{\M}{\mathcal{M}}
\renewcommand{\P}{\mathcal{P}}

% Identity
\newcommand{\id}{\mathrm{id}}

% Almost everywhere
\renewcommand{\ae}{\text{ a.e. }}

% Absolute value
\newcommand{\abs}[1]{\lvert #1 \rvert}

% Exterior measure
\newcommand{\ext}[1]{m^* \! \left( #1 \right)}  


% Pagestyles
\newpagestyle{pagina}{
	\headrule
	\sethead*{\sffamily {\bfseries Anàlisi real.} Teoria de la mesura}{}{\theauthor}
	\footrule
	\setfoot*{}{}{\sffamily \thepage}
}
\renewpagestyle{plain}{
	\footrule
	\setfoot*{}{}{\sffamily \thepage}
}
\pagestyle{pagina}

\title{\sffamily {\bfseries Entrega 1:} Teoria de la mesura}
\author{\sffamily Arnau Mas}
\date{\sffamily 18 d'octubre de 2019}


\begin{document}
\maketitle

\section*{Problema 1}
El conjunt de Cantor és un subconjunt compacte i completament desconnectat de \( [0,1] \) amb
mesura 0. La seva construcció estàndard consisteix en eliminar el terç central de
l'interval unitat. A continuació s'elimina el terç central dels dos intervals resultants,
i el terç central dels quatre intervals que apareixen. Al pas \( n \) apareixen \( 2^n \)
intervals, la unió dels cuals escrivim \( C_n \). El conjunt de Cantor és la
intersecció de tots els \( C_n \). Per a veure que té mesura nu\l.la, observem que \( C_0
\) és l'interval unitat, un conjunt de mesura 1. Al següent pas s'elimina un conjunt de
mesura \( \frac{1}{3} \) i per tant \( C_1 \) té mesura \( \frac{2}{3} \). A cada iteració
s'elimina un terç de la mesura restant, per tant \( m(C_n) = \left(\frac{2}{3}\right)^n
\). De fet es té \( C_{n+1} \subseteq C_n \). A més, tots els \( C_n \) són mesurables per
ser unió d'intervals. Per tant \( C \) també és mesurable perquè és intersecció de mesurables, i
per continuïtat de la mesura de Lebesgue
\begin{equation*}
	m(C) = m\left(\bigcap_{n = 0}^{\infty} C_n\right) = \lim_{n \to \infty} m(C_n) = \lim_{n \to
	\infty} \left(\frac{2}{3}\right)^n = 0. 
\end{equation*}

De fet, aquest procés es pot repetir treient a cada pas intervals centrats amb longitud
en proporció \( \alpha \) amb la longitud dels intervals anteriors, amb \( 0 < \alpha < 1
\) (el cas estàndard del conjunt de Cantor ternari correspon a \( \alpha = \frac{1}{3}
\)). I pel mateix argument concloem que el conjunt resultant també és un conjunt nul. Així
doncs, per a obtenir un conjunt de Cantor de mesura positiva haurem de procedir de manera
diferent. 

\end{document}
