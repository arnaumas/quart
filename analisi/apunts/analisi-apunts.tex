\documentclass[12pt,oneside]{book}

\usepackage[utf8]{inputenc}
\usepackage[T1]{fontenc}
\usepackage[english]{babel}
\usepackage{lmodern}
\usepackage{geometry}
\usepackage{hyperref}
\usepackage[dvipsnames]{xcolor}
\usepackage[bf,sf,small,pagestyles]{titlesec}
\usepackage{titling}
\usepackage[font={footnotesize, sf}, labelfont=bf]{caption} 
\usepackage{siunitx}
\usepackage{graphicx}
\usepackage{tikz-cd}
\usepackage{booktabs}
\usepackage{amsmath,amssymb}
\usepackage[sort]{cleveref}
\usepackage{amsthm,thmtools}
\usepackage[shortlabels]{enumitem}

\geometry{
	a4paper,
	right = 3cm,
	left = 3cm,
	bottom = 3cm,
	top = 3cm
}

\hypersetup{
	colorlinks,
	linkcolor = {red!50!blue},
	linktoc = page
}

\numberwithin{table}{section}
\numberwithin{equation}{section}
\numberwithin{figure}{section}

\newcommand{\qedtriangle}{\ensuremath{\triangle}}
\newcommand{\qedtriangledown}{\ensuremath{\bigtriangledown}}
\declaretheoremstyle[spaceabove=6pt, spacebelow=6pt, headfont=\bfseries, notefont=\normalfont, notebraces={(}{)}, qed=\qedtriangle]{definition}
\declaretheoremstyle[spaceabove=6pt, spacebelow=6pt, headfont=\bfseries, notefont=\normalfont, notebraces={(}{)}, qed=\qedtriangledown]{example}

\declaretheorem[name=Theorem, refname={theorem,theorems}, Refname={Theorem,Theorem}, numberwithin=chapter]{theo}
\declaretheorem[name=Proposition, refname={proposition,propositions}, Refname={Proposition,Propositions}, numberlike=theo]{prop}
\declaretheorem[name=Definition, style=definition, refname={definition,definitions}, Refname={Definitio,Definitions}, numberwithin=chapter]{defn}
\declaretheorem[name=Example, style=example, refname={example,examples}, Refname={Example,Examples}, numberwithin=chapter]{exe}

\newlist{points}{enumerate}{1}
\setlist[points,1]{label=\textup{(}{\itshape \roman*}\textup{)}, wide}

\graphicspath{{./figs/}}
\newcommand{\dummyfig}[1]{
  \centering
  \fbox{
    \begin{minipage}[c][0.33\textheight][c]{0.5\textwidth}
      \centering{\ttfamily #1}
    \end{minipage}
  }
}

% Unitats
\sisetup{
	inter-unit-product = \ensuremath{ \cdot },
	allow-number-unit-breaks = true,
	detect-family = true,
	list-final-separator = { and },
	list-units = single
}

\renewcommand{\vec}[1]{\mathbf{#1}}
\newcommand{\rest}[1]{\raisebox{-.5ex}{$|$}_{#1}}
\newcommand{\R}{\mathbb{R}}
\newcommand{\C}{\mathbb{C}}
\newcommand{\N}{\mathbb{N}}
\newcommand{\A}{\mathcal{A}}
\newcommand{\B}{\mathcal{B}}
\renewcommand{\P}{\mathcal{P}}
\newcommand{\id}{\mathrm{id}}
\newcommand{\abs}[1]{\lvert #1 \rvert}
\newcommand{\parbreak}{
	\begin{center}
		--- $\ast$ ---
	\end{center} 
}
\makeatletter
\newcommand*{\defeq}{\mathrel{\rlap{%
    \raisebox{0.3ex}{$\m@th\cdot$}}%
  \raisebox{-0.3ex}{$\m@th\cdot$}}%
	=
}
\makeatother

\newpagestyle{page}[\sffamily \footnotesize]{
	\headrule
	\sethead*{\ifthesection{{\bfseries \thesection} \sectiontitle}{}}{}{{\bfseries Chapter \thechapter.} \chaptertitle}
	\footrule
	\setfoot*{}{}{\thepage}
}
\renewpagestyle{plain}[\sffamily \footnotesize]{
	\footrule
	\setfoot*{}{}{\thepage}
}
\assignpagestyle{\chapter}{plain}

\titleformat{\chapter}[block]{\sffamily \bfseries \Huge}{\filleft \large Chapter \Huge \thechapter\\}{0pt}{\Huge \titlerule[1pt] \vspace{1ex} \filleft}

\title{Real Analysis}
\author{Arnau Mas}
\date{2019}

\begin{document}
\maketitle

\frontmatter
\pagestyle{plain}
These are notes gathered during the subject \emph{Anàlisi Real i Funcional} as taught by Joan Orobitg between September 2019 and January 2020.

\mainmatter

\chapter{Measure Theory}
\section{Measure Spaces}
\begin{defn}[\( \sigma \)-algebra]
	We say a family of subsets \( \A \subseteq \P(X) \) of a set \( X \) is a \( \sigma \)-algebra over it if
	\begin{points}
	\item \( \emptyset, X \in \A \),
	\item \( \A \) is closed under countable unions, i.e. if there is a countable set \( \{ A_i \}_{i \in \N} \subseteq \A \) then \( \bigcup_{i \in \N} A_i \in \A \),
	\item \( \A \) is closed under countable intersections, i.e. if there is a countable set \( \{ A_i \}_{i \in \N} \subseteq \A \) then \( \bigcap_{i \in \N} A_i \in \A \),
	\item \( \A \) is closed under complements, i.e. if \( A \in \A \) then \( X - A \in \A \).
	\end{points}
\end{defn}

\begin{exe}
	The following are all examples of \( \sigma \)-algebras.
	\begin{points}
	\item For any set \( X \), \( \P(X) \) is a \( \sigma \)-algebra called the \emph{discrete \( \sigma \)-algebra}. It is the \emph{finest} \( \sigma \)-algebra since any other possible \( \sigma \)-algebra over \( X \) is contained in it.
	\item On the other hand, the \emph{coarsest} \( \sigma \)-algebra over any set \( X \) is simply \( \{ \emptyset, X \} \), meaning any other possible \( \sigma \)-algebra contains it. It is called the \emph{trivial \( \sigma \)-algebra}.
	\item If \( \A_1 \) and \( \A_2 \) are \( \sigma \)-algebras over a set \( X \) then so is \( \A_1 \cap \A_2 \).
	\item Given a family of subsets \( S \subseteq \P(X) \) then the \( \sigma \)-algebra generated by it is the intersection of all \( \sigma \)-algebras that contain it and it is the smallest \( \sigma \)-algebra that contains \( S \). We write it \( \sigma(S) \).
	\item The \emph{Borel \( \sigma \)-algebra} over a topological space \( X \) is the \( \sigma \)-algebra generated by the open sets of \( X \), written \( \B(X) \). Since a closed set is the complement of an open set the family of closed sets also generates the Borel \( \sigma \)-algebra.
	\end{points}
\end{exe}
The pair formed by a set and its \( \sigma \)-algebra is called a \emph{measurable space}.

\begin{defn}[Measure]
	Let \( (X,\A) \) be a measurable space. A measure is a map \( \mu \colon A \to [0,\infty] \) such that the following are true
	\begin{points}
	\item \( \mu(\emptyset) = 0 \).
	\item If \( \{ A_i \}_{i \in \N} \subseteq \A \) is a family of pairwise disjoint sets of finite measure then
		\begin{equation*}
			\mu\left(\bigcup_{i \in \N} A_i\right) = \sum_{i \in \N} \mu(A_i). 
		\end{equation*}
	\end{points}
\end{defn}
A measurable space equipped with a measure is called a \emph{measure space}.

\begin{exe}
	The following are all examples of measures.
	\begin{points}
	\item Consider a measurable space \( X \) with the discrete \( \sigma \)-algebra. Then, for any subset \( A \subseteq X \) we define \( \mu(A) = \abs{A} \) if \( A \) is finite and \( \mu(A) = \infty \) if \( A \) is infinite. This is the \emph{counting measure}, for obvious reasons.
	\item On a finite measurable space \( X \) with the discrete \( \sigma \)-algebra we define for any subset \( A \subseteq X \)
		\begin{equation*}
			\mu(A) = \frac{\abs{A}}{\abs{X}}.
		\end{equation*}
		This is a special case of a probability measure since \( \mu(X) = 1 \). In fact a probability is exactly a measure satisfying \( \mu(X) = 1 \).
	\item On a measurable space \( X \) with any \( \sigma \)-algebra fix a point \( x \in X \) and define \( \mu(A) = 1 \) if \( x \in A \) and \( \mu(A) = 0 \) otherwise. This is called the \emph{Dirac measure}.
	\end{points}
\end{exe}

\section{The Lebesgue measure}
A question worth asking is whether any measurable space can be made into a measure space. Carathódory's extension theorem gives an affirmative answer to the question provided we give a starting point for the measure. Roughly speaking the starting point consists of specifying the measure of a collection of subsetsa, subject to some requirements, which can then be extended to a measure on the whole of the \( \sigma \)-algebra. In the case of \( \R^n \), 

\end{document}
